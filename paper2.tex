%\documentclass[a4paper,10pt]{article}
%\documentclass[draftcls, onecolumn, 11pt]{IEEEtran}
\documentclass[journal]{IEEEtran}

\usepackage{mathbf-abbrevs}
%\newcommand {\tbf}[1] {\textbf{#1}}
%\newcommand {\tit}[1] {\textit{#1}}
%\newcommand {\tmd}[1] {\textmd{#1}}
%\newcommand {\trm}[1] {\textrm{#1}}
%\newcommand {\tsc}[1] {\textsc{#1}}
%\newcommand {\tsf}[1] {\textsf{#1}}
%\newcommand {\tsl}[1] {\textsl{#1}}
%\newcommand {\ttt}[1] {\texttt{#1}}
%\newcommand {\tup}[1] {\textup{#1}}
%
%\newcommand {\mbf}[1] {\mathbf{#1}}
%\newcommand {\mmd}[1] {\mathmd{#1}}
%\newcommand {\mrm}[1] {\mathrm{#1}}
%\newcommand {\msc}[1] {\mathsc{#1}}
%\newcommand {\msf}[1] {\mathsf{#1}}
%\newcommand {\msl}[1] {\mathsl{#1}}
%\newcommand {\mtt}[1] {\mathtt{#1}}
%\newcommand {\mup}[1] {\mathup{#1}}

%some math functions and symbols
\newcommand{\reals}{{\mathbb R}}
\newcommand{\expect}{{\mathbb E}}
\newcommand{\indicator}{{\mathbf 1}}
\newcommand{\ints}{{\mathbb Z}}
\newcommand{\complex}{{\mathbb C}}
\newcommand{\integers}{{\mathbb Z}}
\newcommand{\sign}[1]{\mathtt{sign}\left( #1 \right)}
\newcommand{\NP}{\operatorname{NPt}}
\newcommand{\erf}{\operatorname{erf}}
\newcommand{\NS}{\operatorname{NearestSet}}
\newcommand{\bres}{\operatorname{Bres}}
\newcommand{\vol}{\operatorname{vol}}
\newcommand{\vor}{\operatorname{Vor}}
%\newcommand{\re}{\operatorname{Re}}
%\newcommand{\im}{\operatorname{Im}}




\newcommand{\term}{\emph}
\newcommand{\var}{\operatorname{var}}
%\newcommand{\prob}{{\mathbb P}}
\newcommand{\prob}{{\operatorname{Pr}}}

%distribution fucntions
\newcommand{\projnorm}{\operatorname{ProjectedNormal}}
\newcommand{\vonmises}{\operatorname{VonMises}}
\newcommand{\wrapnorm}{\operatorname{WrappedNormal}}
\newcommand{\wrapunif}{\operatorname{WrappedUniform}}

\newcommand{\selectindicies}{\operatorname{selectindices}}
\newcommand{\sortindicies}{\operatorname{sortindices}}
\newcommand{\largest}{\operatorname{largest}}
\newcommand{\quickpartition}{\operatorname{quickpartition}}
\newcommand{\quickpartitiontwo}{\operatorname{quickpartition2}}

%some commonly used underlined and
%hated symbols
\newcommand{\uY}{\ushort{\mbf{Y}}}
\newcommand{\ueY}{\ushort{Y}}
\newcommand{\uy}{\ushort{\mbf{y}}}
\newcommand{\uey}{\ushort{y}}
\newcommand{\ux}{\ushort{\mbf{x}}}
\newcommand{\uex}{\ushort{x}}
\newcommand{\uhx}{\ushort{\mbf{\hat{x}}}}
\newcommand{\uehx}{\ushort{\hat{x}}}

% Brackets
\newcommand{\br}[1]{{\left( #1 \right)}}
\newcommand{\sqbr}[1]{{\left[ #1 \right]}}
\newcommand{\cubr}[1]{{\left\{ #1 \right\}}}
\newcommand{\abr}[1]{\left< #1 \right>}
\newcommand{\abs}[1]{{\left\vert #1 \right\vert}}
\newcommand{\sabs}[1]{{\vert #1 \vert}}
\newcommand{\floor}[1]{{\left\lfloor #1 \right\rfloor}}
\newcommand{\ceiling}[1]{{\left\lceil #1 \right\rceil}}
\newcommand{\ceil}[1]{\left\lceil #1 \right\rceil}
\newcommand{\round}[1]{{\left\lfloor #1 \right\rceil}}
\newcommand{\magn}[1]{\left\| #1 \right\|}
\newcommand{\fracpart}[1]{\left\langle #1 \right\rangle}
\newcommand{\sfracpart}[1]{\langle #1 \rangle}


% Referencing
\newcommand{\refeqn}[1]{\eqref{#1}}
\newcommand{\reffig}[1]{Figure~\ref{#1}}
\newcommand{\reftable}[1]{Table~\ref{#1}}
\newcommand{\refsec}[1]{Section~\ref{#1}}
\newcommand{\refappendix}[1]{Appendix~\ref{#1}}
\newcommand{\refchapter}[1]{Chapter~\ref{#1}}

\newcommand {\figwidth} {100mm}
\newcommand {\Ref}[1] {Reference~\cite{#1}}
\newcommand {\Sec}[1] {Section~\ref{#1}}
\newcommand {\App}[1] {Appendix~\ref{#1}}
\newcommand {\Chap}[1] {Chapter~\ref{#1}}
\newcommand {\Lem}[1] {Lemma~\ref{#1}}
\newcommand {\Thm}[1] {Theorem~\ref{#1}}
\newcommand {\Cor}[1] {Corollary~\ref{#1}}
\newcommand {\Alg}[1] {Algorithm~\ref{#1}}
\newcommand {\etal} {\emph{~et~al.}}
\newcommand {\bul} {$\bullet$ }   % bullet
\newcommand {\fig}[1] {Figure~\ref{#1}}   % references Figure x
\newcommand {\imp} {$\Rightarrow$}   % implication symbol (default)
\newcommand {\impt} {$\Rightarrow$}   % implication symbol (text mode)
\newcommand {\impm} {\Rightarrow}   % implication symbol (math mode)
\newcommand {\vect}[1] {\mathbf{#1}} 
\newcommand {\hvect}[1] {\hat{\mathbf{#1}}}
\newcommand {\del} {\partial}
\newcommand {\eqn}[1] {Equation~(\ref{#1})} 
\newcommand {\tab}[1] {Table~\ref{#1}} % references Table x
\newcommand {\half} {\frac{1}{2}} 
\newcommand {\ten}[1] {\times10^{#1}}
\newcommand {\bra}[2] {\mbox{}_{#2}\langle #1 |}
\newcommand {\ket}[2] {| #1 \rangle_{#2}}
\newcommand {\Bra}[2] {\mbox{}_{#2}\left.\left\langle #1 \right.\right|}
\newcommand {\Ket}[2] {\left.\left| #1 \right.\right\rangle_{#2}}
\newcommand {\im} {\mathrm{Im}}
\newcommand {\re} {\mathrm{Re}}
\newcommand {\braket}[4] {\mbox{}_{#3}\langle #1 | #2 \rangle_{#4}} 
\newcommand{\dotprod}[2]{ \left\langle #1 , #2 \right\rangle}
\newcommand {\trace}[1] {\text{tr}\left(#1\right)}

% spell things correctly
\newenvironment{centre}{\begin{center}}{\end{center}}
\newenvironment{itemise}{\begin{itemize}}{\end{itemize}}

%%%%% set up the bibliography style
\bibliographystyle{IEEEbib}
%\bibliographystyle{uqthesis}  % uqthesis bibliography style file, made
			      % with makebst

%%%%% optional packages
\usepackage[square,comma,numbers,sort&compress]{natbib}
		% this is the natural sciences bibliography citation
		% style package.  The options here give citations in
		% the text as numbers in square brackets, separated by
		% commas, citations sorted and consecutive citations
		% compressed 
		% output example: [1,4,12-15]

%\usepackage{cite}		
			
\usepackage{units}
	%nice looking units
		
\usepackage{booktabs}
		%creates nice looking tables
		
\usepackage{ifpdf}
\ifpdf
  \usepackage[pdftex]{graphicx}
  %\usepackage{thumbpdf}
  %\usepackage[naturalnames]{hyperref}
\else
	\usepackage{graphicx}% standard graphics package for inclusion of
		      % images and eps files into LaTeX document
\fi

\usepackage{amsmath,amsfonts,amssymb, amsthm, bm} % this is handy for mathematicians and physicists
			      % see http://www.ams.org/tex/amslatex.html

		 
\usepackage[vlined, linesnumbered]{algorithm2e}
	%algorithm writing package
	
\usepackage{mathrsfs}
%fancy math script

%\usepackage{ushort}
%enable good underlining in math mode

%------------------------------------------------------------
% Theorem like environments
%
\newtheorem{theorem}{Theorem}
%\theoremstyle{plain}
\newtheorem{acknowledgement}{Acknowledgement}
%\newtheorem{algorithm}{Algorithm}
\newtheorem{axiom}{Axiom}
\newtheorem{case}{Case}
\newtheorem{claim}{Claim}
\newtheorem{conclusion}{Conclusion}
\newtheorem{condition}{Condition}
\newtheorem{conjecture}{Conjecture}
\newtheorem{corollary}{Corollary}
\newtheorem{criterion}{Criterion}
\newtheorem{definition}{Definition}
\newtheorem{example}{Example}
\newtheorem{exercise}{Exercise}
\newtheorem{lemma}{Lemma}
\newtheorem{notation}{Notation}
\newtheorem{problem}{Problem}
\newtheorem{proposition}{Proposition}
\newtheorem{property}{Property}
\newtheorem{remark}{Remark}
\newtheorem{solution}{Solution}
\newtheorem{summary}{Summary}
%\numberwithin{equation}{section}
%--------------------------------------------------------


\usepackage{xr}
\externaldocument{paper1}

\usepackage{amsmath,amsfonts,amssymb, amsthm, bm}

\usepackage[square,comma,numbers,sort&compress]{natbib}

\newcommand{\sgn}{\operatorname{sgn}}
\newcommand{\sinc}{\operatorname{sinc}}
\newcommand{\rect}[1]{\operatorname{rect}\left(#1\right)}

%opening
\title{Carrier phase and amplitude estimation for phase shift keying using pilots and data (Appendix)}
\author{Robby McKilliam, Andre Pollok, Bill Cowley, I. Vaughan L. Clarkson and Barry Quinn
\thanks{
Supported under the Australian Government’s Australian Space Research Program.
Robby McKilliam, Andre Pollok and Bill Cowley are with the Institute for Telecommunications Research, The University of South Australia, SA, 5095.  Vaughan~Clarkson is with the School of Information Technology \& Electrical Engineering, The University of Queensland, QLD., 4072, Australia.  Barry~Quinn is with the Department of Statistics, Macquarie University, Sydney, NSW, 2109, Australia.
}}

\begin{document}

\maketitle

\begin{abstract}
We consider least squares estimators of carrier phase and amplitude from a noisy communications signal that contains both pilot signals, known to the receiver, and data signals, unknown to the receiver.  We focus on signaling constellations that have symbols evenly distributed on the complex unit circle, i.e., $M$-ary phase shift keying.  We show, under reasonably mild conditions on the distribution of the noise, that the least squares estimator of carrier phase is strongly consistent and asymptotically normally distributed.  However, the amplitude estimator is not consistent, it converges to a positive real number that is a function of the true carrier amplitude, the noise distribution, and the size of the constellation.  Our theoretical results can also be applied to the case where no pilot symbols exist, i.e., noncoherent detection.  The results of Monte Carlo simulations are provided and these agree with the theoretical results.   
\end{abstract}
\begin{IEEEkeywords}
Coherent detection, noncoherent detection, phase shift keying, asymptotic statistics
\end{IEEEkeywords}

%set sections and equation counters based on the previous document.
\setcounter{equation}{21}
\newcounter{lasteqplusone}
\setcounter{lasteqplusone}{\value{equation}}
\addtocounter{lasteqplusone}{1}
\setcounter{section}{9}
\newcounter{lastsecplusone}
\setcounter{lastsecplusone}{\value{section}}
\addtocounter{lastsecplusone}{1}


\section{Introduction to this appendix}

Here we compile lemmas required for the proof of almost sure convergence (Theorem~\ref{thm:consistency}) and asymptotic normality (Theorem~\ref{thm:normality}) from the paper~\cite{McKilliam_leastsqPSKpilotsdata_2012}.  This document is not self contained.  It is intended that the reader start at~\cite{McKilliam_leastsqPSKpilotsdata_2012} and refer to this document when prompted in~\cite{McKilliam_leastsqPSKpilotsdata_2012}.  The equation reference numbers in this document start at (\arabic{lasteqplusone}).  All equation numbers less than (\arabic{lasteqplusone}) refer back to~\cite{McKilliam_leastsqPSKpilotsdata_2012}.  The section numbers in this document start at~\arabic{lastsecplusone} and all sections before~\arabic{lastsecplusone} are contained in~\cite{McKilliam_leastsqPSKpilotsdata_2012}.  This document is separated into 3 sections.  Section~\ref{sec:proof-almost-sureappendix} contains those lemmas required for the proof of almost sure convergence (Theorem~\ref{thm:consistency}).  Section~\ref{sec:proof-asympt-normappendix} contains those lemmas required for the proof of asymptotic normality (Theorem~\ref{thm:normality}).  Section~\ref{sec:an-empirical-process} contains proofs for lemmas that arise in Section~\ref{sec:proof-asympt-normappendix}.  These involve the proof of~\emph{tightness} for certain empirical processes~\cite{Billingsley1999_convergence_of_probability_measures,Dudley_unif_central_lim_th_1999,Shorak_emp_proc_stat_2009}. 

\section{Lemmas required for the proof of almost sure convergence (Theorem~\ref{thm:consistency}) } \label{sec:proof-almost-sureappendix}

\begin{lemma}\label{lem:convtoexpGlamL} 
$\sabs{G(\hat{\lambda}_L)} \rightarrow \sabs{G(0)}$ almost surely as $L \rightarrow \infty$.
\end{lemma}
\begin{IEEEproof}
Since $\sabs{G(x)}$ is uniquely maximised at $x=0$,
\[
0 \leq \sabs{G(0)} - \sabs{G(\hat{\lambda}_L)},
\]
and since $\hat{\lambda}_L$ is the maximiser of $\abs{G_L(x)}$,
\[ 
0 \leq \sabs{G_L(\hat{\lambda}_L)} - \sabs{G_L(0)}.
\]
Thus,
\begin{align*}
0 &\leq \sabs{G(0)} - \sabs{G(\hat{\lambda}_L)} \\ 
& \leq \sabs{G(0)} - \sabs{G(\hat{\lambda}_L)} + \sabs{G_L(\hat{\lambda}_L)} - \sabs{G_L(0)}\\
& \leq |G(0) - G_L(0)| + |G_L(\hat{\lambda}_L) - G(\hat{\lambda}_L)| \\
&\leq 2\sup_{\lambda \in [-\pi, \pi)} \sabs{G_L(\lambda) - G(\lambda)},
\end{align*}
and the last line converges almost surely to zero by Lemma~\ref{lem:uniflawGGL}.
\end{IEEEproof}

% \begin{lemma}\label{lem:Zconunif}
% \[
% \prob\left\{\sup_{\lambda \in [-\pi, \pi)}\abs{G_L(\lambda) - G(\lambda)} > \epsilon \right\} = O(e^{-\epsilon^2 L}).
% \]
% \end{lemma}
% \begin{IEEEproof}
% \end{IEEEproof}

\begin{lemma}\label{lem:GLtoG0}
$G_L(\hat{\lambda}_L) \rightarrow G(0)$ almost surely as $L \rightarrow \infty$.
\end{lemma}
\begin{IEEEproof}
By the triangle inequality,
\[
\sabs{G_L(\hat{\lambda}_L) - G(0)} \leq \sabs{G_L(\hat{\lambda}_L) - G(\hat{\lambda}_L)} + \sabs{G(\hat{\lambda}_L) - G(0)}.
\]
Now $\sabs{G_L(\hat{\lambda}_L) - G(\hat{\lambda}_L)} \rightarrow 0$ as $L \rightarrow \infty$ as a result of Lemma~\ref{lem:uniflawGGL}, and $\sabs{G(\hat{\lambda}_L) - G(0)} \rightarrow 0$ almost surely as $L \rightarrow \infty$ because $G$ is continuous and $\hat{\lambda}_L \rightarrow 0$ almost surely as $L \rightarrow \infty$.
\end{IEEEproof}
 

\begin{lemma}\label{lem:uniflawGGL}
The function $G_L$ converges almost surely and uniformly on $[-\pi, \pi)$ to the function $G$, that is,
\[
\sup_{\lambda \in [-\pi, \pi)}\sabs{G_L(\lambda) - G(\lambda)} \rightarrow 0
\] 
almost surely as $L \rightarrow \infty$.
\end{lemma}
\begin{IEEEproof}
Put $T_L(\lambda) = \expect G_L(\lambda)$ and write
\begin{align*}
&\sup_{\lambda \in [-\pi, \pi)}\sabs{G_L(\lambda) - G(\lambda)} \\
&\leq \sup_{\lambda \in [-\pi, \pi)}\sabs{G_L(\lambda) - T_L(\lambda) + T_L(\lambda) - G(\lambda)} \\
&\leq \sup_{\lambda \in [-\pi, \pi)}\sabs{G_L(\lambda) - T_L(\lambda)} +  \sup_{\lambda \in [-\pi, \pi)}\sabs{T_L(\lambda) - G(\lambda)}.
\end{align*}
Now,
\begin{align*}
T_L(\lambda) - G(\lambda) &= \big(\tfrac{\abs{P}}{L} - p\big)h_1(\lambda) + \big(\tfrac{\abs{D}}{L} - d\big)h_2(\lambda) \\
&= o(1) h_1(\lambda) + o(1) h_2(\lambda)
\end{align*}
where $o(1)$ denotes a number converging to zero as $L \rightarrow \infty$.  Since 
\[
h_1(\lambda) = \expect R_1\cos(\lambda + \Phi_1) \leq \expect R_1,
\]
and 
\[
h_2(\lambda) = \expect R_1\cos\fracpart{\lambda + \Phi_1} \leq \expect R_1
\] 
for all $\lambda \in [-\pi, \pi)$, it follows that 
\[
\sup_{\lambda \in [-\pi, \pi)}\sabs{T_L(\lambda) - G(\lambda)} \leq o(1) \expect R_1 \rightarrow 0
\]  
as $L\rightarrow \infty$.  Lemma~\ref{lem:uniflawTGL} shows that 
\[
\sup_{\lambda \in [-\pi, \pi)}\sabs{G_L(\lambda) - T_L(\lambda)} \rightarrow 0
\]
almost surely as $L\rightarrow \infty$.
\end{IEEEproof}


\begin{lemma}\label{lem:uniflawTGL} 
Put $T_L(\lambda) = \expect G_L(\lambda)$, then
\[
\sup_{\lambda \in [-\pi, \pi)}\sabs{G_L(\lambda) - T_L(\lambda)} \rightarrow 0
\] 
almost surely as $L \rightarrow \infty$.
\end{lemma}
\begin{IEEEproof}
Put $D_L(\lambda) = G_L(\lambda) - T_L(\lambda)$ and let
\[
\lambda_n = \tfrac{2\pi}{N}(n-1) - \pi, \qquad n = 1, \dots, N
\]
be $N$ points uniformly spaced on the interval $[-\pi, \pi)$.  Let $L_n$ be the interval $[\lambda_n, \lambda_n + \tfrac{2\pi}{N})$ and observe that $L_1, \dots, L_N$ partition $[-\pi, \pi)$.  Now
\begin{align*}
\sup_{\lambda \in [-\pi, \pi)}&\sabs{D_L(\lambda)} \\
&= \sup_{n = 1,\dots,N}\sup_{\lambda \in L_n}\sabs{D_L(\lambda) - D_L(\lambda_n) + D_L(\lambda_n)} \\
&\leq U_L + V_L,
\end{align*}
where 
\[
U_L = \sup_{n = 1,\dots,N}\sabs{D_L(\lambda_n)},
\]
\[
V_L = \sup_{n = 1,\dots,N}\sup_{\lambda \in L_n}\sabs{D_L(\lambda) - D_L(\lambda_n)}.
\]
Lemma~\ref{lem:supDLlambdan} shows that for any finite $N$, and any $\epsilon > 0$, 
\[
\prob\left\{ \lim_{L\rightarrow\infty} U_L > \epsilon \right\} = 0,
\]
i.e., $U_L \rightarrow 0$ almost surely as $L\rightarrow\infty$.  Lemma~\ref{lem:supsupDLlambda} shows that for any $\epsilon > 0$,
\[
\prob\left\{ \lim_{L \rightarrow \infty} V_L > \epsilon + \tfrac{4\pi }{N}\expect R_i \right\} = 0.
\] 
Choose $N$ large enough that $4\pi\expect R_i < \epsilon N$ and
\begin{align*}
\prob& \left\{ \lim_{L\rightarrow\infty} \sup_{\lambda \in [-\pi, \pi)}\sabs{D_L(\lambda)} > 3\epsilon \right\} \\
&\leq \prob\left\{\lim_{L\rightarrow\infty}(U_L + V_L) > 3\epsilon \right\} \\
&\leq \prob\left\{\lim_{L\rightarrow\infty} (U_L + V_L) > \epsilon + \epsilon + \tfrac{4\pi }{N}\expect R_i \right\} \\
&\leq \prob\left\{\lim_{L\rightarrow\infty} U_L > \epsilon\right\} +  \prob\left\{ \lim_{L\rightarrow\infty} V_L > \epsilon + \tfrac{4\pi }{N}\expect R_i \right\} \\
&= 0.
\end{align*}
Thus, $\sup_{\lambda \in [-\pi, \pi)}\sabs{D_L(\lambda)} \rightarrow 0$ almost surely as $L\rightarrow\infty$.
\end{IEEEproof}


\begin{lemma}\label{lem:supDLlambdan}
If $N$ is finite then $U_L \rightarrow 0$ almost surely as $L \rightarrow \infty$.
\end{lemma}
\begin{IEEEproof}
Put
\[
Z_i(\lambda) = \begin{cases}
R_i\cos(\lambda + \Phi_i) , & i \in P \\
R_i\cos\sfracpart{\lambda + \Phi_i} , & i \in D
\end{cases}
\]
so that
\[
D_L(\lambda) = G_L(\lambda) - T_L(\lambda) = \frac{1}{L}\sum_{i\in P \cup D} \big( Z_i(\lambda) - \expect Z_i(\lambda) \big).
\]
Now $Z_1(\lambda_n), \dots,Z_L(\lambda_n)$ are independent with finite variance (because $\expect R_i^2$ is finite), so for each $n =1, \dots, N$,
\[
\abs{D_L(\lambda_n)} = \abs{\frac{1}{L}\sum_{i\in P \cup D} \big( Z_i(\lambda_n) - \expect Z_i(\lambda)\big)} \rightarrow 0
\]
almost surely as $L\rightarrow\infty$ by the Kolmogorov's strong law of large numbers~\cite[Theorem 2.3.10]{SenSinger_large_sample_stats_1993}.  Now, since $N$ is finite,
\[
U_L = \sup_{n=1,\dots,N}\sabs{D_L(\lambda_n)} \leq \sum_{n=1}^N \sabs{D_L(\lambda_n)} \rightarrow 0
\]
almost surely to zero as $L \rightarrow \infty$.
\end{IEEEproof}

\begin{lemma}\label{lem:supsupDLlambda} For any $\epsilon > 0$,
\[
\prob\left\{ \lim_{L \rightarrow \infty} V_L > \epsilon + \tfrac{4\pi}{N}\expect R_i \right\} = 0.
\]
\end{lemma}
\begin{IEEEproof}
Observe that
\begin{align*}
&\sabs{D_L(\lambda) - D_L(\lambda_n)} \\
&=  \abs{G_L(\lambda) - T_L(\lambda) - G_L(\lambda_n) + T_L(\lambda_n)} \\
&\leq  \abs{G_L(\lambda) - G_L(\lambda_n)} + \abs{\expect G_L(\lambda) - \expect G_L(\lambda_n)} \\
&\leq \abs{G_L(\lambda) - G_L(\lambda_n) } + \expect \abs{G_L(\lambda) - G_L(\lambda_n)},
\end{align*}
the last line following from Jensen's inequality.  Put
\[
C_L = \sup_{n = 1,\dots,N}\sup_{\lambda \in L_n} \abs{G_L(\lambda) - G_L(\lambda_n) },
\]
so that
\begin{align*}
V_L &= \sup_{n = 1,\dots,N}\sup_{\lambda \in L_n}\sabs{D_L(\lambda) - D_L(\lambda_n)} \\
&\leq  C_L +  \sup_{n = 1,\dots,N}\sup_{\lambda \in L_n} \expect\abs{G_L(\lambda) - G_L(\lambda_n) } \\
&\leq C_L + \expect C_L,
\end{align*}
where the last line follows because $\sup \expect\vert \dots\vert \leq \expect \sup \vert \dots \vert$.  Lemma~\ref{lem:CL} shows that $\expect C_L \leq \tfrac{2\pi}{N}\expect R_1$ and also that
\[
\prob\left\{ \lim_{L \rightarrow \infty} C_L > \epsilon + \tfrac{2\pi}{N}\expect R_1 \right\} = 0.
\]
Thus,
\begin{align*}
\prob&\left\{ \lim_{L \rightarrow \infty} V_L > \epsilon + \tfrac{4\pi}{N}\expect R_1 \right\} \\
&\leq \prob\left\{ \lim_{L \rightarrow \infty} (C_L + \expect C_L) > \epsilon + \tfrac{4\pi}{N}\expect R_1 \right\} \\
&\leq \prob\left\{ \lim_{L \rightarrow \infty} C_L > \epsilon + \tfrac{2\pi}{N}\expect R_1 \right\} = 0.
\end{align*}
\end{IEEEproof}

\begin{lemma}\label{lem:CL} The following statements hold:
\begin{enumerate}
\item $\expect C_L \leq \tfrac{2\pi}{N}\expect R_1$,
\item for any $\epsilon > 0$, $\prob\left\{ \lim_{L \rightarrow \infty} C_L > \epsilon + \tfrac{2\pi}{N}\expect R_1 \right\} = 0$.
\end{enumerate}
\end{lemma}
\begin{IEEEproof}
If $\lambda \in L_n$, then $\lambda = \lambda_n + \delta$ with $\delta < \tfrac{2\pi}{N}$, and from Lemma~\ref{lem:coslipshitz},
\begin{align*}
&\abs{\cos(\lambda + \Phi_i) - \cos(\lambda_n + \Phi_i)} \leq \tfrac{2\pi}{N}, \;\;\; \text{and} \\
&\abs{\cos\sfracpart{\lambda + \Phi_i} - \cos\sfracpart{\lambda_n + \Phi_i}} \leq \tfrac{2\pi}{N}.
\end{align*}
Because this results does not depend on $n$,
\[
\sup_{n=1,\dots,N}\sup_{\lambda \in L_n} \abs{Z_i(\lambda) - Z_i(\lambda_n)} \leq R_i \frac{2\pi}{N}
\]
for all $i = P \cup D$.  Now,
\begin{align*}
C_L &= \sup_{n=1,\dots,N}\sup_{\lambda \in L_n}\abs{\frac{1}{L}\sum_{i \in P \cup D} Z_i(\lambda) - Z_i(\lambda_n)} \\
&\leq \frac{1}{L}\sum_{i \in P \cup D} \sup_{n=1,\dots,N}\sup_{\lambda \in L_n} \abs{Z_i(\lambda) - Z_i(\lambda_n)} \\
&\leq \frac{2\pi}{N L}\sum_{i \in P \cup D} R_i .
\end{align*}
Thus, 
\[
\expect C_L \leq \expect \frac{2\pi}{N L}\sum_{i \in P \cup D} R_i = \frac{2\pi}{N}\expect R_1
\]
and the first statement holds.  Now, 
\[
\frac{2\pi}{N L}\sum_{i \in P \cup D} R_i \rightarrow \frac{2\pi}{N}\expect R_1
\] 
almost surely to $\frac{2\pi}{N}\expect R_1$ as $L \rightarrow\infty$ by the strong law of large numbers, and so, for any $\epsilon > 0$,
\begin{align*}
\prob&\left\{ \lim_{L \rightarrow \infty} C_L > \epsilon + \tfrac{2\pi}{N}\expect R_1 \right\} \\
&\leq \prob\left\{ \lim_{L \rightarrow \infty} \frac{2\pi}{N L}\sum_{i \in P \cup D} R_i > \epsilon + \tfrac{2\pi}{N}\expect R_1 \right\} = 0.
\end{align*}
\end{IEEEproof}



\begin{lemma}\label{lem:coslipshitz}
Let $x$ and $\delta$ be real numbers.  Then
\begin{align*}
&\abs{\cos(x + \delta) - \cos(x)} \leq \abs{\delta}, \;\;\; \text{and} \\
&\abs{\cos\fracpart{x + \delta} - \cos\fracpart{x}} \leq \abs{\delta}.
\end{align*}
\end{lemma}
\begin{IEEEproof}
Both $\cos(x)$ and $\cos\fracpart{x}$ are Lipshitz continuous functions from $\reals$ to $\reals$ with constant $K=1$.  That is, for any $x$ and $y$ in $\reals$,
\begin{align*}
&\abs{\cos(y) - \cos(x)} \leq K\abs{x-y} = \abs{x-y}, \;\;\; \text{and} \\
&\abs{\cos\fracpart{y} - \cos\fracpart{x}} \leq K\abs{x-y} = \abs{x-y}.
\end{align*}
 The lemma follows by putting $y = x + \delta$.
\end{IEEEproof}

\section{Lemmas required for the proof of asymptotic normality (Theorem~\ref{thm:normality}) } \label{sec:proof-asympt-normappendix}

\begin{lemma}~\label{lem:diffatlambdaL}
The derivative of $G_L$ exists, and is equal to zero, at $\hat{\lambda}_L$.  That is,
\[
G_L^\prime(\hat{\lambda}_L) = \frac{d G_L}{d \lambda}(\hat{\lambda}_L) = 0.
\]
\end{lemma}
\begin{IEEEproof}
Observe that 
\[
G_L(\lambda) = \frac{1}{L}\sum_{i \in P} R_i \cos(\lambda + \Phi_i) + \frac{1}{L} \sum_{i \in D} R_i \cos\sfracpart{\lambda + \Phi_i}
\] 
is differentiable everywhere except when $\fracpart{\lambda + \Phi_i} = -\tfrac{\pi}{M}$ for any $i \in D$ with $R_i > 0$.  Let $q_i$ be the smallest number from the interval $[-\tfrac{\pi}{M}, 0]$ such that
\[
 \sin(q_i) > -\frac{\rho_o \sabs{\eta}^2 R_i}{4 L \hat{\rho} \sin(\pi/M)}
\]
where $\eta = e^{-j2\pi/M} - 1$.  Observe that if $R_i > 0$ then $q_i < 0$.  Lemma~\ref{lem:fracpartlambdahatnotpi} shows that,
\[
\sabs{\sfracpart{\hat{\lambda}_L + \Phi_i}} \leq \frac{\pi}{M} + q_i
\]
for all $i \in D$.  Thus, $\sfracpart{\hat{\lambda}_L + \Phi_i} \neq -\tfrac{\pi}{M}$ for $i \in D$ when $R_i > 0$ and therefore $G_L$ is differentiable at $\hat{\lambda}_L$.  

It remains to show that $G_L^\prime(\hat{\lambda}_L) = 0$.  If $\sabs{G_L(\hat{\lambda}_L)} > 0$ then $\sabs{G_L(\lambda)}$ is differentiable at $\hat{\lambda}_L$ and, since $\hat{\lambda}_L$ is a maximiser of $\sabs{G_L(\lambda)}$,
\[
0 = \frac{d \sabs{G_L(\lambda)}}{d \lambda}(\hat{\lambda}_L) = \begin{cases} 
G_L^\prime(\hat{\lambda}_L), & G_L(\hat{\lambda}_L) > 0\\
-G_L^\prime(\hat{\lambda}_L), & G_L(\hat{\lambda}_L) < 0.
\end{cases}
\]
Thus, $G_L^\prime(\hat{\lambda}_L) = 0$ if $\sabs{G_L(\hat{\lambda}_L)} > 0$.  On the other hand $\sabs{G_L(\hat{\lambda}_L)} = 0$ only if $G_L(\lambda) = 0$ for all $\lambda \in \reals$, and in this case $G_L^\prime(\hat{\lambda}_L) = 0$ also. 
\end{IEEEproof}

\begin{lemma}\label{lem:fracpartlambdahatnotpi} Let $q_i, i \in D$ be defined as in Lemma~\ref{lem:diffatlambdaL}.  Then $\sabs{\sfracpart{\hat{\lambda}_L + \Phi_i}} \leq \frac{\pi}{M} + q_i$ for all $i \in D$.
\end{lemma}
\begin{IEEEproof}
Recall that $\{\hat{d}_i = \hat{d}_i(\hat{\theta}), i \in D\}$ defined in \eqref{eq:hatdfinxtheta} are the minimisers of the function 
\[
SS(\{d_i, i \in D\}) = A - \frac{1}{L}\sabs{Y(\{d_i, i \in D\})}^2,
\]
defined in~\eqref{eq:SSdatasymbols}. The proof now proceeds by contradiction.  Assume that 
\begin{equation}\label{eq:fraclambassumption}
\sfracpart{\hat{\lambda}_L + \Phi_k} > \frac{\pi}{M} + q_k
\end{equation}
for some $k \in D$.  Recalling the notation $e_k$ defined in Section~\ref{sec:least-squar-estim}, put $r_i = \hat{d}_i e_k$.  We will show that 
\[
SS(\{r_i, i \in D\}) < SS(\{\hat{d}_i, i \in D\}),
\]
violating the fact that $\{\hat{d}_i, i \in D\}$ are minimisers of $SS$.  First observe that,
\begin{align*}
Y(\{ r_i, i \in D\}) = \sum_{i \in P} y_ip_i^* + \sum_{i \in D} y_i\hat{r}_i^* = \hat{Y} + \eta y_k\hat{d}_k^*,
\end{align*}
where $\eta = e^{-j2\pi/M} - 1$ and $\hat{Y} = Y(\{ \hat{d}_i, i \in D\})$.  Now,
\begin{align*}
SS&(\{r_i, i \in D\}) \\
&= A - \frac{1}{L} \sabs{ Y(\{ r_i, i \in D\}) }^2 \\
&= A - \frac{1}{L} \sabs{ \hat{Y} + \eta y_k\hat{d}_k^* }^2 \\
&= A - \frac{1}{L}\sabs{\hat{Y}}^2 - \frac{2}{L}\Re\left(\eta \hat{Y}^* y_k \hat{d}_k^* \right) -  \frac{1}{L}\sabs{ \eta y_k}^2\\
&= SS(\{\hat{d}_i, i \in D\}) - D,
\end{align*}
where 
\[
D = \frac{2}{L}\Re\left(\eta \hat{Y}^* y_k \hat{d}_k^* \right) +  \frac{1}{L}\sabs{ \eta y_k }^2.
\]
Now $\frac{1}{L}\hat{Y} = \hat{a} = \hat{\rho} e^{j\hat{\theta}}$ from~\eqref{eq:hata} and using~\eqref{eq:yethetadhat},
\[
\frac{1}{L} \hat{Y}^* y_k \hat{d}_k^* = \hat{\rho} \rho_0 R_k e^{j\sfracpart{\hat{\lambda}_L + \Phi_k}},
\]
so that
\begin{equation}\label{eq:Ddefn}
D = 2 \hat{\rho} \rho_0 R_k \Re\left( \eta e^{j\sfracpart{\hat{\lambda}_L + \Phi_k}}\right) + \frac{1}{L}\sabs{\eta}^2\rho_0^2 R_k^2.
\end{equation}
Let $v = \sfracpart{\hat{\lambda}_L + \Phi_k} - \tfrac{\pi}{M}$ so that
\begin{align*}
\eta e^{j\sfracpart{\hat{\lambda}_L + \Phi_k}} &= (e^{-j2\pi/M} - 1) e^{j \pi/M} e^{jv} \\
&= (e^{-j\pi/M} - e^{j\pi/M}) e^{jv} \\
&= -2 \sin(\tfrac{\pi}{M}) e^{j ( v + \pi/2) },
\end{align*}
and
\begin{align*}
\Re\left( \eta e^{j\sfracpart{\hat{\lambda}_L + \Phi_k}}\right) &= -2 \sin(\tfrac{\pi}{M}) \cos(v + \tfrac{\pi}{2}) \\
 &= 2 \sin(\tfrac{\pi}{M}) \sin(v),
\end{align*}
Because we assumed~\eqref{eq:fraclambassumption}, it follows that $0 > v > q_k$ and, from the definition of $q_k$,
\[
0 > \sin(v) > -\frac{\rho_o \sabs{\eta}^2 R_i}{4 L \hat{\rho} \sin(\pi/M)}.
\]
Substituting this into~\eqref{eq:Ddefn} gives $D > 0$, but then $SS(\{r_i, i \in D\}) < SS(\{\hat{d}_i, i \in D\})$, violating the fact that $\{\hat{d}_i, i \in D\}$ are minimisers of $SS$.  Thus,~\eqref{eq:fraclambassumption} is false.

To show that $\sfracpart{\hat{\lambda}_L + \Phi_k} \geq -\frac{\pi}{M} - q_k$ we use contradiction again.  Assume that $\sfracpart{\hat{\lambda}_L + \Phi_k} < -\frac{\pi}{M} - q_k$.  Recalling the notation $e_k^*$ defined in Section~\ref{sec:least-squar-estim}, put $r_i = \hat{d}_i e_k^*$.  Now an analogous argument can be used to show that $SS(\{r_i, i \in D\}) < SS(\{\hat{d}_i, i \in D\})$ again.

\end{IEEEproof}


\begin{lemma}\label{lem:Qconv}
Let $Q_L(\lambda) = \expect R_L(\lambda) - \expect R_L(0)$ where the function $R_L$ is defined in~\eqref{eq:RLdef}.  Then,
\[ 
\sqrt{L} Q_L(\hat{\lambda}_L) = \sqrt{L} \hat{\lambda}_L\big( p + Hd  + o_P(1) \big)
\]
where $p, d$ and $H$ are real numbers defined in the statements of Theorems~\ref{thm:consistency}~and~\ref{thm:normality} and $o_P(1)$ denotes a number converging in probability to zero as $L\rightarrow\infty$.
\end{lemma}
\begin{IEEEproof}
We have
\[
Q_L(\lambda) = \expect R_L(\lambda) - \expect R_L(0) = \tfrac{\sabs{P}}{L} k_1(\lambda) + \tfrac{\sabs{D}}{L} k_2(\lambda)
\]
where
\begin{align}
k_1(\lambda) &= \expect R_1\big( \sin(\lambda + \Phi_1) - \sin(\Phi_1) \big), \;\;\; \text{and} \nonumber\\
k_2(\lambda) &= \expect R_1\big( \sin\fracpart{\lambda + \Phi_1} - \sin\fracpart{\Phi_1} \big). \label{eq:k2def}
\end{align}
We analyse $k_1(\lambda)$ first.  By a Taylor expansion of $\sin(\lambda + \Phi_1)$ about $\lambda = 0$,
\[
\sin(\lambda + \Phi_1) - \sin(\Phi_1) = \lambda \big( \cos(\Phi_1) + \zeta(\lambda) \big),
\]
where $\zeta$ is a continuous function with $\zeta(0) = 0$.  So,
\begin{align*}
k_1(\lambda) &= \expect R_1\lambda(\cos(\Phi_1) \; + \; \zeta(\lambda)) \\
&= \lambda \big( 1 \; + \; \zeta(\lambda) \expect R_1 \big) 
\end{align*}
since $h_1(0) = \expect R_1\cos(\Phi_1) = 1$ as a result of~\eqref{eq:expectRepartRphi}.  Because $\hat{\lambda}_L$ converges almost surely to zero as $L \rightarrow\infty$, then $\zeta(\hat{\lambda}_L)$ converges almost surely (and therefore also in probability) to zero as $L \rightarrow\infty$, i.e. $\zeta(\hat{\lambda}_L) = o_P(1)$.  So, 
\[
k_1(\hat{\lambda}_L) = \hat{\lambda}_L \big( 1 \; + \; \zeta(\hat{\lambda}_L) \expect R_1 \big) = \hat{\lambda}_L \big( 1  + o_P(1)\big)
\]
because $\expect R_1$ is finite.  Lemma~\ref{lem:k2conv} shows that $k_2(\hat{\lambda}_L) = \hat{\lambda}_L ( H + o_P(1) )$.  So,
\begin{align*}
Q_L(\hat{\lambda}_L) &=  \tfrac{\sabs{P}}{L} \hat{\lambda}_L \big( 1  + o_P(1)\big)  + \tfrac{\sabs{D}}{L} \hat{\lambda}_L ( H + o_P(1) ) \\
&= \hat{\lambda}_L \left( \tfrac{\sabs{P}}{L}  + \tfrac{\sabs{D}}{L} H + o_P(1) \right) \\
&= \hat{\lambda}_L \left( p  + d H + o_P(1) \right),
\end{align*}
since $\frac{\sabs{P}}{L} \rightarrow p$ and $\frac{\sabs{D}}{L} \rightarrow d$ as $L \rightarrow \infty$.  The lemma follows by multiplying both sides of the above equation by $\sqrt{L}$.
\end{IEEEproof}

\begin{lemma}\label{lem:k2conv}
$k_2(\hat{\lambda}_L) = \hat{\lambda}_L ( H + o_P(1) ).$
\end{lemma}
\begin{IEEEproof}
Because $\hat{\lambda}_L \rightarrow 0$ almost surely as $L\rightarrow\infty$, it is only the behaviour of $k_2(\lambda)$ around $\lambda = 0$ that is relevant.  We will analyse $k_2(\lambda)$ for $0 \leq \lambda < \tfrac{\pi}{M}$.  An analogous argument follows when $-\tfrac{\pi}{M} < \lambda < 0$.  To keep our notation clean put
\[
\psi_k = \tfrac{2\pi}{M}k + \tfrac{\pi}{M}.
\]
For each integer $k$, if $\Phi_1 \in [\psi_{k-1}, \psi_k)$, then
\[
\fracpart{\Phi_1} = \Phi_1 - \round{\Phi_1} = \Phi_1 - \tfrac{2\pi}{M}k.
\]
Similarly, if $\Phi_1 \in  [\psi_{k-1}, \psi_{k} - \lambda)$, then
\[
\fracpart{\Phi_1 + \lambda} = \Phi_1 + \lambda - \tfrac{2\pi}{M}k,
\]
whilst, if $\Phi_1 \in [\psi_{k} - \lambda, \psi_{k})$, then
\[
\fracpart{\Phi_1 + \lambda} = \Phi_1 + \lambda - \tfrac{2\pi}{M}k - \tfrac{2\pi}{M}.
\]
So, when $\Phi_1 \in [\psi_{k-1}, \psi_{k} - \lambda )$,
\begin{align*}
\sin\fracpart{\lambda + \Phi_1} - &\sin\fracpart{\Phi_1} \\
&= \sin(\lambda + \Phi_1 - \tfrac{2\pi}{M}k) - \sin(\Phi_1 - \tfrac{2\pi}{M}k),
\end{align*}
and by a Taylor expansion of $\sin(\lambda + \Phi_1 - \tfrac{2\pi}{M}k)$ about $\lambda = 0$, 
\begin{align*}
\sin\fracpart{\lambda + \Phi_1} - \sin\fracpart{\Phi_1} &= \lambda\big(\cos(\Phi_1 - \tfrac{2\pi}{M}k) + \zeta_k(\lambda)\big) \\
&= \lambda\big(\cos\fracpart{\Phi_1} + \zeta_k(\lambda)\big).
\end{align*}
where, for each $k$, the function $\zeta_k$ is continuous with $\zeta_k(0) = 0$.  Alternatively, when $\Phi_1 \in [ \psi_{k} - \lambda,\psi_{k})$,
\begin{align*}
\sin\fracpart{\lambda + \Phi_1} &- \sin\fracpart{\Phi_1} \\
&= \sin(\lambda + \Phi_1 - \tfrac{2\pi}{M}k - \tfrac{2\pi}{M}) - \sin(\Phi_1 - \tfrac{2\pi}{M}k)
\end{align*}
and by a Taylor expansion of $\sin(\lambda + \Phi_1 - \tfrac{2\pi}{M}k - \tfrac{2\pi}{M})$ about $\lambda = 0$,
\begin{align*}
\sin&\fracpart{\lambda + \Phi_1} - \sin\fracpart{\Phi_1} \\
&= s_k(\Phi_1) + \lambda\big(\cos\fracpart{\Phi_1} + \eta_k(\lambda, \Phi_1)  )\big)
\end{align*}
where
\[
s_k(\Phi_1) = \sin(\Phi_1 - \tfrac{2\pi}{M}k - \tfrac{2\pi}{M}) - \sin(\Phi_1 - \tfrac{2\pi}{M}k),
\]
and
\[
\eta_k(\lambda, \Phi_1) = \cos(\Phi_1 - \tfrac{2\pi}{M}k - \tfrac{2\pi}{M}) - \cos\fracpart{\Phi_1} + \zeta_{k+1}(\lambda).
\]
Observe that both $s_k$ and $\eta_k$ are continuous and that 
\[
s_k(\psi_{k}) = -2\sin(\tfrac{\pi}{M}) \;\;\; \text{and} \;\;\; \eta_k(\lambda,\psi_k) = \zeta_{k+1}(\lambda).
\]

Let us now collate what we have.  For each integer $k$, if $\Phi_1 \in [\psi_{k-1}, \psi_{k} - \lambda )$, then
\[
\sin\fracpart{\lambda + \Phi_1} - \sin\fracpart{\Phi_1} = \lambda\cos\fracpart{\Phi_1} + \lambda\zeta_k(\lambda),
\]
whilst, if $\Phi_1 \in [ \psi_{k} - \lambda,\psi_{k})$, then
\[
\sin\fracpart{\lambda + \Phi_1} - \sin\fracpart{\Phi_1} = \lambda \cos\fracpart{\Phi_1} + s_k(\Phi_1) + \lambda\eta_k(\lambda, \Phi_1).
\]
Since the intervals $[\psi_{k-1}, \psi_{k})$ partition the real line, that is $\reals = \cup_{k\in\ints}[\psi_{k-1}, \psi_{k})$, it follows that for all $\Phi_1 \in \reals$,
\[
\sin\fracpart{\lambda + \Phi_1} - \sin\fracpart{\Phi_1} = \lambda\cos\fracpart{\Phi_1} +  v(\lambda, \Phi_1),
\]
where $v(\lambda, \Phi_1) = \sum_{k\in\ints} v_k(\lambda, \Phi_1)$ and
\begin{equation}\label{eq:vk}
v_k(\lambda,\Phi_1) = \begin{cases}
s_k(\Phi_1) + \lambda\eta_k(\lambda, \Phi_1) &  \Phi_1 \in [ \psi_{k} - \lambda,\psi_{k} ) \\
\lambda\zeta_k(\lambda) & \Phi_1 \in [\psi_{k-1}, \psi_{k} - \lambda ) \\
0 & \text{otherwise}.
\end{cases} 
\end{equation}
Now,
\begin{align*}
k_2(\lambda) &=  \expect R_1\big( \sin\fracpart{\lambda + \Phi_1} - \sin\fracpart{\Phi_1} \big) \\
&= \expect R_1\big( \lambda\cos\fracpart{\Phi_1} + v(\lambda, \Phi_1)  \big) \\
&= \lambda h_2(0) + \expect R_1 v(\lambda, \Phi_1).
\end{align*}
Let $W(\lambda) = \expect R_1 v(\lambda, \Phi_1)$ so that $k_2(\lambda) = \lambda h_2(0) + W(\lambda).$  Lemma~\ref{lem:expRvlamphi} shows that $W(\lambda) =  \lambda(H_1 + \gamma(\lambda))$ where $\gamma(\lambda)$ is a continuous function with $\gamma(0) = 0$, and
\[
H_1 = - 2 \sin(\tfrac{\pi}{M}) \sum_{k = 0}^{M-1} g(\psi_k) = H - h_2(0),
\]
where $H$ is defined in the statement of Theorem~\ref{thm:normality}.  Because $\hat{\lambda}_L$ converges almost surely to zero as $L \rightarrow\infty$, it follows that $\gamma(\hat{\lambda}_L) = o_P(1)$, and so
\[
k_2(\hat{\lambda}_L) = \hat{\lambda}_L ( h_2(0) + H_1 + o_P(1) ) = \hat{\lambda}_L ( H + o_P(1) )
\]
as required.  It remains to prove Lemma~\ref{lem:expRvlamphi}.
\end{IEEEproof}

\begin{lemma}\label{lem:expRvlamphi}
$W(\lambda) =  \lambda(H_1 + \gamma(\lambda))$ where $\gamma$ is a continuous function with $\gamma(0) = 0$.
\end{lemma}
\begin{IEEEproof}
In this lemma, as in Lemma~\ref{lem:k2conv}, we work under the assumption that $0 \leq \lambda < \frac{\pi}{M}$.  An analogous argument follows when $-\tfrac{\pi}{M} \leq \lambda < 0$.  Recalling that $f(r,\phi)$ is the joint pdf of $R_1$ and $\Phi_1$, 
\begin{align}
W(\lambda) &= \expect R_1 v(\lambda, \Phi_1) \nonumber  \\
&= \int_0^{2\pi}\int_{0}^\infty r v(\lambda, \phi) f(r,\phi) dr d\phi \nonumber  \\ 
&= \int_0^{2\pi} v(\lambda, \phi) g(\phi) d\phi \nonumber \\
&= \sum_{k \in \ints} \int_0^{2\pi} v_k(\lambda, \phi) g(\phi) d\phi, \label{eq:Winfsum}
\end{align}
where $g$ is defined in the statement of Theorem~\ref{thm:normality}
Now, for $k = 1, \dots, M-1$ we have $0 < \psi_{k-1} < \psi_k < 2\pi$, and so, using~\eqref{eq:vk},
\begin{align*}
\int_0^{2\pi} v_k(\lambda, \phi) &g(\phi) d\phi = \int_{\psi_{k-1}}^{\psi_{k}} v_k(\lambda, \phi) g(\phi) d\phi \\
&= \lambda a_k(\lambda) + \lambda b_k(\lambda) + \lambda c_k(\lambda),
\end{align*}
where, for $k = 1, \dots, M-1$,
\begin{equation}\label{eq:ak}
a_k(\lambda) = \int_{\psi_k-\lambda}^{\psi_k}\eta_k(\lambda, \phi) g(\phi) d\phi,
\end{equation}
\[
b_k(\lambda) = \zeta_k(\lambda) \int_{\psi_{k-1}}^{\psi_k -\lambda}g(\phi) d\phi,
\]
\begin{equation}\label{eq:ck}
c_k(\lambda) = \frac{1}{\lambda} \int_{\psi_{k} - \lambda}^{\psi_{k}} s_k(\phi) g(\phi) d\phi.
\end{equation}
Both $a_k(\lambda)$ and $b_k(\lambda)$ converge to zero as $\lambda$ goes to zero, and, since $g$ is continuous at $\psi_k = \tfrac{2\pi}{M}k + \tfrac{\pi}{M}$ for each $k = 0, 1, \dots, M-1$ by Proposition~\ref{prop:gg2cont} in~\cite{McKilliam_leastsqPSKpilotsdata_2012}, it follows that $c_k(\lambda)$ converges to 
\[
s_k(\psi_k)g(\psi_k) = -2\sin(\tfrac{\pi}{M}) g(\tfrac{2\pi}{M}k + \tfrac{\pi}{M})
\]
as $\lambda$ goes to zero from above.  We are only interested in the limit from above because we are working under the assumption that $0 \leq \lambda < \frac{\pi}{M}$.  The analogous argument when $-\tfrac{\pi}{M} \leq \lambda < 0$ would involve limits as $\lambda$ approaches zero from below.

Now, when $k = 0$, we have $\psi_{-1} < 0 < \psi_0 - \lambda < \psi_0 < 2\pi$, so that
\[
\int_0^{2\pi} v_0(\lambda, \phi) g(\phi) d\phi = \lambda a_0(\lambda) + \lambda b_0(\lambda) +  \lambda c_0(\lambda),
\]
where $a_0(\lambda)$ and $c_0(\lambda)$ are defined as in~\eqref{eq:ak}~and~\eqref{eq:ck}, and
\[
b_0(\lambda) = \zeta_0(\lambda) \int_{0}^{\psi_k-\lambda} g(\phi) d\phi.
\]
When $k = M$, we have $0 < \psi_{M-1} < 2\pi < \psi_{M} - \lambda$, so that
\[
\int_0^{2\pi} v_{M}(\lambda, \phi) g(\phi) d\phi = \lambda b_{M}(\lambda)
\]
where 
\[
b_{M}(\lambda) = \zeta_{M}(\lambda) \int_{\psi_{M-1}}^{2\pi}g(\phi) d\phi.
\]
Both $b_0(\lambda)$ and $b_M(\lambda)$ are functions that converge to zero as $\lambda$ converges to zero.
 
Finally, when $k < 0$ or $k > M$, 
\[
\int_0^{2\pi} v_k(\lambda, \phi) g(\phi) d\phi = 0.
\]
So the infinite sum in~\eqref{eq:Winfsum} can be written as the finite sum,
\begin{align*}
W(\lambda) &= \sum_{k=0}^{M} \int_{0}^{2\pi} v_k(\lambda, \phi) g(\phi) d\phi \\
&= \sum_{k=0}^{M-1} \lambda c_k(\lambda) + \sum_{k=0}^{M-1} \lambda a_k(\lambda) + \sum_{k=0}^{M} \lambda b_k(\lambda) \\
&= \lambda \left( C(\lambda)  + A(\lambda) + B(\lambda) \right),
\end{align*}
where 
\[
A(\lambda) = \sum_{k=0}^{M-1} a_k(\lambda), \;\;\; B(\lambda) = \sum_{k=0}^{M} b_k(\lambda), \;\;\; \text{and},
\]
\[
C(\lambda) = \sum_{k=0}^{M-1} c_k(\lambda).
\]
Both $A(\lambda)$ and $B(\lambda)$ converge to zero as $\lambda$ goes to zero, whilst $C(\lambda)$ converges to $H_1$ as $\lambda$ converges to zero.  Put
\[
\gamma(\lambda) = A(\lambda) + B(\lambda) + C(\lambda) - H_1.
\]
Then $\gamma(\lambda)$ converges to zero as $\lambda$ goes to zero and $W(\lambda) = \lambda(H_1 + \gamma(\lambda))$ as required.
\end{IEEEproof}

\begin{lemma}\label{lem:empiricprocc} Let $Q_L(\lambda) = \expect R_L(\lambda) - \expect R_L(0)$ where the function $R_L$ is defined in~\eqref{eq:RLdef}.  Then,
\[
\sqrt{L}\big( R_L(\hat{\lambda}_L) - Q_L(\hat{\lambda}_L) \big) = o_P(1) + \sqrt{L} R_L(0).
\]
\end{lemma}
\begin{IEEEproof}
Put 
\begin{equation}\label{eq:WLdef}
W_L(\lambda) = \sqrt{L}\big( R_L(\lambda) - Q_L(\lambda) - R_L(0) \big)
\end{equation}
so that
\[
\sqrt{L}\big( R_L(\hat{\lambda}_L) - Q_L(\hat{\lambda}_L) \big) = W_L(\hat{\lambda}_L) + \sqrt{L} R_L(0)
\]
Lemma~\ref{lem:ZYLempiricproc} in Section~\ref{sec:an-empirical-process} shows that for any $\delta > 0$ and $\nu > 0$, there exists $\epsilon > 0$ such that
\[
\Pr\left\{ \sup_{\sabs{\lambda}<\epsilon}\sabs{W_L(\lambda)} > \delta  \right\} < \nu
\]
for all positive integers $L$.  Since $\hat{\lambda}_L$ converges almost surely to zero, it follows that for any $\epsilon > 0$,
\[
\lim_{L\rightarrow\infty}\prob\left\{ \sabs{\hat{\lambda}_L} \geq \epsilon \right\} = 0
\] 
and therefore $\prob\{ \sabs{\hat{\lambda}_L} \geq \epsilon \} < \nu$ for all sufficiently large $L$.  Now
\begin{align*}
  \prob&\left\{\sabs{ W_L(\hat{\lambda}_L) } > \delta \right\} \\
&= \prob\left\{ \sabs{W_L(\hat{\lambda}_L)} > \delta \;\text{and} \; \sabs{\hat{\lambda}_L} < \epsilon \right\} \\
&\hspace{0.7cm} + \prob\left\{ \sabs{W_L(\lambda_L)} > \delta  \;\text{and} \; \sabs{\hat{\lambda}_L} \geq \epsilon \right\} \\
&\leq \prob\left\{  \sup_{\lambda < \epsilon} \sabs{ W_L(\lambda) } > \delta \right\} + \prob\left\{ \sabs{\hat{\lambda}_L} \geq \epsilon \right\} \\
&\leq 2\nu.
\end{align*}
for all sufficiently large $L$.  Since $\nu$ and $\delta$ can be chosen arbitrarily small, it follows that $W_L(\hat{\lambda}_L)$ converges in probability to zero as $N\rightarrow\infty$.
\end{IEEEproof}




\begin{lemma}\label{lem:convdistGLdash}
The distribution of $\sqrt{L}R_L(0)$ converges to the normal with zero mean and variance $pA_1 + dA_2$.
\end{lemma}
\begin{IEEEproof}
Observe that $\sqrt{L} R_L(0) = C_L + D_L$ where
\[
C_L = \frac{1}{\sqrt{L}} \sum_{i \in P} R_i \sin(\Phi_i), \qquad D_L = \frac{1}{\sqrt{L}} \sum_{i \in D} R_i \sin\sfracpart{\Phi_i}.
\]
From the standard central limit theorem the distribution of $C_L$ converges to the normal with mean $\sqrt{p}\expect R_1 \sin(\Phi_1) = 0$ as a result of~\eqref{eq:expectImpartRphi}, and variance
\[
p A_1 = p \expect R_1^2 \sin^2(\Phi_1).
\]
Similarly, the distribution of $D_L$ converges to the normal with mean $\sqrt{d}\expect R_1 \sin\fracpart{\Phi_1} = 0$ as a result of Lemma~\ref{lem:expectImRfracpart}, and variance
\[
d A_2 = d \expect R_1^2 \sin^2\fracpart{\Phi_1}.
\]
The lemma holds because $C_L$ and $D_L$ are independent. 
\end{IEEEproof}

\begin{lemma}\label{lem:expectImRfracpart}
$\expect R_1 \sin\fracpart{\Phi_1} = 0$.
\end{lemma}
\begin{IEEEproof}
Recalling that $f(r,\phi)$ is the joint pdf of $R_1$ and $\Phi_1$, we have
\begin{align*}
\expect R_1 \sin\fracpart{\Phi_1} &= \int_{0}^{2\pi} \int_{0}^\infty r \sin\fracpart{\phi} f(r,\phi) dr d\phi \\
&= \int_{-\pi}^{\pi} \sin\fracpart{\phi} g(\phi) d\phi,
\end{align*}
where $g$ is defined in the statement of Theorem~\ref{thm:normality} and the change in bounds from $0, 2\pi$ to $-\pi,\pi$ is valid because both $\sin\fracpart{\phi}$ and $g(\phi)$ have period $2\pi$.  Proposition~\ref{prop:gg2cont} in \cite{McKilliam_leastsqPSKpilotsdata_2012} shows that $g$ is even on $(-\pi,\pi)$.  Also $\sin\fracpart{\cdot}$ has period $\tfrac{2\pi}{M}$ and is odd on $(-\frac{\pi}{M}, \tfrac{\pi}{M})$.  So, for any $k \in \ints$ and $\phi \in (-\tfrac{\pi}{M}, 0)$,
\[
g(-\phi + k \tfrac{\pi}{M}) = g(\phi - k \tfrac{\pi}{M})
\]
and
\[
\sin\fracpart{-\phi + k \tfrac{\pi}{M}} = -\sin\fracpart{\phi - k \tfrac{\pi}{M}}.
\]
Now,
\begin{align*}
I_k &= \int_{-(k+1)\tfrac{\pi}{M}}^{-k\tfrac{\pi}{M}}\sin\fracpart{\phi} g(\phi) d\phi + \int_{k\tfrac{\pi}{M}}^{(k+1)\tfrac{\pi}{M}}\sin\fracpart{\phi} g(\phi) d\phi \\
&= \int_{-\tfrac{\pi}{M}}^{0}\sin\fracpart{\phi -  k\tfrac{\pi}{M}} g(\phi - k\tfrac{\pi}{M}) d\phi \\
&\hspace{1cm}+ \int_{0}^{\tfrac{\pi}{M}}\sin\fracpart{\phi + k\tfrac{\pi}{M}} g(\phi + k\tfrac{\pi}{M}) d\phi \\
&= 0
\end{align*}
for all integers $k$ because
\begin{align*}
\int_{0}^{\tfrac{\pi}{M}}&\sin\fracpart{\phi + k\tfrac{\pi}{M}} g(\phi + k\tfrac{\pi}{M}) d\phi \\
&= \int_{-\tfrac{\pi}{M}}^{0}\sin\fracpart{-\phi + k\tfrac{\pi}{M}} g(-\phi + k\tfrac{\pi}{M}) d\phi \\
&= -\int_{-\tfrac{\pi}{M}}^{0}\sin\fracpart{\phi - k\tfrac{\pi}{M}} g(\phi - k\tfrac{\pi}{M}) d\phi.
\end{align*}
Observe that the sets
\[
 [-(k+1)\tfrac{\pi}{M},-k\tfrac{\pi}{M}] \cup [k\tfrac{\pi}{M},(k+1)\tfrac{\pi}{M}], \;\;\; k = 0, \dots, M-1
\]
partition $[-\pi, \pi]$, and so
\[
\expect R_1 \sin\fracpart{\Phi_1} = \int_{-\pi}^{\pi} \sin\fracpart{\phi} g(\phi) d\phi = \sum_{k=0}^{M-1} I_k = 0.
\]
\end{IEEEproof}

\begin{lemma}\label{lem:empiricprocforrho} Let $T_L(\lambda) = \expect G_L(\lambda)$.  Then
\[
\sqrt{L}\big( G_L(\hat{\lambda}_L) - T_L(\hat{\lambda}_L) \big) = o_P(1) + X_L,
\]
where $X_L = \sqrt{L} \big( G_L(0) - T_L(0) \big)$.
\end{lemma}
\begin{IEEEproof}
Let
\begin{equation}\label{eq:YLdef}
Y_L(\lambda) = \sqrt{L}\big( G_L(\hat{\lambda}_L) - T_L(\hat{\lambda}_L) \big) - X_L
\end{equation}
so that
\[
\sqrt{L}\big( G_L(\hat{\lambda}_L) - T_L(\hat{\lambda}_L) \big) = Y_L(\hat{\lambda}_L) + X_L
\]
Lemma~\ref{lem:ZYLempiricproc} in Section~\ref{sec:an-empirical-process} shows that for any $\delta > 0$ and $\nu > 0$, there exists $\epsilon > 0$ such that
\[
\Pr\left\{ \sup_{\sabs{\lambda}<\epsilon}\sabs{Y_L(\lambda)} > \delta  \right\} < \nu.
\]
The proof now follows by an argument analogous to that in Lemma~\ref{lem:empiricprocc}.
\end{IEEEproof}

\begin{lemma}\label{lem:HLtoG} $\sqrt{L}\big( T_L(\hat{\lambda}_L) - G(0) \big) = o_P(1)$.
\end{lemma}
\begin{IEEEproof}
The argument is similar to that used in Lemma~\ref{lem:Qconv}.  First observe that
\begin{align*}
T_L(\lambda) &= \frac{\sabs{P}}{L}\expect R_1\cos(\lambda + \Phi_1) + \frac{\sabs{D}}{L} \expect R_1\cos\fracpart{\lambda + \Phi_1} \\
%&= \big( p + o(1) \big) \expect R_1\cos(\lambda + \Phi_1) \\
%& \hspace{2cm} + \big( d + o(1) \big) \expect R_1\cos\fracpart{\lambda + \Phi_1} \\
&=  \big( p + o(L^{-1/2}) \big) \expect R_1\cos(\lambda + \Phi_1) \\
&\hspace{2cm} + \big( d + o(L^{-1/2}) \big) \expect R_1\cos\fracpart{\lambda + \Phi_1},
\end{align*}
because $\tfrac{\abs{P}}{L} = p + o(L^{-1/2})$ and $\tfrac{\abs{D}}{L} = d + o(L^{-1/2})$ is an assumption in the statement of Theorem~\ref{thm:normality}.  Now,
\begin{align*}
\sqrt{L}\big( T_L(\lambda) - G(0) \big) = p q_1(\lambda) + d q_2(\lambda) + o(1),
\end{align*}
where
\[
q_1(\lambda) = \expect R_1\big( \cos(\lambda + \Phi_1) - \cos(\Phi_1) \big),
\]
\begin{equation}
q_2(\lambda) = \expect R_1\big( \cos\fracpart{\lambda + \Phi_1} - \cos\fracpart{\Phi_1} \big). \label{eq:q2def}
\end{equation}
We analyse $q_1(\lambda)$ first.  By a Taylor expansion of $\cos(\lambda + \Phi_1)$ about $\lambda = 0$,
\[
\cos(\lambda + \Phi_1) - \cos(\Phi_1) = -\lambda \big( \sin(\Phi_1) + \zeta(\lambda) \big),
\]
where $\zeta$ is a continuous function with $\zeta(0) = 0$. So,
\begin{align*}
q_1(\lambda) &= -\expect R_1\lambda(\sin(\Phi_1) +  \zeta(\lambda)) \\
&= -\lambda \big( \expect R_1\sin(\Phi_1) +  \zeta(\lambda) \expect R_1 \big) \\
&= -\lambda \zeta(\lambda) \expect R_1,
\end{align*}
since $\expect R_1\sin(\Phi_1) = 0$ as a result of~\eqref{eq:expectImpartRphi}.  Now, 
\begin{align*}
\sqrt{L} q_1(\hat{\lambda}_L) = -\sqrt{L} \hat{\lambda}_L \zeta(\hat{\lambda}_L) \expect R_1 = o_P(1)
\end{align*}
because $\sqrt{L} \hat{\lambda}_L$ converges in distribution and $\zeta(\hat{\lambda}_L) = o_P(1)$.  An argument analogous to that used in Lemma~\ref{lem:k2conv} shows that $\sqrt{L} q_2(\hat{\lambda}_L) = o_P(1)$.  Thus, 
\[
\sqrt{L}\big( T_L(\lambda) - G(0) \big) = p q_1(\lambda) + d q_2(\lambda) + o(1) = o_P(1)
\]
as required.
\end{IEEEproof}


% \begin{lemma}\label{lem:XL} 
% $\sqrt{L} q_2(\hat{\lambda}_L) = o_P(1)$.
% \end{lemma}
% \begin{IEEEproof}
% The argument is similar to that used in Lemma~\ref{lem:k2conv}.  As $\hat{\lambda}_L \in [-\pi, \pi)$ it is only the behaviour of the function $q_2(\lambda)$ for $\lambda\in [-\pi, \pi)$ that is relevant.  We will analyse $q_2(\lambda)$ for $0 \leq \lambda < \tfrac{\pi}{M}$.  An analagous argument follows when $-\tfrac{\pi}{M} \leq \lambda < 0$.  Recall that
% \[
% \psi_k = \tfrac{2\pi}{M}k + \tfrac{\pi}{M}.
% \]
% when $\Phi_1 \in [\psi_{k-1}, \psi_{k} - \lambda )$,
% \begin{align*}
% \cos\fracpart{\lambda + \Phi_1} - &\cos\fracpart{\Phi_1} \\
% &= \cos(\lambda + \Phi_1 - \tfrac{2\pi}{M}k) - \cos(\Phi_1 - \tfrac{2\pi}{M}k),
% \end{align*}
% and by a Taylor expansion of $\cos(\lambda + \Phi_1 - \tfrac{2\pi}{M}k)$ about $\lambda = 0$, 
% \begin{align*}
% \cos\fracpart{\lambda + \Phi_1} - \cos\fracpart{\Phi_1} &= -\lambda\big(\sin(\Phi_1 - \tfrac{2\pi}{M}k) + \zeta_k(\lambda)\big) \\
% &= -\lambda\big(\sin\fracpart{\Phi_1} + \zeta_k(\lambda)\big).
% \end{align*}
% where, for each integer $k$, the function $\zeta_k$ is continuous with $\zeta_k(0) = 0$.   Alternatively, when $\Phi_1 \in [ \psi_{k} - \lambda,\psi_{k})$,
% \begin{align*}
% \cos\fracpart{\lambda + \Phi_1} &- \cos\fracpart{\Phi_1} \\
% &= \cos(\lambda + \Phi_1 - \tfrac{2\pi}{M}k - \tfrac{2\pi}{M}) - \cos(\Phi_1 - \tfrac{2\pi}{M}k)
% \end{align*}
% and by a Taylor expansion of $\cos(\lambda + \Phi_1 - \tfrac{2\pi}{M}k - \tfrac{2\pi}{M})$ about $\lambda = 0$,
% \begin{align*}
% \cos&\fracpart{\lambda + \Phi_1} - \sin\fracpart{\Phi_1} \\
% &= c_k(\Phi_1) - \lambda\big(\sin\fracpart{\Phi_1} + \eta_k(\lambda, \Phi_1)  )\big)
% \end{align*}
% where
% \[
% c_k(\Phi_1) = \cos(\Phi_1 - \tfrac{2\pi}{M}k - \tfrac{2\pi}{M}) - \cos(\Phi_1 - \tfrac{2\pi}{M}k),
% \]
% and
% \[
% \eta_k(\lambda, \Phi_1) = \sin(\Phi_1 - \tfrac{2\pi}{M}k - \tfrac{2\pi}{M}) - \sin(\Phi_1 - \tfrac{2\pi k}{M}) + \zeta_{k+1}(\lambda).
% \]
% Observe that $c_k$ is continuous and $c_k(\psi_{k}) = 0$.

% Let us now collate what we have.  For each integer $k$, if $\Phi_1 \in [\psi_{k-1}, \psi_{k} - \lambda )$, then
% \[
% \cos\fracpart{\lambda + \Phi_1} - \cos\fracpart{\Phi_1} = -\lambda\sin\fracpart{\Phi_1} + \lambda\zeta_k(\lambda),
% \]
% whilst, if $\Phi_1 \in [ \psi_{k} - \lambda,\psi_{k})$, then
% \[
% \cos\fracpart{\lambda + \Phi_1} - \cos\fracpart{\Phi_1} = -\lambda \sin\fracpart{\Phi_1} + c_k(\Phi_1) + \lambda\eta_k(\lambda, \Phi_1).
% \]
% Since the intervals $[\psi_{k-1}, \psi_{k})$ partion the real line, that is $\reals = \cup_{k\in\ints}[\psi_{k-1}, \psi_{k})$, it follows that for all $\Phi_1 \in \reals$,
% \[
% \cos\fracpart{\lambda + \Phi_1} - \cos\fracpart{\Phi_1} = -\lambda\sin\fracpart{\Phi_1} +  v(\lambda, \Phi_1),
% \]
% where $v(\lambda, \Phi_1) = \sum_{k\in\ints} v_k(\lambda, \Phi_1)$ and
% \begin{equation}\label{eq:cvk}
% v_k(\lambda,\Phi_1) = \begin{cases}
% c_k(\Phi_1) - \lambda\eta_k(\lambda, \Phi_1) &  \Phi_1 \in [ \psi_{k} - \lambda,\psi_{k} ) \\
% \lambda\zeta_k(\lambda) & \Phi_1 \in [\psi_{k-1}, \psi_{k} - \lambda ) \\
% 0 & \text{otherwise}.
% \end{cases}
% \end{equation}
% \end{IEEEproof}



\begin{lemma}\label{lem:XL} 
The distribution of 
\[
X_L = \sqrt{L} \big( G_L(0) - T_L(0) \big) = \sqrt{L} \big( G_L(0) - \expect G_L(0) \big)
\] 
converges to the normal with zero mean and covariance 
\[
pB_1 + d B_2
\] 
as $L \rightarrow\infty$.
\end{lemma}
\begin{IEEEproof}
Observe that $X_L = C_L + D_L$ where
\[
C_L = \frac{1}{\sqrt{L}} \sum_{i \in P} \big( R_i \cos(\Phi_i) - h_1(0) \big),
\]
\[
D_L = \frac{1}{\sqrt{L}} \sum_{i \in D} (R_i \cos\fracpart{\Phi_i} - h_2(0) ),
\]
where $h_1$ and $h_2$ are defined in the statement of Theorem~\ref{thm:consistency}.  From the standard central limit theorem the distribution of $C_L$ converges to the normal with zero mean and variance
\[
p B_1 = p \expect R_1^2 \cos^2(\Phi_1) - p h_1^2(0) =  p \expect R_1^2 \cos^2(\Phi_1) - p 
\]
since $h_1(0) = 1$.  Similarly the distribution of $D_L$ converges to the normal with zero mean and variance
\[
d B_2 = d \expect R_1^2 \cos^2\fracpart{\Phi_1} - d h_2^2(0).
\]
The lemma follows since $C_L$ and $D_L$ are independent.
\end{IEEEproof}

%\section{Tricks with fractional parts}
%Throughout the paper, and particularly in  have made use of a number of results involving the fractional part function

\section{An empirical process result}\label{sec:an-empirical-process}

During the proof of asymptotic normality in Section~\ref{sec:proof-asympt-norm} we made use of the following result regarding the functions $W_L(\lambda)$ and $Y_L(\lambda)$ defined in~\eqref{eq:WLdef}~and~\eqref{eq:YLdef}.

\begin{lemma}\label{lem:ZYLempiricproc}
For any $\delta > 0$ and $\nu > 0$, there exists $\epsilon > 0$ such that:
\begin{enumerate}
\item $\Pr\left\{ \sup_{\sabs{\lambda}<\epsilon}\sabs{W_L(\lambda)} > \delta  \right\} < \nu$, and,
\item $\Pr\left\{ \sup_{\sabs{\lambda}<\epsilon}\sabs{Y_L(\lambda)} > \delta  \right\} < \nu$,
\end{enumerate}
for all positive integers $L$.
\end{lemma}
\begin{IEEEproof}
These results are related to what is called \emph{tightness} in the literature on empirical processes and weak convergence on metric spaces~\cite{Billingsley1999_convergence_of_probability_measures,Dudley_unif_central_lim_th_1999,Shorak_emp_proc_stat_2009}.  The proof is based on a technique called \emph{symmetrisation} and another technique called \emph{chaining} (also known as \emph{bracketing})~\cite{Pollard_asymp_empi_proc_1989,Gine_Zinn_symmetrisation_1984,Ossiander_clt_bracketing_1984}.  We will only give a proof of the first statement regarding the function $W_L$.  The proof for the second statement regarding $Y_L$ is similar.  

Put 
\[
f_i(\lambda, R_i, \Phi_i) = \begin{cases}
R_i( \sin(\lambda + \Phi_i) - \sin(\Phi_i)), & i \in P \\
R_i( \sin\sfracpart{\lambda + \Phi_i} - \sin\sfracpart{\Phi_i}), & i \in D \\
\end{cases}
\]
so that
\[
W_L(\lambda) = \frac{1}{\sqrt{L}} \sum_{i \in P \cup D} \big( f_i(\lambda, R_i, \Phi_i) - \expect f_i(\lambda, R_i, \Phi_i) \big).
\]
Let $\{g_i\}$ be a sequence of independent standard normal random variables, independent of $\{R_i\}$ and $\{\Phi_i\}$.  Lemma~\ref{lem:symmetrisation} shows that
\[
\expect \sup_{\sabs{\lambda} < \epsilon} \abs{ W_L(\lambda)} \leq \sqrt{2\pi} \; \expect \sup_{\sabs{\lambda} < \epsilon}  \abs{ X(\lambda) },
\]
where 
\begin{equation}\label{eq:ZpsiCondGaussProc}
X(\lambda) = \frac{1}{\sqrt{L}} \sum_{i \in P \cup D} g_i f_i(\lambda, R_i, \Phi_i),
\end{equation}
and where $\expect$ runs over all of $\{g_i\}$, $\{R_i\}$ and $\{\Phi_i\}$.  Conditionally on $\{\Phi_i\}$ and $\{R_i\}$, $\{X(\lambda), \lambda \in \reals \}$ is a \emph{Gaussian process}, and numerous techniques exist for its analysis.  Lemma~\ref{lem:chaining} shows that for any $\kappa > 0$ there exists $\epsilon > 0$ such that
\[
\expect \sup_{\sabs{\lambda} < \epsilon} \abs{ X(\lambda) } < \kappa.
\]
It follows that,
\[
\expect \sup_{\sabs{\lambda} < \epsilon} \abs{ W_L(\lambda)}  <  \sqrt{2\pi} \; \kappa,
\]
and by Markov's inequality,
\[
\prob \cubr{  \sup_{\sabs{\lambda} < \epsilon} \abs{ W_L(\lambda)} > \delta } \leq  \sqrt{2\pi} \; \frac{\kappa}{\delta},
\]
for any $\delta > 0$.  The proof follows by choosing $\nu =  \sqrt{2\pi} \kappa/\delta$.  It remains to prove Lemmas~\ref{lem:symmetrisation}~and~\ref{lem:chaining}.
\end{IEEEproof}


\begin{lemma}\label{lem:symmetrisation} (Symmetrisation)
\[
\expect \sup_{\sabs{\lambda} < \epsilon} \abs{ W_L(\lambda)} \leq \sqrt{2\pi} \; \expect \sup_{\sabs{\lambda} < \epsilon}  \abs{ X(\lambda) }
\]
\end{lemma}
\begin{IEEEproof}
Let $\{R_i'\}$ be a sequence identically distributed to $\{R_i\}$ and let $\{\Phi_i'\}$ be a sequence identically distributed to $\{\Phi_i\}$.  Both $\{R_i'\}$ and $\{\Phi_i'\}$ are chosen independently of $\{R_i\}$ and $\{\Phi_i\}$.  Let $\expect_{R,\Phi}$ denote expectation conditional on $\{R_i\}$ and $\{\Phi_i\}$.  Now,
\begin{align*}
W_L(\lambda) &= \frac{1}{\sqrt{L}} \sum_{i \in P \cup D} \big( f_i(\lambda, R_i, \Phi_i) - \expect_{R,\Phi} f_i(\lambda, R_i', \Phi_i') \big) \\
&= \expect_{R,\Phi} \frac{1}{\sqrt{L}} \sum_{i \in P \cup D} ( f_{i} - f_{i}^\prime ),
\end{align*}
where, for notational convenience, we put 
\[
f_{i} = f_i(\lambda, R_i, \Phi_i) \qquad \text{and} \qquad  f_{i}^\prime = f_i(\lambda, R_i', \Phi_i').
\] 
Taking absolute values followed by supremums,
\begin{align*}
 \sup_{\sabs{\lambda} < \epsilon} \abs{ W_L(\lambda)} &= \sup_{\sabs{\lambda} < \epsilon}  \abs{\frac{1}{\sqrt{L}} \expect_{R,\Phi} \sum_{i \in P \cup D} (f_{i} - f_{i}^\prime )} \\
&\leq \sup_{\sabs{\lambda} < \epsilon} \expect_{R,\Phi}  \abs{ \frac{1}{\sqrt{L}} \sum_{i \in P \cup D}  (f_{i} - f_{i}^\prime) },
\end{align*}
the upper bound following from Jensen's inequality.  Since $\sup \expect \abs{\dots} \leq \expect \sup \abs{\dots}$, the bound can be increased by moving $\expect_{R,\Phi}$ outside the supremum,
\begin{equation}\label{eq:Eoutsidesup}
 \sup_{\sabs{\lambda} < \epsilon} \abs{ W_L(\lambda)} \leq \expect_{R,\Phi} \sup_{\sabs{\lambda} < \epsilon} \abs{ \frac{1}{\sqrt{L}}\sum_{i \in P \cup D} ( f_{i} - f_{i}^\prime ) }.
\end{equation}
Applying $\expect$ to both sides gives the inequality
\begin{equation}\label{eq:Gninequfin}
 \expect \sup_{\sabs{\lambda} < \epsilon} \abs{ W_L(\lambda)} \leq  \expect \sup_{\sabs{\lambda} < \epsilon} \abs{ \frac{1}{\sqrt{L}}\sum_{i \in P \cup D} (f_{i} - f_{i}^\prime) },
\end{equation}
since $\expect \expect_{R,\Phi}$ is equivalent to $\expect$.   Let
\[
\sigma_i = \frac{g_i}{\abs{g_i}},
\]
and put $\sigma_i = 1$ if $g_i = 0$.  The $\sigma_i$ are thus independent with 
\[
\prob\{ \sigma_i = -1 \} = \tfrac{1}{2} \qquad \text{and} \qquad  \prob\{ \sigma_i = 1 \} = \tfrac{1}{2}.
\]  
As $R_i$ and $\Phi_i$ are independent and identically distributed to $R_i^\prime$ and $\Phi_i^\prime$, the random variable $f_{i} - f_{i}^\prime$ is symmetrically distributed about zero, and is therefore distributed identically to $\sigma_i( f_{i} - f_{i}^\prime)$.  Thus,
%\[
%\sum_{n=1}^{N}  (f_{nN} - f_{nN}^\prime) \qquad \text{and} \qquad \sum_{n=1}^{N}  \sigma_n(f_{nN} - f_{nN}^\prime)
%\]
%are identically distributed, and therefore
\begin{align*}
 \expect \sup_{\sabs{\lambda} < \epsilon}  &\abs{  \frac{1}{\sqrt{L}}\sum_{i \in P \cup D}  (f_{i} - f_{i}^\prime) } \\
&= \expect \sup_{\sabs{\lambda} < \epsilon}  \abs{  \frac{1}{\sqrt{L}}\sum_{i \in P \cup D} \sigma_i ( f_{i} - f_{i}^\prime) }.
\end{align*}
By the triangle inequality,
\[
\abs{ \sum_{i\in P \cup D} \sigma_i (f_{i} - f_{i}^\prime)} \leq \abs{ \sum_{i \in P \cup D} \sigma_i f_{i} } + \abs{ \sum_{i\in P \cup D}  \sigma_i f_{i}^\prime },
\]
and it follows from~\eqref{eq:Gninequfin}, that
\begin{align*}
 \expect \sup_{\sabs{\lambda} < \epsilon} \abs{ W_L(\lambda)} &\leq \expect \sup_{\sabs{\lambda} < \epsilon}  \abs{ \frac{1}{\sqrt{L}}\sum_{i\in P \cup D} \sigma_i f_{i} } \\
&\hspace{1cm} + \expect \sup_{\sabs{\lambda} < \epsilon}  \abs{\frac{1}{\sqrt{L}} \sum_{i \in P \cup D}  \sigma_i  f_{i}^\prime } \\
&= 2 \; \expect \sup_{\sabs{\lambda} < \epsilon}  \abs{ \frac{1}{\sqrt{L}}\sum_{i \in P \cup D} \sigma_i f_{i} }.
\end{align*}
Because $g_i$ is a standard normal random variable it follows that $\expect \abs{g_i} = \expect_{R,\Phi} \abs{g_i} = \sqrt{2/\pi}$, and
\begin{align*}
\expect \sup_{\sabs{\lambda} < \epsilon}  &\abs{ \frac{1}{\sqrt{L}}\sum_{i \in P \cup D} \sigma_i f_{i} } \\
&= \expect \sup_{\sabs{\lambda} < \epsilon}  \abs{ \frac{1}{\sqrt{L}}\sum_{i \in P \cup D} \sigma_i f_{i} \sqrt{\frac{\pi}{2}}\expect_{R,\Phi} \abs{g_i}} \\
&\leq \sqrt{\frac{\pi}{2}} \expect \sup_{\sabs{\lambda} < \epsilon}  \abs{ \frac{1}{ \sqrt{L}}\sum_{i \in P \cup D} \sigma_i \abs{g_i}  f_{i}},
\end{align*}
the last line following from Jensen's inequality and by moving $\expect_{R,\Phi}$ outside the supremum similarly to~\eqref{eq:Eoutsidesup}. As $g_i = \sigma_i \abs{g_i}$,
\begin{align*}
\expect \sup_{\sabs{\lambda} < \epsilon} \abs{W_L(\lambda)}  &\leq 2\sqrt{\frac{\pi}{2}} \expect \sup_{\sabs{\lambda} < \epsilon}  \abs{ \frac{1}{\sqrt{L}}\sum_{i\in P \cup D} g_i  f_{i}} \\
&= \sqrt{2\pi} \; \expect \sup_{\sabs{\lambda} < \epsilon}  \abs{ X(\lambda) } 
\end{align*}
as required.
\end{IEEEproof}

\begin{lemma} \label{lem:chaining}
For any $\kappa > 0$ there exists $\epsilon > 0$ such that
\[
\expect \sup_{\sabs{\lambda} < \epsilon} \abs{ X(\lambda) } < \kappa,
\]
where $X(\lambda)$ is the Gaussian process defined in~\eqref{eq:ZpsiCondGaussProc}.
\end{lemma}
\begin{IEEEproof}
Without loss of generality, assume that $\epsilon < \tfrac{\pi}{M}$.  Lemma~\ref{lem:chaining2} shows that
\begin{equation}\label{eq:supZCK1} 
\expect_{R,\Phi} \sup_{\sabs{\lambda} < \epsilon} \abs{ X(\lambda) } \leq K_1 \sqrt{C_\epsilon}
\end{equation}
where $K_1$ is a finite, positive constant, and
\begin{equation}\label{eq:Cepsdefn}
C_\epsilon = \frac{2}{L}\epsilon\sum_{i\in P \cup D}R_i^2 + \frac{4}{L}\sum_{i\in P \cup D}R_i^2 I_{\epsilon}(\Phi_i)
\end{equation}
where $I_\epsilon$ is the indicator function
\[
I_{\epsilon}(x) = \begin{cases}
1, & x \in \cup_{k\in\ints}[\psi_k - \epsilon, \psi_k + \epsilon] \\
0, & \text{otherwise},
\end{cases}
\] 
and $\psi_k = \tfrac{2\pi}{M}k + \tfrac{\pi}{M}$.  Now,
\[
\expect C_\epsilon =  2\epsilon\expect R_1^2 + 4 \expect R_i^2 I_{\epsilon}(\Phi_i).
\]
Recalling that $f(r,\phi)$ is the joint probability density function of $R_i$ and $\Phi_i$,
\begin{align*}
\expect R_i^2 I_\epsilon(\Phi_i) &= \int_0^{2\pi} \int_{0}^\infty  r^2 f(r,\phi) I_\epsilon(\phi) dr d\phi \\ 
&= \int_0^{2\pi}g_2(\phi) I_\epsilon(\phi) d\phi 
\end{align*}
where the function
\[
g_2(\phi) = \int_{0}^\infty  r^2 f(r,\phi) dr d\phi 
\]
is continuous at $\psi_k = \tfrac{2\pi}{M}k + \tfrac{\pi}{M}$ for each $k = 0, 1, \dots, M-1$ by Proposition~\ref{prop:gg2cont}.  From the definition of $I_\epsilon$,
\begin{align*}
\expect R_i^2 I_\epsilon(\Phi_i) &= \sum_{k=0}^{M-1}\int_{\psi_k-\epsilon}^{\psi_k-\epsilon}g_2(\phi) d\phi \\
&= 2\epsilon \sum_{k=0}^{M-1} (g_2(\psi_k) + o(1))
\end{align*}
where $o(1)$ goes to zero as $\epsilon$ goes to zero, by the mean value theorem.  Thus $\expect R_i^2 I_\epsilon(\Phi_i) \leq K_2\epsilon$ for some constant $K_2$. Since $\sqrt{\cdot}$ is a concave function on the positive real line, and since $C_{\epsilon}$ is always positive,
\[
\expect \sqrt{C_\epsilon} \leq  \sqrt{\expect  C_\epsilon} < \sqrt{\epsilon( K_2 + 2\expect R_1^2)  }
\]
by Jensen's inequality.  Applying $\expect$ to both sides of~\eqref{eq:supZCK1},
\[
\expect \sup_{\sabs{\lambda} < \epsilon} \abs{ X(\lambda) } \leq K_1 \sqrt{\tfrac{1}{L}\expect C_\epsilon} < K_1 \sqrt{\epsilon( K_2 + 2\expect R_1^2)}.
\]
Choosing 
\[
\epsilon = \frac{\kappa^2}{ K_1^2 ( K_2 + 2\expect R_1^2)}
\] 
completes the proof.  It remains to prove Lemma~\ref{lem:chaining2}.
\end{IEEEproof}

The proofs of Lemmas~\ref{lem:chaining2}~and~\ref{lem:chaining3} are based on a technique called \emph{chaining} (or \emph{bracketing})~\cite{Dudley_unif_central_lim_th_1999,Ossiander_clt_bracketing_1984,Pollard_asymp_empi_proc_1989,Pollard_new_ways_clts_1986}.  The proofs here follow those of Pollard~\cite[Section 3]{Pollard_asymp_empi_proc_1989}.  %Within the remaining proofs, we are interested in expectations conditional on the phase noise sequence $\{\Phi_n\}$, so for the purpose of these lemmas, $\{\Phi_n\}$ is treated as a fixed realisation.  
For notational convenience we use the abbreviation 
\[
f_{i}(\lambda) = f_{i}(\lambda, R_i, \Phi_i)
\]
in what follows.  As in Lemma~\ref{lem:chaining} we assume, without loss of generality, that $\epsilon < \tfrac{\pi}{M}$.  %This comes at no loss since Lemma~\ref{lem:chaining} only asks for the existence of an $\epsilon > 0$.

\begin{lemma}\label{lem:chaining2}
There exists a positive constant $K_1$ such that
\[
\expect_{R,\Phi} \sup_{\sabs{\lambda} < \epsilon} \abs{ X(\lambda) } \leq K_1 \sqrt{C_\epsilon}.
\]
where $C_\epsilon$ is defined in~\eqref{eq:Cepsdefn}
\end{lemma}
\begin{IEEEproof}
Let $B_\epsilon = \{x \in \reals \mid \abs{x} < \epsilon\}$ be the set of real numbers with magnitude less than $\epsilon$.  For each non negative integer $k$, let $T_\epsilon(k)$ be a discrete subset of $B_\epsilon$ with the property that for every $\lambda \in B_\epsilon$ there exists $\lambda_1$ and $\lambda_2$ in $T_\epsilon(k)$ such that the pseudometric
\[
d_1(\lambda, \lambda_1) = \sum_{i \in P \cup D} \big( f_{i}(\lambda) - f_{i}(\lambda_1) \big)^2 \leq 2^{-k} C_\epsilon L,
\]
and the pseudometric
\[
d_2(\lambda, \lambda_2) = \sum_{i \in P \cup D} \sabs{f_{i}(\lambda) - f_{i}(\lambda_2)} \leq 2^{-k} D_\epsilon L
\]
where 
\begin{equation}\label{eq:Depsdefn}
D_\epsilon = \frac{1}{L}\epsilon\sum_{i\in P \cup D}R_i + \frac{2}{L}\sum_{i\in P \cup D}R_i I_{\epsilon}(\Phi_i).
\end{equation}
We specifically define $T_\epsilon(0)$ to contain only zero.  Defined this way, $T_\epsilon(0)$ satisfies the inequality above because $d_1(\lambda, 0) \leq C_\epsilon L $ and $d_2(\lambda, 0) \leq D_\epsilon L$ for all $\lambda \in B_\epsilon$, as a result of Lemma~\ref{lem:epslmlemma}.  
The existence of $T_\epsilon(k)$ for every positive integer $k$ will be proved by construction in Lemma~\ref{lem:metricentropy}.  Lemma~\ref{lem:metricentropy} also shows that the number of elements in $T_\epsilon(k)$ is no more that $K_3 2^k$ where $K_3$ is a constant, independent of $L$, $\epsilon$ and $k$.

Lemma~\ref{lem:limexpTk} shows that
\[
\lim_{k\rightarrow\infty}\expect_{R,\Phi}\sup_{\lambda \in T_\epsilon(k)} \sabs{X(\lambda)} = \expect_{R,\Phi} \sup_{\sabs{\lambda} < \epsilon} \sabs{X(\lambda)}.
\]
Lemma~\ref{lem:chaining3} shows that
\[
\expect_{R,\Phi} \sup_{\lambda \in T_\epsilon(k)} \sabs{X(\lambda)} \leq \sqrt{C_\epsilon} \sum_{i=1}^k \frac{5\sqrt{i \log 2 + \log K_3}}{2^{i/2}}
\]
The sum above converges as $k\rightarrow\infty$ so lemma holds with
\[
K_1 = \sum_{i=1}^\infty \frac{5\sqrt{i \log 2 + \log K_3}}{2^{i/2}}.
\]
\end{IEEEproof}

\begin{lemma}\label{lem:epslmlemma}
Both $d_1(\lambda, 0) \leq C_\epsilon L $ and $d_2(\lambda, 0) \leq D_\epsilon L$ for all $\lambda \in B_\epsilon$.
\end{lemma}
\begin{IEEEproof}
We prove the lemma for $d_2$.  The proof for $d_1$ is similar.  Observe that $d_2(\lambda, 0) = c_1 + c_2$ where 
 \[
 c_1 = \sum_{i \in P} R_i\sabs{\sin(\lambda + \Phi_i) - \sin(\Phi_i) },
 \] 
 \[
 c_2 = \sum_{i \in D} R_i \sabs{\sin\fracpart{\lambda + \Phi_i} - \sin\fracpart{\Phi_i}}.
 \]
 As $\sin(\cdot)$ is Lipshitz continuous with constant $K=1$,
 \[
 \sabs{\sin(\lambda + \Phi_i) - \sin(\Phi_i)} \leq K\sabs{\lambda} = \sabs{\lambda} < \epsilon
 \]
 for all $i \in P \cup D$, and therefore $c_1 \leq \epsilon \sum_{i \in P} R_i$.  Let 
\[
S = \{ i \in D \mid I_\epsilon(\Phi_i) = 1\}.
\] 
Now $c_2 = c_3 + c_4$ where 
 \[
 c_3 = \sum_{i \in D, i \in S} R_i \sabs{\sin\fracpart{\lambda + \Phi_i} - \sin\fracpart{\Phi_i}},
 \]
 \[
 c_4 =  \sum_{i \in D, i \notin S} R_i \sabs{\sin\fracpart{\lambda + \Phi_i} - \sin\fracpart{\Phi_i}}.
 \]
Observe that $I_\epsilon(\Phi_i) = 1$ if and only if $\abs{\fracpart{\Phi_i}} \geq \tfrac{\pi}{M} - \epsilon$.  So, if $i \notin S$ and $\sabs{\lambda} < \epsilon$, then
\begin{align*}
\round{\lambda + \Phi_i} &= \round{\lambda + \fracpart{\Phi_i} + \round{\Phi_i}} \\ 
&= \round{\lambda + \fracpart{\Phi_i}} + \round{\Phi_i} = \round{\Phi_i}
\end{align*}
because
\[
\sabs{\lambda + \fracpart{\Phi_i}} < \epsilon + \tfrac{\pi}{M} - \epsilon = \tfrac{\pi}{M}
\] 
and therefore $\round{\lambda + \fracpart{\Phi_i}} = 0$.  Now,
 \begin{align*}
 \vert\sin&\fracpart{\lambda + \Phi_i} - \sin\fracpart{\Phi_i}\vert \\
 &= \sabs{\sin(\lambda + \Phi_i - \round{\Phi_i}) - \sin(\Phi_i - \round{\Phi_i})} \\
 &\leq \sabs{\lambda} < \epsilon
 \end{align*}
and so, 
\[
c_4 \leq \epsilon \sum_{i \in D, i \notin S} R_i \leq \epsilon \sum_{i \in D} R_i.
\]
Otherwise, if $i \in S$, then $\sabs{\sin\fracpart{\lambda + \Phi_i} - \sin\fracpart{\Phi_i}} \leq 2$ because $\sin(\cdot) \leq 1$.  Thus 
\[
c_3 \leq 2 \sum_{i \in D, i \in S} R_i \leq 2 \sum_{i \in P \cup D} R_i I_\epsilon(\Phi_i).
\]
Now,
\begin{align*}
d_2(\lambda, 0) &= c_1 + c_2 = c_1 + c_3 + c_4 \\
&\leq \epsilon \sum_{i \in P \cup D} R_i + 2 \sum_{i \in P \cup D} R_i I_\epsilon(\Phi_i) = D_\epsilon L
\end{align*}
as required
\end{IEEEproof}


\begin{lemma}\label{lem:limexpTk} With $T_\epsilon(k)$ defined as in Lemma~\ref{lem:chaining2},
\[
\lim_{k\rightarrow\infty}\expect_{R,\Phi}\sup_{\lambda \in T_\epsilon(k)} \sabs{X(\lambda)} = \expect_{R,\Phi} \sup_{\sabs{\lambda} < \epsilon} \sabs{X(\lambda)}
\]
\end{lemma}
\begin{IEEEproof}
Let $b_2(\lambda)$ be a function that maps $\lambda \in B_\epsilon$ to a point in $T_\epsilon(k)$ such that  $d_2(\lambda,b_2(\lambda)) \leq 2^{-k}D_\epsilon L$.  The existence of $b_2$ is guaranteed by the definition of $T_\epsilon(k-1)$.  So, for all $\lambda \in B_\epsilon$,
\begin{align*}
\sabs{X(\lambda) - X(b_2(\lambda))} &= \abs{\frac{1}{\sqrt{L}}\sum_{i\in P \cup D}g_i(f_i(\lambda) - f_i(b_2(\lambda)))} \\
&\leq  \frac{1}{\sqrt{L}}\sum_{i\in P \cup D}\abs{g_i}\sabs{f_i(\lambda) - f_i(b_2(\lambda))} \\
&\leq \frac{1}{\sqrt{L}} \sum_{i\in P \cup D} g_\text{max} \sabs{f_i(\lambda) - f_i(b_2(\lambda))} \\
&\leq \frac{g_\text{max}}{\sqrt{L}} d_2(\lambda,b_2(\lambda)) \\
&\leq \frac{g_\text{max}}{\sqrt{L}} 2^{-k} D_\epsilon L,
\end{align*}
where $g_\text{max} = \sup_{i\in P\cup D} \abs{g_i}$. The bound applies uniformly over $\lambda \in B_\epsilon$, so
\[
\sup_{\sabs{\lambda} < \epsilon} \sabs{X(\lambda) - X(b_2(\lambda))} \leq \frac{g_\text{max}}{\sqrt{L}} 2^{-k} D_\epsilon L.
\]
Applying $\expect_{R,\Phi}$ to both sides,
\[
\expect_{R,\Phi} \sup_{\sabs{\lambda} < \epsilon} \sabs{X(\lambda) - X(b_2(\lambda))} \leq K_4 2^{-k} D_\epsilon L,
\]
where, as a result of Lemma~\ref{lem:maxineq},
\[
K_4 = \tfrac{1}{\sqrt{L}}\expect_{R,\Phi} g_\text{max} \leq 3 \sqrt{\frac{\log L}{L}}
\]
is independent of $\epsilon$ and $k$.  Because $T_\epsilon(k)$ is a subset of $B_\epsilon$,
\[
\sup_{\lambda \in T_\epsilon(k)} \expect_{R,\Phi} \sabs{X(\lambda)} \leq \sup_{\sabs{\lambda} < \epsilon} \expect_{R,\Phi} \sabs{X(\lambda)},
\]
and so,
\begin{align*}
0 &\leq \sup_{\sabs{\lambda} < \epsilon} \expect_{R,\Phi} \sabs{X(\lambda)} - \sup_{\lambda \in T_\epsilon(k)} \expect_{R,\Phi} \sabs{X(\lambda)} \nonumber \\
&\leq \sup_{\sabs{\lambda} < \epsilon} \expect_{R,\Phi} \sabs{X(\lambda)} - \sup_{\sabs{\lambda} < \epsilon} \expect_{R,\Phi} \sabs{X(b_2(\lambda))} \nonumber \\
&\leq \sup_{\sabs{\lambda} < \epsilon} \expect_{R,\Phi} \big( \sabs{X(\lambda)} - \sabs{X(b_2(\lambda))} \big) \nonumber \\
&\leq \sup_{\sabs{\lambda} < \epsilon} \expect_{R,\Phi} \sabs{X(\lambda) - X(b_2(\lambda))} \nonumber \\
&\leq K_4 2^{-k} D_\epsilon L. \label{eq:Deps2nktozero}
\end{align*}
The second line in the inequality above follows because,
\[
\sup_{\sabs{\lambda} < \epsilon} \expect_{R,\Phi} \sabs{X(b_2(\lambda))} \leq \sup_{\lambda \in T_\epsilon(k)} \expect_{R,\Phi} \sabs{X(\lambda)},
\]
since $b_2(\lambda) \in T_\epsilon(k)$.  The lemma follows because $K_4 2^{-k} D_\epsilon L$ goes to zero as $k\rightarrow\infty$.
\end{IEEEproof}


% \begin{lemma}\label{lem:epslmlemma}
% Suppose $\sabs{\lambda} < \epsilon$ and $\sabs{\lambda^*} < \epsilon$ with $\epsilon < \tfrac{\pi}{M}$. Then
% \begin{align*}
% d(\lambda, \lambda^*) &= \sum_{i \in P \cup D} \big( f_{i}(\lambda) - f_{i}(\lambda^*) \big)^2 \\
% &\leq R_i^2 L \left( (\lambda - \lambda^*)^2 + 4 C_\epsilon \right).
% \end{align*}
% \end{lemma}


% \begin{lemma}
% If $i \in P$, then $\big(f_{i}(\lambda) - f_{i}(\lambda^*)\big)^2 \leq R_i^2(\lambda - \lambda^*)^2.$
% \end{lemma}
% \begin{IEEEproof} 
% We have
% \[
% \big(f_{i}(\lambda) - f_{i}(\lambda^*)\big)^2 = R_i^2\sabs{\sin(\lambda + \Phi_i) - \sin(\lambda^* + \Phi_i)}^2
% \]
% and, since $\sin(\cdot)$ is Lipshitz continuous with constant $K=1$,
% \[
% \sabs{\sin(\lambda + \Phi_i) - \sin(\lambda^* + \Phi_i)} \leq K\sabs{\lambda - \lambda^*} = \sabs{\lambda - \lambda^*}.
% \]
% \end{IEEEproof}

% \begin{lemma}
% Suppose $\sabs{\lambda} < \epsilon$ and $\sabs{\lambda^*} < \epsilon$ with $\epsilon < \tfrac{\pi}{M}$.  If $i \in D$, then
% \begin{equation}\label{eq:fismall}
% \big(f_{i}(\lambda) - f_{i}(\lambda^*)\big)^2 \leq R_i^2(\lambda - \lambda^*)^2
% \end{equation}
% when $\Phi_i \notin S_\epsilon$, otherwise,
% \begin{equation}\label{eq:fibig}
% \big(f_{i}(\lambda) - f_{i}(\lambda^*)\big)^2 \leq 4 R_i^2.
% \end{equation}
% \end{lemma}
% \begin{IEEEproof}
% Because $\sin(\cdot) \leq 1$ it follows that
% \begin{align*}
% \big(f_{i}(\lambda) - f_{i}(\lambda^*)\big)^2 &= R_i^2\sabs{\sin\fracpart{\lambda + \Phi_i} - \sin\fracpart{\lambda^* + \Phi_i}}^2 \\
% &\leq 4 R_i^2
% \end{align*}
% for any $\Phi_i \in \reals$, and so~\eqref{eq:fibig} holds.  If $\Phi_i \notin S_\epsilon$ then $\Phi_i \in (\psi_{k-1} + \epsilon, \psi_{k} - \epsilon)$ for some $k\in\ints$ and 
% \[
% \fracpart{\Phi_i + \lambda} = \Phi_i + \lambda - \tfrac{2\pi}{M}k
% \] 
% and
% \[
% \fracpart{\Phi_i + \lambda^*} = \Phi_i + \lambda^* - \tfrac{2\pi}{M}k 
% \]
% and, since $\sin(\cdot)$ is Lipshitz continuous with constant $K=1$,
% \begin{align*}
% \vert \sin&\fracpart{\Phi_i + \lambda} - \sin\fracpart{\Phi_i + \lambda^*}\vert \\
% &=  \sabs{\sin(\Phi_i + \lambda - \tfrac{2\pi}{M}k) - \sin(\Phi_i + \lambda^* - \tfrac{2\pi}{M}k)} \\
% &\leq K \sabs{\lambda - \lambda^*} \\
% &= \sabs{\lambda - \lambda^*}.
% \end{align*}
% Thus~\eqref{eq:fismall} holds. 
% \end{IEEEproof}


\begin{lemma}\label{lem:chaining3}(Chaining)
For all positive integers $k$,
\[
\expect_{R,\Phi} \sup_{\lambda \in T_\epsilon(k) } \abs{ X(\lambda) } \leq \sqrt{C_\epsilon} \sum_{i=1}^k \frac{5\sqrt{i \log 2 + \log K_3}}{2^{i/2}}.
\]
\end{lemma}
\begin{IEEEproof}
Let $b_1(\lambda)$ map each $\lambda \in T_\epsilon(k)$ to an element in $T_\epsilon(k-1)$ such that $d(\lambda, b_1(\lambda)) \leq 2^{1-k}C_\epsilon L$.  The existence of $b_1$ is guaranteed by the definition of $T_\epsilon(k)$.  By the triangle inequality
\[
\sabs{X(\lambda)} \leq \sabs{X(b_2(\lambda))} + \sabs{X(\lambda) - X(b_2(\lambda))}  
\]
and taking supremums on both sides,
\begin{align}
&\sup_{\lambda \in T_\epsilon(k)}\sabs{X(\lambda)} \nonumber \\
&\leq \sup_{\lambda \in T_\epsilon(k)}\sabs{X(b_2(\lambda))} + \sup_{\lambda \in T_\epsilon(k)} \sabs{X(\lambda) - X(b_2(\lambda))}   \nonumber \\
&\leq \sup_{\lambda \in T_\epsilon(k-1)}\sabs{X(\lambda)} + \sup_{\lambda \in T_\epsilon(k)} \sabs{X(\lambda) - X(b_2(\lambda))}, \label{eq:Xb2recpreexp}
\end{align}
the last line because $b_2(\lambda) \in T_\epsilon(k-1)$ and so
\[
\sup_{\lambda \in T_\epsilon(k)}\sabs{X(b_2(\lambda))} \leq \sup_{\lambda \in T_\epsilon(k-1)}\sabs{X(\lambda)}.
\]
Conditional on $\{R_i\}$ and $\{\Phi_i\}$ the random variable
\[
U(\lambda) = X(\lambda) - X(b_2(\lambda))
\]
has zero mean and is normally distributed with variance
\begin{align*}
\sigma_U^2 &= \expect_{R,\Phi} \frac{1}{L} \sum_{i \in P \cup D} g_i^2 \big( f_i(\lambda) - f_i(b_2(\lambda)) \big)^2 \\
&=  \frac{1}{L} \sum_{i \in P \cup D} \big( f_i(\lambda) - f_i(b_2(\lambda)) \big)^2 \\
&= d(\lambda, b_2(\lambda)) \leq 2^{1-k} C_\epsilon,
\end{align*}
since $\expect_{R,\Phi}g_i^2 = 1$.  So, from Lemma~\ref{lem:maxineq},
\begin{align*}
\expect_{R,\Phi} \sup_{\lambda \in T_\epsilon(k)} \sabs{U(\lambda)} &\leq 3 \sqrt{ 2^{1-k} C_\epsilon \log \abs{T_\epsilon(k)} } \\
&\leq \sqrt{C_\epsilon} \frac{5\sqrt{k \log 2 + \log K_3}}{2^{k/2}}.
\end{align*}
Taking expectations on both sides of~\eqref{eq:Xb2recpreexp},
\begin{align*}
\expect_{R,\Phi}&\sup_{\lambda \in T_\epsilon(k)}\sabs{X(\lambda)} \\
&\leq \expect_{R,\Phi}\sup_{\lambda \in T_\epsilon(k-1)}\sabs{X(\lambda)} + \sqrt{C_\epsilon} \frac{5\sqrt{k \log 2 + \log K_3}}{2^{k/2}} ,
\end{align*}
which involves a recursion in $k$.  By unravelling the recursion and using that $T_\epsilon(k)$ contains only zero, and therefore
\[
\expect_{R,\Phi} \sup_{\lambda \in T_k(0)} \sabs{X(\lambda)} = \expect_{R,\Phi} \sabs{X(0)} = 0
\]
we obtain,
\[
\expect_{R,\Phi}\sup_{\lambda \in T_\epsilon(k)}\sabs{X(\lambda)} \leq  \sqrt{C_\epsilon} \sum_{i=1}^k \frac{5\sqrt{i \log 2 + \log K_3}}{2^{i/2}}  
\]
as required.
\end{IEEEproof}

\begin{lemma}\label{lem:maxineq}(Maximal inequalities)
Suppose $X_1, \dots, X_N$ are zero mean Gaussian random variables each with variance less than some positive constant $K$, then
\[
\expect \sup_{n = 1, \dots, N} \abs{X_n} \leq 3  \sqrt{K \log N}
\]
where $\log N$ is the natural logarithm of $N$.
\end{lemma}
\begin{IEEEproof}
This result is well known, see for example~\cite[Section 3]{Pollard_asymp_empi_proc_1989}  
\end{IEEEproof}


\begin{lemma}\label{lem:metricentropy}(Covering numbers)
For $k \in \ints$ there exists a discrete set $T_\epsilon(k) \subset B_\epsilon$ with the property that, for every $\lambda \in B_\epsilon$, there is a $\lambda_1$ and $\lambda_2$ in $T_\epsilon(k)$ such that the pseudometric
\[
d_1(\lambda, \lambda_1) = \sum_{i \in P \cup D} \big( f_{i}(\lambda) - f_{i}(\lambda_1) \big)^2 \leq 2^{-k} C_\epsilon L,
\]
and the pseudometric
\[
d_2(\lambda, \lambda_2) = \sum_{i \in P \cup D} \sabs{f_{i}(\lambda) - f_{i}(\lambda_2)} \leq 2^{-k} D_\epsilon L,
\]
where $C_\epsilon$ and $D_\epsilon$ are defined in~\eqref{eq:Cepsdefn} and~\eqref{eq:Depsdefn}.  The number of elements in $T_\epsilon(k)$ is no more than $K_3 2^{k}$ where $K_3$ is a positive constant, independent of $L$, $\epsilon$ and $k$.
\end{lemma}
\begin{IEEEproof}
% We will first construct a set $T^2_\epsilon(k)$ containing no more than $K_{23}2^{k}$ elements and so that for every $\lambda \in B_\epsilon$ there exists $\lambda_2 \in T^2_\epsilon(k)$ such that $d_2(\lambda, \lambda_2) \leq 2^{-k} D_\epsilon L$ with $K_{23}$ a constant, independent of $L$, $\epsilon$ and $k$.  Using a similar construction we find a set $T^1_\epsilon(k)$ containing no more than $K_{13}2^{k}$ elements and so that for every $\lambda \in B_\epsilon$ there exists $\lambda_1 \in T^1_\epsilon(k)$ such that $d_1(\lambda, \lambda_2) \leq 2^{-k} D_\epsilon L$ with $K_{13}$ a constant, independent of $L$, $\epsilon$ and $k$.  Puting $T_\epsilon(k) = T^1_\epsilon(k) \cup T^2_\epsilon(k)$ will then complete the proof with $K_3 = K_{13} + K_{23}$.

% We will only give a construction for $T^2_\epsilon(k)$, the construction for $T^1_\epsilon(k)$ being similar.  First observe that
% \[
% f_{i}(\lambda) - f_{i}(\lambda_2) = \begin{cases}
% R_i\big( \sin(\lambda + \Phi_i) - \sin(\lambda_2 + \Phi_i) \big) & i \in P \\
% R_i\big(\sin\fracpart{\lambda + \Phi_i} - \sin\fracpart{\lambda_2 + \Phi_i}\big) & i \in D
% \end{cases}
% \]
% Let 
% \[
% U = \{\epsilon(\ell 2^{-k} - 1) \mid \ell = 0, 1, \dots, 2^{k+1} \} 
% \]
% be a set containing $2^{k+1} + 1$ elements uniformly spaced over $B_\epsilon$.  For every $\lambda \in B_\epsilon$ there exists a $\lambda^* \in U$ such that $\abs{\lambda - \lambda^*} < \epsilon 2^{-k}$.  Since $\sin(\cdot)$ is Lipshitz continuous with constant $K=1$, it follows that if $i \in P$ then,
% Let 
% \begin{align*}
% \abs{f_{i}(\lambda) - f_{i}(\lambda_2)} &= R_i\abs{ \sin(\lambda + \Phi_i) - \sin(\lambda^* + \Phi_i) } \\
% &\leq R_i\sabs{\lambda - \lambda^*} \\
% &< \epsilon 2^{-k} R_i.
% \end{align*}
% Thus
% \[
% \sum_{i \in P} \sabs{f_{i}(\lambda) - f_{i}(\lambda^*)} < \epsilon 2^{-k} \sum_{i\in P} R_i.
% \]
% As in Lemma~\ref{lem:epslmlemma} let $S = \{ i \in D \mid I_\epsilon(\Phi_i) \}$. Now, if $i \in D$ but $i \notin S$ then 
% \[
% \round{\lambda + \Phi_i} = \round{\lambda^* + \Phi_i} = \round{\Phi_i},
% \] 
% so
%  \begin{align*}
%  \vert f_{i}(\lambda) - &f_{i}(\lambda^*)\vert \\
% &= R_i\vert\sin\fracpart{\lambda + \Phi_i} - \sin\fracpart{\lambda^* + \Phi_i}\vert \\
%  &= R_i\vert\sin(\lambda + \Phi_i - \round{\Phi_i}) - \sin(\lambda^* + \Phi_i - \round{\Phi_i})\vert  \\
%  &\leq R_i \sabs{\lambda - \lambda^*} \\
% &< \epsilon 2^{-k} R_i.
%  \end{align*}
% Thus,
% \[
% \sum_{i \in D, i \notin S} \sabs{f_{i}(\lambda) - f_{i}(\lambda^*)} < \epsilon 2^{-k} \sum_{i \in D, i \notin S} R_i 
% \]

% To handle the case when $i \in S$ we need to add some extra elements to the set $U$.  Let 
% \[
% r_i = \begin{cases}
% \tfrac{\pi}{M} - \fracpart{\Phi_i} & \fracpart{\Phi_i} > 0 \\
% -\tfrac{\pi}{M} - \fracpart{\Phi_i} & \fracpart{\Phi_i} \leq 0
% \end{cases}
% \]
% and let $\sigma$ be a permutation of the indices in $S$ such that $r_i$ are in ascending order, that is,
% \begin{equation}\label{eq:sigmasortind}
% r_{\sigma(i)} \leq r_{\sigma(k)}
% \end{equation}
% whenever $i < k $ where $i, k \in \{0, 1, \dots, \abs{S}-1\}$.  Now,
% \begin{align*}
% \sum_{i\in S} \sabs{f_{i}(\lambda) - f_{i}(\lambda^*)} = \sum_{i = 0}^{\abs{S}-1} \sabs{ f_{\sigma(i)}(\lambda) - f_{\sigma(i)}(\lambda^*)}.
% \end{align*}
% If $\epsilon > \lambda \geq r_{\sigma(k)} > 0$ for some $k$, then
% \[
% \round{\lambda + \fracpart{\Phi_{\sigma(i)}}} = \round{\lambda + \tfrac{\pi}{M} - r_{\sigma(i)}} = \tfrac{\pi}{M}
% \]
% for all $i \leq k$, whilst
% \[
% \round{\lambda + \fracpart{\Phi_{\sigma(i)}}} = \round{\lambda + \tfrac{\pi}{M} - r_{\sigma(i)}} = 0
% \]
% for all $i > k$.  Otherwise if $-\epsilon < \lambda < r_{\sigma(k)} \leq 0$ for some $k$, then
% \[
% \round{\lambda + \fracpart{\Phi_{\sigma(i)}}} = \round{\lambda + -\tfrac{\pi}{M} - r_{\sigma(i)}} = -\tfrac{\pi}{M}
% \]
% for all $i \leq k$, whilst
% \[
% \round{\lambda + \fracpart{\Phi_{\sigma(i)}}} = \round{\lambda + \tfrac{\pi}{M} - r_{\sigma(i)}} = 0
% \]
% for all $i > k$.

% Let $R = \sum_{i \in S} R_i$ and let $d(\ell)$ be the largest integer such that
% \[
% \sum_{i=0}^{d(\ell)} R_{\sigma(i)} \leq \frac{\ell}{2^k} R.
% \]
% for integers $\ell = 1, \dots, 2^k$.  Define the intervals
% \[
% L_\ell = \in [r_{d(\ell)}, r_{d(\ell+1)})
% \]
% where $r_{0} = -\epsilon$ and $r_{2^k+1} = \epsilon$.  Then $L_0, L_1, \dots, L_{2^k}$ partition the closure of $B_\epsilon$.  If $\lambda$ and $\lambda^*$ are from interval $L_\ell$, then for all $i \in \{0, 1, \dots, \abs{S}-1\}$ less than $d(\ell)$,
% \begin{align*}
% \fracpart{\lambda + \Phi_{\sigma(i)}} &= \fracpart{\lambda + \fracpart{\Phi_{\sigma(i)}}} \\
% &= \fracpart{\lambda + \fracpart{\Phi_{\sigma(i)}}}
% \end{align*}
% \begin{align*}
% \vert f_{\sigma(i)}(\lambda) - &f_{\sigma(i)}(\lambda^*) \vert \\
% &= R_{\sigma(i)}\vert\sin\fracpart{\lambda + \Phi_{\sigma(i)}} - \sin\fracpart{\lambda^* + \Phi_{\sigma(i)}}\vert 
% \end{align*}
BLERG

\end{IEEEproof}


\small
\bibliography{bib}



\end{document}
