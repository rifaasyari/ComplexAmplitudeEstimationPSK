\documentclass[a4paper,10pt]{article}
%\documentclass[draftcls, onecolumn, 11pt]{IEEEtran}
%\documentclass[journal]{IEEEtran}
 
\usepackage{mathbf-abbrevs}
%\newcommand {\tbf}[1] {\textbf{#1}}
%\newcommand {\tit}[1] {\textit{#1}}
%\newcommand {\tmd}[1] {\textmd{#1}}
%\newcommand {\trm}[1] {\textrm{#1}}
%\newcommand {\tsc}[1] {\textsc{#1}}
%\newcommand {\tsf}[1] {\textsf{#1}}
%\newcommand {\tsl}[1] {\textsl{#1}}
%\newcommand {\ttt}[1] {\texttt{#1}}
%\newcommand {\tup}[1] {\textup{#1}}
%
%\newcommand {\mbf}[1] {\mathbf{#1}}
%\newcommand {\mmd}[1] {\mathmd{#1}}
%\newcommand {\mrm}[1] {\mathrm{#1}}
%\newcommand {\msc}[1] {\mathsc{#1}}
%\newcommand {\msf}[1] {\mathsf{#1}}
%\newcommand {\msl}[1] {\mathsl{#1}}
%\newcommand {\mtt}[1] {\mathtt{#1}}
%\newcommand {\mup}[1] {\mathup{#1}}

%some math functions and symbols
\newcommand{\reals}{{\mathbb R}}
\newcommand{\expect}{{\mathbb E}}
\newcommand{\indicator}{{\mathbf 1}}
\newcommand{\ints}{{\mathbb Z}}
\newcommand{\complex}{{\mathbb C}}
\newcommand{\integers}{{\mathbb Z}}
\newcommand{\sign}[1]{\mathtt{sign}\left( #1 \right)}
\newcommand{\NP}{\operatorname{NPt}}
\newcommand{\erf}{\operatorname{erf}}
\newcommand{\NS}{\operatorname{NearestSet}}
\newcommand{\bres}{\operatorname{Bres}}
\newcommand{\vol}{\operatorname{vol}}
\newcommand{\vor}{\operatorname{Vor}}
%\newcommand{\re}{\operatorname{Re}}
%\newcommand{\im}{\operatorname{Im}}




\newcommand{\term}{\emph}
\newcommand{\var}{\operatorname{var}}
%\newcommand{\prob}{{\mathbb P}}
\newcommand{\prob}{{\operatorname{Pr}}}

%distribution fucntions
\newcommand{\projnorm}{\operatorname{ProjectedNormal}}
\newcommand{\vonmises}{\operatorname{VonMises}}
\newcommand{\wrapnorm}{\operatorname{WrappedNormal}}
\newcommand{\wrapunif}{\operatorname{WrappedUniform}}

\newcommand{\selectindicies}{\operatorname{selectindices}}
\newcommand{\sortindicies}{\operatorname{sortindices}}
\newcommand{\largest}{\operatorname{largest}}
\newcommand{\quickpartition}{\operatorname{quickpartition}}
\newcommand{\quickpartitiontwo}{\operatorname{quickpartition2}}

%some commonly used underlined and
%hated symbols
\newcommand{\uY}{\ushort{\mbf{Y}}}
\newcommand{\ueY}{\ushort{Y}}
\newcommand{\uy}{\ushort{\mbf{y}}}
\newcommand{\uey}{\ushort{y}}
\newcommand{\ux}{\ushort{\mbf{x}}}
\newcommand{\uex}{\ushort{x}}
\newcommand{\uhx}{\ushort{\mbf{\hat{x}}}}
\newcommand{\uehx}{\ushort{\hat{x}}}

% Brackets
\newcommand{\br}[1]{{\left( #1 \right)}}
\newcommand{\sqbr}[1]{{\left[ #1 \right]}}
\newcommand{\cubr}[1]{{\left\{ #1 \right\}}}
\newcommand{\abr}[1]{\left< #1 \right>}
\newcommand{\abs}[1]{{\left\vert #1 \right\vert}}
\newcommand{\sabs}[1]{{\vert #1 \vert}}
\newcommand{\floor}[1]{{\left\lfloor #1 \right\rfloor}}
\newcommand{\ceiling}[1]{{\left\lceil #1 \right\rceil}}
\newcommand{\ceil}[1]{\left\lceil #1 \right\rceil}
\newcommand{\round}[1]{{\left\lfloor #1 \right\rceil}}
\newcommand{\magn}[1]{\left\| #1 \right\|}
\newcommand{\fracpart}[1]{\left\langle #1 \right\rangle}
\newcommand{\sfracpart}[1]{\langle #1 \rangle}


% Referencing
\newcommand{\refeqn}[1]{\eqref{#1}}
\newcommand{\reffig}[1]{Figure~\ref{#1}}
\newcommand{\reftable}[1]{Table~\ref{#1}}
\newcommand{\refsec}[1]{Section~\ref{#1}}
\newcommand{\refappendix}[1]{Appendix~\ref{#1}}
\newcommand{\refchapter}[1]{Chapter~\ref{#1}}

\newcommand {\figwidth} {100mm}
\newcommand {\Ref}[1] {Reference~\cite{#1}}
\newcommand {\Sec}[1] {Section~\ref{#1}}
\newcommand {\App}[1] {Appendix~\ref{#1}}
\newcommand {\Chap}[1] {Chapter~\ref{#1}}
\newcommand {\Lem}[1] {Lemma~\ref{#1}}
\newcommand {\Thm}[1] {Theorem~\ref{#1}}
\newcommand {\Cor}[1] {Corollary~\ref{#1}}
\newcommand {\Alg}[1] {Algorithm~\ref{#1}}
\newcommand {\etal} {\emph{~et~al.}}
\newcommand {\bul} {$\bullet$ }   % bullet
\newcommand {\fig}[1] {Figure~\ref{#1}}   % references Figure x
\newcommand {\imp} {$\Rightarrow$}   % implication symbol (default)
\newcommand {\impt} {$\Rightarrow$}   % implication symbol (text mode)
\newcommand {\impm} {\Rightarrow}   % implication symbol (math mode)
\newcommand {\vect}[1] {\mathbf{#1}} 
\newcommand {\hvect}[1] {\hat{\mathbf{#1}}}
\newcommand {\del} {\partial}
\newcommand {\eqn}[1] {Equation~(\ref{#1})} 
\newcommand {\tab}[1] {Table~\ref{#1}} % references Table x
\newcommand {\half} {\frac{1}{2}} 
\newcommand {\ten}[1] {\times10^{#1}}
\newcommand {\bra}[2] {\mbox{}_{#2}\langle #1 |}
\newcommand {\ket}[2] {| #1 \rangle_{#2}}
\newcommand {\Bra}[2] {\mbox{}_{#2}\left.\left\langle #1 \right.\right|}
\newcommand {\Ket}[2] {\left.\left| #1 \right.\right\rangle_{#2}}
\newcommand {\im} {\mathrm{Im}}
\newcommand {\re} {\mathrm{Re}}
\newcommand {\braket}[4] {\mbox{}_{#3}\langle #1 | #2 \rangle_{#4}} 
\newcommand{\dotprod}[2]{ \left\langle #1 , #2 \right\rangle}
\newcommand {\trace}[1] {\text{tr}\left(#1\right)}

% spell things correctly
\newenvironment{centre}{\begin{center}}{\end{center}}
\newenvironment{itemise}{\begin{itemize}}{\end{itemize}}

%%%%% set up the bibliography style
\bibliographystyle{IEEEbib}
%\bibliographystyle{uqthesis}  % uqthesis bibliography style file, made
			      % with makebst

%%%%% optional packages
\usepackage[square,comma,numbers,sort&compress]{natbib}
		% this is the natural sciences bibliography citation
		% style package.  The options here give citations in
		% the text as numbers in square brackets, separated by
		% commas, citations sorted and consecutive citations
		% compressed 
		% output example: [1,4,12-15]

%\usepackage{cite}		
			
\usepackage{units}
	%nice looking units
		
\usepackage{booktabs}
		%creates nice looking tables
		
\usepackage{ifpdf}
\ifpdf
  \usepackage[pdftex]{graphicx}
  %\usepackage{thumbpdf}
  %\usepackage[naturalnames]{hyperref}
\else
	\usepackage{graphicx}% standard graphics package for inclusion of
		      % images and eps files into LaTeX document
\fi

\usepackage{amsmath,amsfonts,amssymb, amsthm, bm} % this is handy for mathematicians and physicists
			      % see http://www.ams.org/tex/amslatex.html

		 
\usepackage[vlined, linesnumbered]{algorithm2e}
	%algorithm writing package
	
\usepackage{mathrsfs}
%fancy math script

%\usepackage{ushort}
%enable good underlining in math mode

%------------------------------------------------------------
% Theorem like environments
%
\newtheorem{theorem}{Theorem}
%\theoremstyle{plain}
\newtheorem{acknowledgement}{Acknowledgement}
%\newtheorem{algorithm}{Algorithm}
\newtheorem{axiom}{Axiom}
\newtheorem{case}{Case}
\newtheorem{claim}{Claim}
\newtheorem{conclusion}{Conclusion}
\newtheorem{condition}{Condition}
\newtheorem{conjecture}{Conjecture}
\newtheorem{corollary}{Corollary}
\newtheorem{criterion}{Criterion}
\newtheorem{definition}{Definition}
\newtheorem{example}{Example}
\newtheorem{exercise}{Exercise}
\newtheorem{lemma}{Lemma}
\newtheorem{notation}{Notation}
\newtheorem{problem}{Problem}
\newtheorem{proposition}{Proposition}
\newtheorem{property}{Property}
\newtheorem{remark}{Remark}
\newtheorem{solution}{Solution}
\newtheorem{summary}{Summary}
%\numberwithin{equation}{section}
%--------------------------------------------------------


%\usepackage{xr}
%\externaldocument{paper2}

\usepackage[left=2cm,right=2cm,top=2cm,bottom=2cm]{geometry}

\usepackage{amsmath,amsfonts,amssymb, amsthm, bm}

%\usepackage[square,comma,numbers,sort&compress]{natbib}

\usepackage{color}
\newcommand{\comment}[1]{\textcolor{red}{#1}}

\newcommand{\sgn}{\operatorname{sgn}}
\newcommand{\sinc}{\operatorname{sinc}}
\newcommand{\rect}[1]{\operatorname{rect}\left(#1\right)}

%opening
\title{Carrier phase and amplitude estimation for phase shift keying using pilots and data}
\author{Robby McKilliam, Andr\'{e} Pollok, Bill Cowley, I.\ Vaughan L.\ Clarkson and Barry Quinn  
%\thanks{}
}

\renewcommand{\theenumii}{(\alph{enumii})}
\renewcommand{\labelenumii}{\theenumii}

\begin{document}

\maketitle


\section*{Reviewer 1}

\begin{enumerate}
\item\textbf{Comment:} 
	This paper is concerned with carrier phase and amplitude estimation for PSK modulated signalling.
The considered least squares estimator seems to be only a minor extension of the one from [7]. Yet, more than 4 pages are dedicated to its derivation. Perhaps this section could be made more concise.
	\\\textbf{Response:}
%	We have shortened Sec.\ II to make it more concise and improve readability.
\item\label{R1it2}\textbf{Comment:} 
	As stated by the authors themselves, the key results and main contributions of the paper involve the asymptotic properties of the least squares estimator. As far as I can see the derivation seems correct and complete, yet it is mathematically very involved and rather difficult to follow.
	\\\textbf{Response:}	
%	For improved readability of the manuscript, we have now moved all of the lemmas into the appendix. We have also added a discussion of the implications and the practical relevance of the theorems to Sec.\ III (Sec.\ IV in the original manuscript).

%	\comment{Stress practical relevance by showing ASRP PER results. Mention trials and patents. Point out that knowledge of the estimator statistics makes it possible to assess the estimator performance for a given number of symbols/pilot symbols at a certain SNR without having to run time consuming simulations. Also point out that the results are valid for a wider range of noise distributions, whilst CRBs need to be derived for a specific noise distribution\dots}
\item
A few minor remarks are:
\begin{enumerate}
\item\label{R1it3a}\textbf{Comment:} 
	Estimation from data symbols only is referred to as noncoherent, I rather consider a detector with non-data-aided synchronization as coherent (since it uses estimates of the synchronization parameters to detect the data).
	\\\textbf{Response:}
%	We agree with the reviewer and apologise for causing confusion with the use of the terms \emph{coherent} and \emph{noncoherent} in the original manuscript. In the updated manuscript, we refer to an estimator that uses known pilot symbols as \emph{data-aided} and to one that relies on receiver decisions on the data symbols as \emph{decision-directed}.
\item\textbf{Comment:} 
	missing `,' after (5)
	\\\textbf{Response:} 
%We thank the reviewer for spotting the missing comma and have added it.
\item\textbf{Comment:} 
	p.\ 19, at the bottom: sortindicIes (typo, delete 'i')
	\\\textbf{Response:} % Thank you for pointing out this typographical error. We have corrected it.
\item\textbf{Comment:} 
	In Section VI, I would prefer a reference to the Lemma's in the Appendix at the start of the section.
	\\\textbf{Response:} %We have added a reference to the Lemmas in the appendix at the beginning of the section. Please note that Sec.\ VI of the original manuscript is now Sec.\ III.
\item\textbf{Comment:} 
	Section VIII, ``Theorem 2 does not model this behaviour\dots'' can you please further explain why? + Comparison with the Cramer-Rao bound results from the existing literature would be interesting.
	\\\textbf{Response:}
%	We have removed the aforementioned sentence from the manuscript as we feel it has caused more confusion than provided insight. We hope the following explanation clears up any confusion it may have caused.  Theorem 2 shows that for $L\rightarrow\infty$, $\sqrt{L}\sfracpart{\theta_0-\hat{\theta}}_\pi$ is normally distributed with zero mean and finite variance.  The support of $\sqrt{L}\sfracpart{\theta_0-\hat{\theta}}_\pi$ is the interval $[-\sqrt{L}\pi, \sqrt{L}\pi)$, i.e.\ $(-\infty, \infty)$ as $L\rightarrow\infty$.  For finite $L$, however, $\sqrt{L}\sfracpart{\theta_0-\hat{\theta}}_\pi$ has finite support and thus, cannot be exactly normally distributed.  Nevertheless, provided the variance of the noise is sufficiently small (i.e.\ the SNR sufficiently high), the distribution of $\sqrt{L}\sfracpart{\theta_0-\hat{\theta}}_\pi$ will be concentrated in a finite interval much smaller than $[-\sqrt{L}\pi, \sqrt{L}\pi)$ and is very close to a normal distribution.  On the other hand, as the SNR decreases, $\sqrt{L}\sfracpart{\theta_0-\hat{\theta}}_\pi$ approaches a uniform distribution on $[-\sqrt{L}\pi, \sqrt{L}\pi)$.  Consequently, the variance of the phase estimator approaches that of the uniform distribution on $[-\pi, \pi)$.  The argument for the case without pilots is analogous.
		
%	\comment{Add CRBs to plots.} Please note that Sec.\ VIII of the original manuscript is now Sec.\ VI.
\end{enumerate}
\end{enumerate}

\section*{Reviewer 2}

\begin{enumerate}
\item
	General remarks
\begin{enumerate}
\item\textbf{Comment:} 
	Estimation of carrier phase and amplitude for $M$-ary PSK signals has been studied by the authors. Extending Mackenthun�s algorithm to pilots, a least square (LS) method is developed, where major parts of the manuscript are dedicated to the statistical analysis of the LS approach.

	Generally speaking it is to admit that both wording and writing follow the IEEE standards, although I would say that the list of References should be checked and updated in this respect.
	\\\textbf{Response:}
%	We thank the reviewer for pointing us to the suggested references and have added \comment{some/most of} them. Specifically, we have included references [c, d, e]. \comment{Add the remaining ones?}
\end{enumerate}
\item
Some editorial comments
\begin{enumerate}
\item\textbf{Comment:} 
	Paragraph before (13): \dots can be computed from its predecessor \dots
	\\\textbf{Response:} %Thank you for pointing out this typographical error. We have corrected it.
\item\textbf{Comment:} 
	Section III and VII: it would be helpful to provide $f(r, \phi)$
	\\\textbf{Response:}
%	We thank the reviewer for pointing out the missing definition of $f(r, \phi)$ and have added it. Please note that in response to comment \ref{R4it2e} of reviewer 4, Sec.\ III has been slightly shortened and has been moved to the appendix.
\item\textbf{Comment:} 
	In the introductory section (paragraph before Eq.\ 3), the authors are talking about ``noncoherent detection where no pilot symbols exist''. In my mind this is somewhat confusing, because parameter estimation without knowing the underlying data is usually denoted as non-data aided (NDA) or decision-directed (DD), whereas when these data are known we have a data-aided (DA) situation [a]. On the other hand, coherent/noncoherent scenarios are given if the carrier phase is known/unknown to the receiver station.
	\\\textbf{Response:}
%	Please refer to our response to comment \ref{R1it3a} of reviewer 1.
\end{enumerate}
\item
Major issues
\begin{enumerate}
\item\textbf{Comment:} 
	Figs.1 to 3 illustrate the evolution of the phase MSE vs.\ SNR. Only in case that $|P| = 0$,
the LS results are compared to an alternative solution represented by the Viterbi and Viterbi (VV) algorithm. Why not doing the same with $|P| > 0$ by the following approach, which is definitely much simpler from the computational point of view (complexity in the order $O(L)$ instead of $O(L \log(L))$ required by LS) and the jitter performance will probably differ not very much from that of the LS variant (in Figs.\ 1�3 this can be seen for $|P| = 0$):
	\begin{itemize}
	\item
		If SNR $<$ SNRth, then the DA algorithm has to be used, else the VV algorithm must be applied.
	\item
		The SNRth is a threshold value known in advance, e.g., by the intersection of the Cramer-Rao lower bounds (CRLBs) for DA and NDA phase estimation [a, b].
	\item
		The SNR estimates might be obtained by a simple method based on second- and fourth-order moments [c].
	\item
		The same philosophy applies also to the estimation of amplitudes; for carrier- blind NDA results, the moment-based approach used for SNR estimation can be exploited.
	\end{itemize}
	\textbf{Response:}
%	\comment{add response\dots}
\item\textbf{Comment:} 
	By detailed inspection of Fig.\ 4 showing the jitter variance for amplitude estimates, it is easily observed that $|P| < L$ performs better than $|P| = L$ in the medium-to-low SNR range. In my mind this is somewhat strange, since it should be the other way round (similar to the evolution of the phase curves in Figs.\ 1�3 and 5). Can you comment on this phenomenon ?
	\\\textbf{Response:}
%	\comment{Replace variance plot by MSE plot.}
\item\textbf{Comment:} 
	In a much more powerful approach to the topic addressed in the current paper, the pilot sequence would be employed for a coarse estimation/correction of the carrier phase of the payload data. In the sequel, by embedding carrier phase recovery into an iterative (turbo) decoding scheme [d, e] or any other error correcting method [f, g], the jitter performance is close to the CRLB for DA estimation even at very small SNRs, which are mainly determined by the convergence region of the (turbo) decoder. By taking into account the fact that modern and powerful communication systems are not conceivable without any FEC scheme, it is not clear to me why I should choose your algorithm in such a context.
	\\\textbf{Response:}
%	We have added a brief discussion of the need for sufficiently accurate coarse estimates to initialise an iterative code-aided synchronisation stage, citing [d, e].  We have also added a set of new simulation results, which show the packet error rate of an LDPC-coded satellite communication system to demonstrate the use and performance of the proposed estimator in a practical system.  \comment{Add details\dots}
	
%	\comment{Add response related to [f, g]\dots}
\end{enumerate}
\end{enumerate}

\subsection*{Reviewer 2 references}
\begin{enumerate}\footnotesize
\renewcommand{\theenumi}{\alph{enumi}}
\renewcommand{\labelenumi}{[\theenumi]}
\addtolength{\itemsep}{-1.5ex}
\item
	Mengali, D'Andrea, \textit{Synchronization Techniques for Digital Receivers}, Plenum Press: New York, 1997.
\item
	Rice, Cowley, Moran, Rice, ``Cramer-Rao Lower Bounds for QAM Phase and Frequency Estimation'', \textit{IEEE Trans.\ Commun.}, Sept 2001.
\item
	Pauluzzi, Beaulieu, ``A Comparison of SNR Estimation Techniques for the AWGN Channel'', \textit{IEEE Trans.\ Commun.}, Oct.\ 2000.
\item
	Lottici, Luise, ``Embedding Carrier Phase Recovery into Iterative Decoding of Turbo-Coded Linear Modulations'', \textit{IEEE Trans.\ Commun.}, April 2004.
\item
	Noels et al., ``A Theoretical Framework for Soft-Information-Based Synchronization in Iterative (Turbo) Receivers'', \textit{Eurasip Journal Wireless Commun.\ and Networking}, April 2005.
\item
	D'Andrea, Mengali, Vitetta, ``Approximate ML Decoding of Coded PSK with no Explicit Carrier Phase Reference'', \textit{IEEE Trans.\ Commun.}, Feb./March/April 1994.
\item
	Vanelli, Salmi, Cioni, Corazza, Polydoros, ``A Performance Review of PSP for Joint Phase/Frequency and Data Estimation in Future Broadband Satellite Networks'', \textit{IEEE Journal on Selected Areas in Commun.}, Dec.\ 2001.
\end{enumerate}

\section*{Reviewer 3}

\begin{enumerate}
\item
	\textbf{Comment:} 
The paper studies the estimation of amplitude and phase due to the distortion introduced by the channel in communication systems with $M$-PSK signals. The estimator is derived from the observation of $L$ noisy samples taken at a rate of one sample per symbol. Part of the $L$ observed samples corresponds to pilot transmitted symbols, that are already known a priori by the receiver, while the rest of samples are associated with data symbols which are unknown at the receiver. The focus of the article is to determine the amplitude and phase without considering the estimation of the transmitted data like this problem is usually treated in the literature. The work modifies the Mackenthun algorithm to include samples corresponding to pilot symbols while the Mackenthun paper calculates a least squares estimator for a similar problem but without pilot symbols.

The major contribution of this paper is the description of the asymptotic properties of the least squares estimator for the phase and amplitude. Although it seems technically correct, the authors do not make clear the relevance of their work to signal processing applications and to timely problems. They limit themselves in presenting several lemmas and proofs. Therefore, I cannot recommend this manuscript for publication in the IEEE Transactions on Signal Processing.
	\\\textbf{Response:}
%	We thank the reviewer for the comments. Please refer to our response to comment \ref{R1it2} of reviewer 1.
\end{enumerate}

\section*{Reviewer 4}

\begin{enumerate}
\item\textbf{Comment:}
	This work addresses the problem of carrier phase and amplitude estimation for $M$-ary PSK signals. Based on a modification of the Mackhentum's algorithm by inserting pilots, the main contribution of this manuscript is in fact the study of the asymptotic behavior of the least squares estimator for carrier phase and amplitude. Although the derivations seems correct, it is not clear to the reader why this method should be applicable in a communications context.  A cconsiderable part of the manuscript is dedicated to proofs of Lemmas and Theorems, which makes the paper rather difficult to read. There are several methods in the litterature that proposes the use of pilots to estimate the carrier phase followed by efficient decoding such as turbo decoding. How does the proposed method compare with the existing uncoherent methods that do not require estimation of the amplitudes? Even if the data are coded, what are the gains in terms of data recovery of the proposed method and traditional solutions? In my opinion, the authors fail to convincing on this aspect and this is an important point.
	\\\textbf{Response:}
%	\comment{Stress practical relevance by showing ASRP PER results. In particular ST1, where the estimators are used to seed turbo sync. Mention trials and patents\dots}
	
%	\comment{Not clear what the reviewer means by ``existing uncoherent methods that do not require estimation of the amplitudes''.}

\item
Some additional comments:
\begin{enumerate}
\item\textbf{Comment:} 
	The authors should better motivate the interest in using their proposed method in actual communication systems. The lack of motivation for addressing the problem raised in the manuscript is a weak aspect of this work.
	\\\textbf{Response:}
%	\comment{Show PER ASRP results. Mention trials and patents\dots}
\item\textbf{Comment:} 
	The simulation results should bring comparisons with existing methods that use error correction and turbo decoding techniques, which have shown to perform reasonably well under adverse situations and low SNR levels.
	\\\textbf{Response:}
%	\comment{Show PER ASRP ST1 turbo sync results. Comparison with other estimators?}
\item\textbf{Comment:} 
	In Fig.\ 4, the influence of $P$ on the performance is not clear to me and does not seem to be in accordance with other results in the manuscript.
	\\\textbf{Response:} 
%	Please note that Fig.\ 4 in the original submission showed the variance of the amplitude estimator $\hat\rho$ (after bias removal), rather than the estimator MSE. This was done to validate the theoretical value of the asymptotic variance of $\hat\rho$ predicted by Theorem~2 against Monte-Carlo simulations. For consistency with the other figures and for easier interpretation, we have replaced the variance plot by a plot of the MSE of $\hat\rho$.
\item\textbf{Comment:} 
	It is important that the authors address an appropriate method for correcting the bias, as they mentioned in the text. A solution to provide the receiver with an estimate of $G(0)$ must be addressed in order to make this contribution applicable in a real communications context.
	\\\textbf{Response:} 
%	Biased estimators are quite common in the communications literature and in fact, are often used in practice. Prime examples are state-of-the-art signal-to-noise ratio (SNR) estimators for $M$-PSK modulated signals such as those discussed by Pauluzzi and Beaulieu \cite{Pauluzzi2000}. With the exception of the data-aided SNR estimators, all of the estimators in \cite{Pauluzzi2000} are biased and particularly at low SNR, the bias can be large. To the best of the authors' knowledge, the value of the bias of these estimators has only been determined by computer simulation and has not been analysed theoretically. In contrast, the asymptotic analysis of the amplitude estimator $\hat\rho$ presented in Theorem~1 of our manuscript results in a closed-form expression for the bias and thus, lays the foundation for improving the estimator by addressing the bias. However, removal of the bias is beyond the scope of this manuscript.
\item\textbf{Comment:} \label{R4it2e}
	Some sectioms can be shortened to make the work more readable and concise. This is the case of Section II for instance. Section III is not necessary. It is too short and the main concepts are well known. A reference to a textbook would be sufficient. Section VII is also too short and is not justified.
	\\\textbf{Response:}
%	We have shortened Sec.\ II to make it more concise and improve readability. Sec.\ VII has been removed from the revised manuscript.
	
%	Furthermore, we agree with the reviewer that circularly symmetric random variables are well known, but would like to point out that Sec.\ III specifically discusses random variables of the form $Re^{j\varPhi}=1{+}W$, where $W$ is circularly symmetric. The joint pdf of $R$ and $\varPhi$ is a crucial ingredient for the asymptotic analysis presented in the manuscript and therefore, we feel the section should be retained. However, for improved readability, Sec.\ III has been slightly shortened and has been moved to the appendix.
\end{enumerate}
\end{enumerate}

\bibstyle{IEEEtran}
\bibliography{bib}

\end{document}





















