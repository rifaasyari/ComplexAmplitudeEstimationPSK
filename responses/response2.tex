\documentclass{article}

\usepackage[pdftex]{graphicx}
%\usepackage{thumbpdf}
%\usepackage[naturalnames]{hyperref}
\usepackage{xr}
\externaldocument{../async}

\usepackage{bm, booktabs}
\usepackage{amsmath}%
\usepackage{amsfonts}%
\usepackage{amssymb}%
%\usepackage{amsthm}
\usepackage{mathrsfs}
%\usepackage{graphicx}
%\usepackage{setspace}
\usepackage{cite}
\usepackage{units}
	%nice looking units
%\usepackage{times}
\usepackage[normalem]{ulem}
%\usepackage{algorithmic}
%\usepackage[figure, vlined, linesnumbered]{algorithm2e}
\usepackage[vlined, linesnumbered]{algorithm2e}
%\usepackage{algorithm}

%%%%% set up the bibliography style
\bibliographystyle{../IEEEbst}
%\usepackage[square,comma,numbers]{natbib}

\newcommand{\fracpart}[1]{\left\langle #1 \right\rangle}
\newcommand{\sfracpart}[1]{\langle #1 \rangle}
\newcommand{\round}[1]{\lfloor #1 \rceil}
\newcommand{\floor}[1]{\lfloor #1 \rfloor}
\newcommand{\ceil}[1]{\lceil #1 \rceil}
\newcommand{\reals}{{\mathbb R}}
\newcommand{\ints}{{\mathbb Z}}
\newcommand{\integers}{{\mathbb Z}}
\newcommand{\uy}{\underline{\bm{Y}}}
\newcommand{\uey}{\underline{Y}}
\newcommand{\sign}[1]{\mathtt{sign}(#1)}
\newcommand{\Qbf}{{\mathbf Q}}
\newcommand{\Bbf}{{\mathbf B}}
\newcommand{\Ibf}{{\mathbf I}}
\newcommand{\ybf}{{\mathbf y}}
\newcommand{\xbf}{{\mathbf x}}
\newcommand{\zbf}{{\mathbf z}}
\newcommand{\ebf}{{\mathbf e}}
\newcommand{\vbf}{{\mathbf v}}
\newcommand{\wbf}{{\mathbf w}}
\newcommand{\kbf}{{\mathbf k}}
\newcommand{\sbf}{{\mathbf s}}
\newcommand{\tbf}{{\mathbf t}}
\newcommand{\pbf}{{\mathbf p}}
\newcommand{\Pibf}{{\mathbf \Pi}}
\newcommand{\onebf}{{\mathbf 1}}
\newcommand{\fbf}{{\mathbf f}}
\newcommand{\ubf}{{\mathbf u}}

\newcommand{\NP}{\operatorname{NearestPt}}
\newcommand{\NS}{\operatorname{NearestSet}}
\newcommand{\bres}{\operatorname{Bres}}
\newcommand{\vol}{\operatorname{vol}}
\newcommand{\vor}{\operatorname{Vor}}
\newcommand{\coef}{\operatorname{Coef}}
\newcommand{\Int}{\operatorname{Int}}

\newtheorem{theorem}{Theorem}

\begin{document}


\section{Comments from reviewer 1}

\begin{enumerate}

\item \textbf{Comment:} Consistently write ``non-data-aided'' throughout the whole text \\
\textbf{Response:}

\item \textbf{Comment:} Page (P) 42, right column (RC), line (L) 50: Aside from this extension of \dots \\
\textbf{Response:} 


\item \textbf{Comment:} P 43, LC, L 44: \dots convincing response \dots \\
\textbf{Response:}

\item \textbf{Comment:} P 43, RC, L 2+4: \dots as multiple antenna systems \dots must withstand rapidly \dots \\
\textbf{Response:}

\item \textbf{Comment:} P 43, RC, L 21+28: \dots aided the design of \dots to other receiver components \dots \\
\textbf{Response:}

\item \textbf{Comment:} Conference papers: location missing \\
\textbf{Response:}

\item \textbf{Comment:} Some journal papers: month of publication missing \\
\textbf{Response:}

\item \textbf{Comment:} [8] \dots IEEE Commun. Letters \dots \\
\textbf{Response:}

\item \textbf{Comment:} [31] N. Noels, V. Lottici, \dots \\
\textbf{Response:}

\end{enumerate}

\section{Comments from reviewer 2}

\begin{enumerate}

\item \textbf{Comment:} In the abstract: non data-aided $\to$ change to: non-data-aided \\
\textbf{Response:}

\item \textbf{Comment:} ``we assume that the time-offset has been previously handled'' $\to$ perhaps some motivation for this assumption and the consequences with respect to the system model would be in place \\
\textbf{Response:}

\item \textbf{Comment:} ``such as binary phase shift keying (BPSK), \dots'' $\to$ mentioning BPSK and QPSK apart from general M-PSK is pure overhead in this stage, I would drop it \\
\textbf{Response:}

\item \textbf{Comment:} ``the data symbols are not known to the receiver and must also be estimated'' $\to$ I would drop the second part of the sentence because (e.g. Viterbi and Vterbi estimator) non-data-aided estimators typically do not actually estimate the unknown data symbols \\
\textbf{Response:}

\item \textbf{Comment:} p. 3: ``bounds that are valid asymptoticially'': typo in asymptotically \\
\textbf{Response:}

\item \textbf{Comment:} p. 3: ``all of the exiting bounds are applicable only in the fully non-data-aided case or in the completely data aided case \dots'' --> bounds for the common scenario where there is a mixture of pilot and data symbols can be found in "N. Noels, H. Steendam, M. Moeneclaey en H. Bruneel, Carrier Phase and Frequency Estimation for Pilot Symbol Assisted Transmission: Bounds and Algorithms, IEEE TSP, Vol. 53, No. 12, p. 4578-4587, Dec. 2005. \\
\textbf{Response:}

\item \textbf{Comment:} p. 3: ``the situations is'' --> drop `s' at the end of situations \\
\textbf{Response:}

\item \textbf{Comment:} p. 3: in continuous text I prefer the use of ``approached infinity'' rather than ``$\to \infty$'' \\
\textbf{Response:}

\item \textbf{Comment:} p. 3: last line: typo in ``convincing'' \\
\textbf{Response:}

\item \textbf{Comment:} p. 4: ``systems such as mulitple anntannae'' $\to$ should be ``multiple antennae'' \\
\textbf{Response:}

\item \textbf{Comment:} p. 4: ``the other reciever components'' $\to$ should be ``receiver components'' \\
\textbf{Response:}

\item \textbf{Comment:} p. 5 (and further throughout the text): ``pilots symbols" (with `s') and ``pilot symbols'' (without `s') are both used $\to$ use only 1, I prefer ``pilot symbols'' \\
\textbf{Response:}

\item \textbf{Comment:} I think it would be a good idea to specify the main notational conventions at the end of the introductory section \\
\textbf{Response:}

\item \textbf{Comment:} between (5) and (6): start new paragraph \\
\textbf{Response:}

\item \textbf{Comment:} before (6): drop the 'to' \\
\textbf{Response:}

\item \textbf{Comment:} before (7): change ``for given $\theta$'' to ``for given $\theta$ and $\rho$'' \\
\textbf{Response:}

\item \textbf{Comment:} what is the use of keeping the constants `A' and `B' in the expressions $\to$ can they not be dropped? \\
\textbf{Response:}

\item \textbf{Comment:} I would prefer to see comma's between two subindices; so, `$k,i$' rather than`'$ki$' \\
\textbf{Response:}

\item \textbf{Comment:} below (11) '$g_{ik}$' should be '$g_{ki}$' (change order of the subscripts) \\
\textbf{Response:}

\item \textbf{Comment:} I would prefer to see brackets around the functions of which the mean is computed (use of the operator $\mathbb E$) \\
\textbf{Response:}

\item \textbf{Comment:} p. 12: ``the amplitude estimation error with bias removed'' $\to$ bit strange formulation \\
\textbf{Response:}

\item \textbf{Comment:} p. 14: "Equivalently $\hat\theta$ is the maximiser of \dots" $\to$ could you please provide some further explanation (is the maximum of $\Re(Z(\theta)$) never negative? and why) \\
\textbf{Response:}

\item \textbf{Comment:} p. 15: the use of `A' as notation for a sample space is confusing since `A' has already been used as a constant \\
\textbf{Response:}

\item \textbf{Comment:} in general, I find the numerical results section rather short \\
\textbf{Response:}

\item \textbf{Comment:} although different figures are shown for M=2,4,8 (phase estimation), the effect of the constellation size is not addressed in the text  \\
\textbf{Response:}

\item \textbf{Comment:} p. 18, first sentence: ``when $L$ is sufficiently large'' $\to$ in these figures the value of $L$ is fixed, should this be ``when the SNR is sufficiently large''  \\
\textbf{Response:}

\item \textbf{Comment:} p. 18, first sentence: a qualitative explanation for the fact the theory and simulations diverge at low SNR would be appreciated  \\
\textbf{Response:}

\item \textbf{Comment:} p.18, last sentence of first paragraph: ``non-data aided'' $\to$ should be ``non-data-aided''  \\
\textbf{Response:}

\item \textbf{Comment:}  although the most important contribution of the paper is with respect to the biased amplitude estimate, only 1 figure is presented for amplitude estimation, namely for L=2048 and BPSK --> this seems not well balanced  \\
\textbf{Response:}

\item \textbf{Comment:}  a separate plot of the two terms that contribute to Fig. 4 would be welcome \\
\textbf{Response:}

\end{enumerate}

\section{Comments from reviewer 3}

\begin{enumerate}

\item \textbf{Comment:}  Eq. (14) would be easier to follow with the intermediate step inserted $\dots = (a_0s_i + w_i)/(a_0s_i) = \dots$ \\
\textbf{Response:}

\item \textbf{Comment:}  There are many formulas (starting from page 11) with the notation $\fracpart{x}$ meaning ``$x$ taken modulo $2\pi/M$''. These angle brackets don't strike the eye and can be easily confused with ordinary parentheses. To make them much more visible, the authors could use the same approach as for ``$x$ taken modulo $2\pi$'' - by using corresponding subscript, i.e. use $\fracpart{x}_{2\pi/M}$ instead of $\fracpart{x}$. \\
\textbf{Response:}

 \end{enumerate}

\section{Comments from reviewer 4}

\begin{enumerate}

\item \textbf{Comment:}  The variable  ``a'' in the equation 2 is not defined in the text \\
\textbf{Response:}

\item \textbf{Comment:}  There are typografical errors in the words  ``multiple antenna'' in line 2, page 2, section 1 \\
\textbf{Response:}

\item \textbf{Comment:}  And also in the word ``receiver" in line 28 of the same page.
 \\
\textbf{Response:}


\end{enumerate}


\bibliography{../../../bib/bib}

\end{document}
