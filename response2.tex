\documentclass{article}

\usepackage[pdftex]{graphicx}
%\usepackage{thumbpdf}
%\usepackage[naturalnames]{hyperref}
\usepackage{xr}
\externaldocument{../async}

\usepackage{bm, booktabs}
\usepackage{amsmath}%
\usepackage{amsfonts}%
\usepackage{amssymb}%
%\usepackage{amsthm}
\usepackage{mathrsfs}
%\usepackage{graphicx}
%\usepackage{setspace}
\usepackage{cite}
\usepackage{units}
	%nice looking units
%\usepackage{times}
\usepackage[normalem]{ulem}
%\usepackage{algorithmic}
%\usepackage[figure, vlined, linesnumbered]{algorithm2e}
\usepackage[vlined, linesnumbered]{algorithm2e}
%\usepackage{algorithm}

%%%%% set up the bibliography style
\bibliographystyle{../IEEEbst}
%\usepackage[square,comma,numbers]{natbib}

\newcommand{\round}[1]{\lfloor #1 \rceil}
\newcommand{\floor}[1]{\lfloor #1 \rfloor}
\newcommand{\ceil}[1]{\lceil #1 \rceil}
\newcommand{\reals}{{\mathbb R}}
\newcommand{\ints}{{\mathbb Z}}
\newcommand{\integers}{{\mathbb Z}}
\newcommand{\uy}{\underline{\bm{Y}}}
\newcommand{\uey}{\underline{Y}}
\newcommand{\sign}[1]{\mathtt{sign}(#1)}
\newcommand{\Qbf}{{\mathbf Q}}
\newcommand{\Bbf}{{\mathbf B}}
\newcommand{\Ibf}{{\mathbf I}}
\newcommand{\ybf}{{\mathbf y}}
\newcommand{\xbf}{{\mathbf x}}
\newcommand{\zbf}{{\mathbf z}}
\newcommand{\ebf}{{\mathbf e}}
\newcommand{\vbf}{{\mathbf v}}
\newcommand{\wbf}{{\mathbf w}}
\newcommand{\kbf}{{\mathbf k}}
\newcommand{\sbf}{{\mathbf s}}
\newcommand{\tbf}{{\mathbf t}}
\newcommand{\pbf}{{\mathbf p}}
\newcommand{\Pibf}{{\mathbf \Pi}}
\newcommand{\onebf}{{\mathbf 1}}
\newcommand{\fbf}{{\mathbf f}}
\newcommand{\ubf}{{\mathbf u}}

\newcommand{\NP}{\operatorname{NearestPt}}
\newcommand{\NS}{\operatorname{NearestSet}}
\newcommand{\bres}{\operatorname{Bres}}
\newcommand{\vol}{\operatorname{vol}}
\newcommand{\vor}{\operatorname{Vor}}
\newcommand{\coef}{\operatorname{Coef}}
\newcommand{\Int}{\operatorname{Int}}

\newtheorem{theorem}{Theorem}

\begin{document}


\section{Comments from reviewer 1}

\begin{enumerate}

\item \textbf{Comment:} Consistently write ``non-data-aided'' throughout the whole text \\
\textbf{Response:}

\item \textbf{Comment:} Page (P) 42, right column (RC), line (L) 50: Aside from this extension of \dots \\
\textbf{Response:} 


\item \textbf{Comment:} P 43, LC, L 44: \dots convincing response \dots \\
\textbf{Response:}

\item \textbf{Comment:} P 43, RC, L 2+4: \dots as multiple antenna systems \dots must withstand rapidly \dots \\
\textbf{Response:}

\item \textbf{Comment:} P 43, RC, L 21+28: \dots aided the design of \dots to other receiver components \dots \\
\textbf{Response:}

\item \textbf{Comment:} Conference papers: location missing \\
\textbf{Response:}

\item \textbf{Comment:} Some journal papers: month of publication missing \\
\textbf{Response:}

\item \textbf{Comment:} [8] \dots IEEE Commun. Letters \dots \\
\textbf{Response:}

\item \textbf{Comment:} [31] N. Noels, V. Lottici, \dots \\
\textbf{Response:}

\end{enumerate}

\section{Comments from reviewer 2}

\begin{enumerate}

\item \textbf{Comment:} In the abstract: non data-aided --> change to: non-data-aided \\
\textbf{Response:}

\item \textbf{Comment:} In the abstract: non data-aided --> change to: non-data-aided \\
\textbf{Response:}

\item \textbf{Comment:} In the abstract: non data-aided --> change to: non-data-aided \\
\textbf{Response:}

- In the introductory section:
     * "we assume that the time-offset has been previously handled" --> perhaps some motivation for this assumption and the consequences with respect to the system model would be in place
     * "such as binary phase shift keying (BPSK),...." --> mentioning BPSK and QPSK apart from general M-PSK is pure overhead in this stage, I would drop it
     * "the data symbols are not known to the receiver and must also be estimated" --> I would drop the second part of the sentence because (e.g. Viterbi and Vterbi estimator) non-data-aided estimators typically do not actually estimate the unknown data symbols
     * p. 3: "bounds that are valid asymptoticially": typo in asymptotically
     * p. 3: "all of the exiting bounds are applicable only in the fully non-data-aided case or in the completely data aided case..." --> bounds for the common scenario where there is a mixture of pilot and data symbols can be found in "N. Noels, H. Steendam, M. Moeneclaey en H. Bruneel, "Carrier Phase and Frequency Estimation for Pilot Symbol Assisted Transmission: Bounds and Algorithms, IEEE TSP, Vol. 53, No. 12, p. 4578-4587, Dec. 2005. "
     * p. 3: "the situations is" --> drop 's' at the end of situations
     * p. 3: in continuous text I prefer the use of "approached infinity" rather than "--> \infty"
     * p. 3: last line: typo in "convincing"
     * p. 4: "systems such as mulitple anntannae" --> should be "multiple antennae"
     * p. 4: "the other reciever components" --> should be "receiver components"
     * p. 5 (and further throughout the text): "pilots symbols" (with 's') and "pilot symbols" (without 's') are both used --> use only 1, I prefer "pilot symbols"
     * I think it would be a good idea to specify the main notational conventions at the end of the introductory section
- section II:
     * between (5) and (6): start new paragraph
     * before (6): drop the 'to'
     * before (7): change "for given \theta" to "for given \theta and \rho"
     * what is the use of keeping the constants 'A' and 'B' in the expressions --> can they not be dropped?
     * I would prefer to see comma's between two subindices; so, 'k,i' rather than 'ki'
     * below (11) 'g_ik' should be 'g_ki' (change order of the subscripts)
- section III:
     * I would prefer to see brackets around the functions of which the mean is computed (use of the operator \mathbb E)
     * p. 12: "the amplitude estimation error with bias removed" --> bit strange formulation
- section IV:
     * p. 14: "Equivalently \hat\theta is the maximiser of ..." --> could you please provide some further explanation (is the maximum of \Re(Z(\theta)) never negative? and why)
     * p. 15: the use of 'A' as notation for a sample space is confusing since 'A' has already been used as a constant
- section VI:
     * in general, I find the numerical results section rather short
     * although different figures are shown for M=2,4,8 (phase estimation), the effect of the constellation size is not addressed in the text
     * p. 18, first sentence: "when L is sufficiently large" --> in these figures the value of L is fixed, should this be "when the SNR is sufficiently large"
     * a qualitative explanation for the fact the theory and simulations diverge at low SNR would be appreciated
     * p.18, last sentence of first paragraph: non-data aided --> should be "non-data-aided"
     * although the most important contribution of the paper is with respect to the biased amplitude estimate, only 1 figure is presented for amplitude estimation, namely for L=2048 and BPSK --> this seems not well balanced
     * a separate plot of the two terms that contribute to Fig. 4 would be welcome

\end{enumerate}

\section{Comments from reviewer 3}


\bibliography{../../../bib/bib}

\end{document}
