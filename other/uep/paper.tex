%\documentclass[a4paper,10pt]{article}
%\documentclass[draftcls, onecolumn, 11pt]{IEEEtran}
\documentclass[journal]{IEEEtran}

\usepackage{mathbf-abbrevs}
%\newcommand {\tbf}[1] {\textbf{#1}}
%\newcommand {\tit}[1] {\textit{#1}}
%\newcommand {\tmd}[1] {\textmd{#1}}
%\newcommand {\trm}[1] {\textrm{#1}}
%\newcommand {\tsc}[1] {\textsc{#1}}
%\newcommand {\tsf}[1] {\textsf{#1}}
%\newcommand {\tsl}[1] {\textsl{#1}}
%\newcommand {\ttt}[1] {\texttt{#1}}
%\newcommand {\tup}[1] {\textup{#1}}
%
%\newcommand {\mbf}[1] {\mathbf{#1}}
%\newcommand {\mmd}[1] {\mathmd{#1}}
%\newcommand {\mrm}[1] {\mathrm{#1}}
%\newcommand {\msc}[1] {\mathsc{#1}}
%\newcommand {\msf}[1] {\mathsf{#1}}
%\newcommand {\msl}[1] {\mathsl{#1}}
%\newcommand {\mtt}[1] {\mathtt{#1}}
%\newcommand {\mup}[1] {\mathup{#1}}

%some math functions and symbols
\newcommand{\reals}{{\mathbb R}}
\newcommand{\expect}{{\mathbb E}}
\newcommand{\indicator}{{\mathbf 1}}
\newcommand{\ints}{{\mathbb Z}}
\newcommand{\complex}{{\mathbb C}}
\newcommand{\integers}{{\mathbb Z}}
\newcommand{\sign}[1]{\mathtt{sign}\left( #1 \right)}
\newcommand{\NP}{\operatorname{NPt}}
\newcommand{\erf}{\operatorname{erf}}
\newcommand{\NS}{\operatorname{NearestSet}}
\newcommand{\bres}{\operatorname{Bres}}
\newcommand{\vol}{\operatorname{vol}}
\newcommand{\vor}{\operatorname{Vor}}
%\newcommand{\re}{\operatorname{Re}}
%\newcommand{\im}{\operatorname{Im}}




\newcommand{\term}{\emph}
\newcommand{\var}{\operatorname{var}}
%\newcommand{\prob}{{\mathbb P}}
\newcommand{\prob}{{\operatorname{Pr}}}

%distribution fucntions
\newcommand{\projnorm}{\operatorname{ProjectedNormal}}
\newcommand{\vonmises}{\operatorname{VonMises}}
\newcommand{\wrapnorm}{\operatorname{WrappedNormal}}
\newcommand{\wrapunif}{\operatorname{WrappedUniform}}

\newcommand{\selectindicies}{\operatorname{selectindices}}
\newcommand{\sortindicies}{\operatorname{sortindices}}
\newcommand{\largest}{\operatorname{largest}}
\newcommand{\quickpartition}{\operatorname{quickpartition}}
\newcommand{\quickpartitiontwo}{\operatorname{quickpartition2}}

%some commonly used underlined and
%hated symbols
\newcommand{\uY}{\ushort{\mbf{Y}}}
\newcommand{\ueY}{\ushort{Y}}
\newcommand{\uy}{\ushort{\mbf{y}}}
\newcommand{\uey}{\ushort{y}}
\newcommand{\ux}{\ushort{\mbf{x}}}
\newcommand{\uex}{\ushort{x}}
\newcommand{\uhx}{\ushort{\mbf{\hat{x}}}}
\newcommand{\uehx}{\ushort{\hat{x}}}

% Brackets
\newcommand{\br}[1]{{\left( #1 \right)}}
\newcommand{\sqbr}[1]{{\left[ #1 \right]}}
\newcommand{\cubr}[1]{{\left\{ #1 \right\}}}
\newcommand{\abr}[1]{\left< #1 \right>}
\newcommand{\abs}[1]{{\left\vert #1 \right\vert}}
\newcommand{\sabs}[1]{{\vert #1 \vert}}
\newcommand{\floor}[1]{{\left\lfloor #1 \right\rfloor}}
\newcommand{\ceiling}[1]{{\left\lceil #1 \right\rceil}}
\newcommand{\ceil}[1]{\left\lceil #1 \right\rceil}
\newcommand{\round}[1]{{\left\lfloor #1 \right\rceil}}
\newcommand{\magn}[1]{\left\| #1 \right\|}
\newcommand{\fracpart}[1]{\left\langle #1 \right\rangle}
\newcommand{\sfracpart}[1]{\langle #1 \rangle}


% Referencing
\newcommand{\refeqn}[1]{\eqref{#1}}
\newcommand{\reffig}[1]{Figure~\ref{#1}}
\newcommand{\reftable}[1]{Table~\ref{#1}}
\newcommand{\refsec}[1]{Section~\ref{#1}}
\newcommand{\refappendix}[1]{Appendix~\ref{#1}}
\newcommand{\refchapter}[1]{Chapter~\ref{#1}}

\newcommand {\figwidth} {100mm}
\newcommand {\Ref}[1] {Reference~\cite{#1}}
\newcommand {\Sec}[1] {Section~\ref{#1}}
\newcommand {\App}[1] {Appendix~\ref{#1}}
\newcommand {\Chap}[1] {Chapter~\ref{#1}}
\newcommand {\Lem}[1] {Lemma~\ref{#1}}
\newcommand {\Thm}[1] {Theorem~\ref{#1}}
\newcommand {\Cor}[1] {Corollary~\ref{#1}}
\newcommand {\Alg}[1] {Algorithm~\ref{#1}}
\newcommand {\etal} {\emph{~et~al.}}
\newcommand {\bul} {$\bullet$ }   % bullet
\newcommand {\fig}[1] {Figure~\ref{#1}}   % references Figure x
\newcommand {\imp} {$\Rightarrow$}   % implication symbol (default)
\newcommand {\impt} {$\Rightarrow$}   % implication symbol (text mode)
\newcommand {\impm} {\Rightarrow}   % implication symbol (math mode)
\newcommand {\vect}[1] {\mathbf{#1}} 
\newcommand {\hvect}[1] {\hat{\mathbf{#1}}}
\newcommand {\del} {\partial}
\newcommand {\eqn}[1] {Equation~(\ref{#1})} 
\newcommand {\tab}[1] {Table~\ref{#1}} % references Table x
\newcommand {\half} {\frac{1}{2}} 
\newcommand {\ten}[1] {\times10^{#1}}
\newcommand {\bra}[2] {\mbox{}_{#2}\langle #1 |}
\newcommand {\ket}[2] {| #1 \rangle_{#2}}
\newcommand {\Bra}[2] {\mbox{}_{#2}\left.\left\langle #1 \right.\right|}
\newcommand {\Ket}[2] {\left.\left| #1 \right.\right\rangle_{#2}}
\newcommand {\im} {\mathrm{Im}}
\newcommand {\re} {\mathrm{Re}}
\newcommand {\braket}[4] {\mbox{}_{#3}\langle #1 | #2 \rangle_{#4}} 
\newcommand{\dotprod}[2]{ \left\langle #1 , #2 \right\rangle}
\newcommand {\trace}[1] {\text{tr}\left(#1\right)}

% spell things correctly
\newenvironment{centre}{\begin{center}}{\end{center}}
\newenvironment{itemise}{\begin{itemize}}{\end{itemize}}

%%%%% set up the bibliography style
\bibliographystyle{IEEEbib}
%\bibliographystyle{uqthesis}  % uqthesis bibliography style file, made
			      % with makebst

%%%%% optional packages
\usepackage[square,comma,numbers,sort&compress]{natbib}
		% this is the natural sciences bibliography citation
		% style package.  The options here give citations in
		% the text as numbers in square brackets, separated by
		% commas, citations sorted and consecutive citations
		% compressed 
		% output example: [1,4,12-15]

%\usepackage{cite}		
			
\usepackage{units}
	%nice looking units
		
\usepackage{booktabs}
		%creates nice looking tables
		
\usepackage{ifpdf}
\ifpdf
  \usepackage[pdftex]{graphicx}
  %\usepackage{thumbpdf}
  %\usepackage[naturalnames]{hyperref}
\else
	\usepackage{graphicx}% standard graphics package for inclusion of
		      % images and eps files into LaTeX document
\fi

\usepackage{amsmath,amsfonts,amssymb, amsthm, bm} % this is handy for mathematicians and physicists
			      % see http://www.ams.org/tex/amslatex.html

		 
\usepackage[vlined, linesnumbered]{algorithm2e}
	%algorithm writing package
	
\usepackage{mathrsfs}
%fancy math script

%\usepackage{ushort}
%enable good underlining in math mode

%------------------------------------------------------------
% Theorem like environments
%
\newtheorem{theorem}{Theorem}
%\theoremstyle{plain}
\newtheorem{acknowledgement}{Acknowledgement}
%\newtheorem{algorithm}{Algorithm}
\newtheorem{axiom}{Axiom}
\newtheorem{case}{Case}
\newtheorem{claim}{Claim}
\newtheorem{conclusion}{Conclusion}
\newtheorem{condition}{Condition}
\newtheorem{conjecture}{Conjecture}
\newtheorem{corollary}{Corollary}
\newtheorem{criterion}{Criterion}
\newtheorem{definition}{Definition}
\newtheorem{example}{Example}
\newtheorem{exercise}{Exercise}
\newtheorem{lemma}{Lemma}
\newtheorem{notation}{Notation}
\newtheorem{problem}{Problem}
\newtheorem{proposition}{Proposition}
\newtheorem{property}{Property}
\newtheorem{remark}{Remark}
\newtheorem{solution}{Solution}
\newtheorem{summary}{Summary}
%\numberwithin{equation}{section}
%--------------------------------------------------------


\usepackage{xr}
\externaldocument{paper2}

\usepackage{amsmath,amsfonts,amssymb, amsthm, bm}

\usepackage[square,comma,numbers,sort&compress]{natbib}

\newcommand{\sgn}{\operatorname{sgn}}
\newcommand{\sinc}{\operatorname{sinc}}
\newcommand{\rect}[1]{\operatorname{rect}\left(#1\right)}

%opening
\title{Carrier phase and amplitude estimation for phase shift keying with unequal error protection}
\author{Robby McKilliam, Andre Pollok, Bill Cowley
\thanks{
Supported under the Australian Government’s Australian Space Research Program.
Robby McKilliam, Andre Pollok and Bill Cowley are with the Institute for Telecommunications Research, The University of South Australia, SA, 5095.}}

\begin{document}

\maketitle

\begin{abstract}
We consider least squares estimators of carrier phase and amplitude from a noisy communications signal.  We focus on signaling constellations that have symbols evenly distributed on the complex unit circle, i.e., $M$-phase shift keying, and consider specifically the case where a number of different constellations sizes $M$ are used simultaneously.  We describe an algorithm to compute the least squares estimators of carrier phase and amplitude that requires only $O(L \log L)$ arithmetic operations, where $L$ is the number of recieved symbols.   
\end{abstract}
\begin{IEEEkeywords}
Coherent detection, noncoherent detection, phase shift keying, asymptotic statistics
\end{IEEEkeywords}

\section{Introduction}

In passband communication systems the transmitted signal typically undergoes time offset (delay), phase shift and attenuation (amplitude change).  These effects must be compensated for at the receiver. In this paper we assume that the time offset has been previously handled, and we focus on estimating the phase shift and attenuation.  We consider signalling constellations that have symbols evenly distributed on the complex unit circle such as binary phase shift keying (BPSK), quaternary phase shift keying (QPSK) and $M$-ary phase shift keying ($M$-PSK).  In this case, the transmitted symbols take the form,
\[
s_i = e^{j u_i},
\]
where $j = \sqrt{-1}$ and $u_i$ is from the set $\{0, \tfrac{2\pi}{M}, \dots, \tfrac{2\pi(M-1)}{M}\}$ and $M \geq 1$ is the size of the constellation.  We assume that some of the transmitted symbols are \emph{pilot symbols} known to the receiver and the remainder are information carrying \emph{data symbols} with phase that is unknown to the receiver.  So,
\[
s_i = \begin{cases}
p_i & i \in P \\
d_i & i \in D,
\end{cases}
\]
where $P$ is the set of indices describing the position of the pilot symbols $p_i$, and $D$ is a set of indices describing the position of the data symbols $d_i$.  The sets $P$ and $D$ are disjoint, i.e. $P \cap D = \emptyset$, and $L = \abs{P \cup D}$ is the total number of symbols transmitted.  The data symbols are further separated into subsets according to the size of the constellation used to modulate each symbol.  Let $D_2,D_3,\dots$ denote a partition of the indices in $D$ such that $D_2$ is the set of indices corresponding to data symbols modulated with $2$-PSK (i.e BSPK), and $D_3$ is the set of symbols modulated with $3$-PSK, and $D_4$ is the set of symbols modulated with $4$-PSK (i.e. QPSK) and so on.  Let $G$ be the set of integers for which $D_m$ is not empty, i.e. $D_m \neq \emptyset$ whenever $m \in G$.  We have 
\[
D = \bigcup_{m=2}^{\infty}D_m = \bigcup_{m\in G} D_m.
\]

We assume that time offset estimation has been performed and that $L$ noisy $M$-PSK symbols are observed by the receiver.  The received signal is then,
\begin{equation}\label{eq:sigmod}
y_i = a_0 s_i + w_i, \qquad i \in P \cup D,
\end{equation}
where $w_i$ is noise and $a_0 = \rho_0 e^{j\theta_0}$ is a complex number representing both carrier phase $\theta_0$ and amplitude $\rho_0$ (by definition $\rho_0$ is a positive real number).  Our aim is to estimate $a_0$ from the noisy symbols $\{ y_i, i \in P \cup D \}$.  Complicating matters is that the data symbols $\{d_i, i \in D\}$ are not known to the reciever and must also be estimated.  Estimation problems of this type have undergone extensive prior study~\cite{ViterbiViterbi_phase_est_1983,Cowley_ref_sym_carr_1998,Wilson1989,Makrakis1990,Liu1991,Mackenthun1994,Sweldens2001,McKilliamLinearTimeBlockPSK2009,Divsalar1990}.  A practical approach is the least squares estimator, that is, the minimisers of the sum of squares function
\begin{equation}\label{eq:SSdefn}
\begin{split}
SS(a, &\{d_i, i \in D\}) = \sum_{i \in P \cup D} \abs{ y_i - a s_i }^2  \\
&= \sum_{i \in P} \abs{ y_i - a s_i }^2 + \sum_{i \in D} \abs{ y_i - a d_i }^2, 
%\\ &= \sum_{i \in P} \abs{ y_i - a s_i }^2 + \sum_{m \in G} \sum_{i \in D_m} \abs{ y_i - a d_i }^2
\end{split}
\end{equation}
where $\abs{x}$ denotes the magnitude of the complex number $x$.  The least squares estimator is also the maximum likelihood estimator under the assumption that the noise sequence $\{w_i, i \in \ints\}$ is additive white and Gaussian.  However, as we will show, the estimator works well under less stringent assumptions.  %It is the least squares estimator that we primarily study in this paper.

The existing literature~\cite{Mackenthun1994,Cowley_ref_sym_carr_1998,ViterbiViterbi_phase_est_1983,Sweldens2001,Wilson1989,Makrakis1990,Liu1991} mostly considers what is called \emph{noncoherent detection} where no pilot symbols exist ($P = \emptyset$ where $\emptyset$ is the empty set) and where only one constellation size is used, i.e., $G = \{M\}$ contains precisely one element.  In the noncoherent setting \emph{differential encoding} is often used, and for this reason the estimation problem has been called \emph{multiple symbol differential detection}.  A popular approach is the so called \emph{non-data aided}, sometimes also called \emph{non-decision directed}, estimator based on the paper of Viterbi and Viterbi~\cite{ViterbiViterbi_phase_est_1983}.  The idea is to `strip' the modulation from the recieved signal by taking $y_i / \abs{y_i}$ to the power of $M$.  A function $F: \reals \mapsto \reals$ is chosen and the estimator of the carrier phase $\theta_0$ is taken to be $\tfrac{1}{M}\angle{A}$ where $\angle$ denotes the complex argument and
\[
A = \frac{1}{L}\sum_{i \in P \cup D} F(\abs{y_i}) \big(\tfrac{y_i}{\abs{y_i}}\big)^M.
\]
Various choices for $F$ are suggested in~\cite{ViterbiViterbi_phase_est_1983} and a statistical analysis is presented.  It is not entirely obvious how pilot symbols or data symbols from mulitple constellations should be included in this estimator.  One approach is to choose,
\[
A = \frac{1}{L} \sum_{i \in P} F(\abs{y_i}) \tfrac{y_i}{y_i} p_i^* +  \frac{1}{L} \sum_{m \in G} \sum_{i \in D_m} F(\abs{y_i}) \big(\tfrac{y_i}{\abs{y_i}}\big)^m.
\]
The simulations we present in Section~\ref{sec:simulations} suggest that the accuracy of this estimator is poor when compared to the least squares estimator.  %This problem does not occur with the least square estimator.

An important paper is by Mackenthun~\cite{Mackenthun1994} who described an algorithm to compute the least squares estimator requiring only $O(L \log L)$ arithmetic operations.  Sweldens~\cite{Sweldens2001} rediscovered Mackenthun's algorithm in 2001.  Both Mackenthun and Swelden's considered only the noncoherent setting and where only one constellation size is used, but we show in Section~\ref{sec:least-squar-estim}~that Mackenthun's algorithm can be modified to include pilot symbols and multiple constellation sizes. Our model includes the noncoherent case by setting the number of pilot symbols to zero, that is, putting $P = \emptyset$.  


%It is worth commenting on our use of $\prob$ rather than the more usual $\expect$ or $E$ for the expected value operator.  The notation comes from Pollard~\cite[Ch 1]{Pollard_users_guide_prob_2002}.  The notation is good because it removes unecessary distinction between `probability' and expectation.  Given a random variable $X$ with cumulative density function $F$, the probability of an event, say $X \in S$, where $S$ is some subset of the range of $X$, is 
%\[
%\prob \indicator \{X \in S\} = \int \indicator \{X \in S\}(x) dF(x) = \int_{S} dF(x)
%\]
%where $\indicator \{X \in S\}$ is the indicator function of the set $S$, i.e $\indicator \{X \in S\}(x) = 1$ when the argument $x \in S$ and zero otherwise.  We will usually drop the $\onebf$ and simply write $\prob \{ X \in S \}$ to mean $\prob \onebf\{ X \in S \}$.  To illustrate the utility of this notation, Markov's inequality becomes 
%\[
%\prob \{ \abs{X} > \delta \}  \leq \prob \frac{\abs{X}}{\delta}\onebf\{ \abs{X} > \delta \} \leq \frac{1}{\delta}\prob\abs{X},
%\]
%where $\frac{\abs{X}}{\delta}\onebf\{ \abs{X} > \delta \}(x)$ is the function equal to $\abs{x}/\delta$ when the argument $x > \delta$ and zero otherwise.

\section{Mackenthun's algorithm with pilots}\label{sec:least-squar-estim}

In this section we derive an algorithm to compute the least squares estimator of the carrier phase and amplitude.  Our algorithm is motivated by the algorithm of Mackenthun~\cite{Mackenthun1994} who considered the noncoherent setting and where only one constellation size is used for the data symbols.  For the purpose of anaylsing computational complexity, we will assume that the number of data symbols $\abs{D}$ is proportional to the total number of symbols $L$, so that, for example, $O(L) = O(\abs{D})$.  In this case Mackentun's algorithm requires $O(L \log L)$ arithmetic operations.

%We use order notation in the standard way, that is, for functions $h$ and $g$, we write $h(N) = O(g(N))$ to mean that there exists a constant $K > 0$ and a finite $N_0$ such that $h(N) \leq K g(N)$ for all $N > N_0$.

%In Section~\ref{sec:line-time-algor} we will show that a full sort is not neccesary, and that the least squares estimator can be implemented in $O(L)$ operations.

Define the sum of squares function
\begin{align}
SS(a, &\{d_i, i \in D\}) = \sum_{i \in P \cup D} \abs{ y_i - a s_i }^2 \nonumber \\
&= \sum_{i \in P \cup D} \abs{y_i}^2 - a s_i y_i^* - a^* s_i^* y_i + aa^*, \label{eq:SS}
\end{align}
where $*$ denotes the complex congugate.  Fixing the data symbols $\{d_i, i \in D\}$, differentiating with respect to $a^*$, and setting the resulting equation to zero we find the least squares estimator of $a_0$ as a function of $\{d_i, i \in D\}$,
\begin{equation}\label{eq:hata}
\hat{a}(\{d_i, i \in D\}) = \frac{1}{L} \sum_{i \in P \cup D} y_i s_i^* = \frac{1}{L} Y
\end{equation}
where $L = \abs{P \cup D}$ is the total number of symbols transmitted, and to simplify our notation we have put 
\[
Y = \sum_{i \in P \cup D} y_i s_i^* = \sum_{i \in P } y_i p_i^* + \sum_{i \in D } y_i d_i^*.
\]  
Note that $Y$ is a function of the unknown data symbols $\{ d_i, i \in D\}$ and we could write $Y(\{ d_i, i \in D\})$, but have chosen to surpress the argument $(\{ d_i, i \in D\})$ for notational clarity.  Substituting $\frac{1}{L}Y$ for $a$ into~\eqref{eq:SS} we obtain the $SS$ conditioned on minimisation with respect to $a$,
\begin{equation}\label{eq:SSdatasymbols}
SS(\{d_i, i \in D\}) = A - \frac{1}{L}\abs{Y}^2,
\end{equation}
where $A = \sum_{i \in P \cup D}\abs{y_i}^2$ is a constant.  Observe that given candidate values for the data symbols, we can compute the corresponding $SS(\{d_i, i \in D\})$ in $O(L)$ arithmetic operations.  It turns out that there are atmost $(M+1)\abs{D}$ candidate values of the least squares estimator of the data symbols~\cite{Sweldens2001,Mackenthun1994}.  %When the number of data symbols is not small, this set is substantially smaller than the entire set of possible transmitted symbols, which is of size $M^{\abs{D}}$.

To see this, let $a = \rho e^{j\theta}$ where $\rho$ is a nonnegative real.  Now,
\begin{align}
SS(\rho, \theta, &\{d_i, i \in D\}) = \sum_{i \in P \cup D} \abs{ y_i - \rho e^{j\theta} s_i }^2  \nonumber \\
&= \sum_{i \in P} \abs{ y_i - \rho e^{j\theta} p_i }^2 + \sum_{i \in D} \abs{ y_i - \rho e^{j\theta} d_i }^2. \label{eq:SSallparams}
\end{align}
For given $\theta$, the least squares estimator of the $i$th data symbol $d_i \in D_m$ for some $m \in G$ is given by minimising $\abs{ y_i - \rho e^{j\theta} d_i }^2$, that is,
\begin{equation}\label{eq:hatdfinxtheta}
\hat{d}_i(\theta) = e^{j\hat{u}_i(\theta)} \qquad \text{where} \qquad \hat{u}_i(\theta) = \round{\angle( e^{-j\theta}y_i)}_{m},
\end{equation}
where $\angle(\cdot)$ denotes the complex argument (or phase), and $\round{\cdot}_{m}$ rounds its argument to the nearest multiple of $\frac{2\pi}{m}$.  If the function $\operatorname{round}(\cdot)$ takes its argument to the nearest integer then,
\[
\round{x}_{m} = \tfrac{2\pi}{m}\operatorname{round}\left(\tfrac{m}{2\pi}x\right).
\] 
Note that $\hat{d}_i(\theta)$ does not depend on $\rho$.  As defined, $\hat{u}_i(\theta)$ is not strictly inside the set $\{0, \tfrac{2\pi}{m}, \dots, \tfrac{2\pi(m-1)}{m}\}$, but this is not of consequence, as we intend its value to be considered equivalent modulo $2\pi$.  With this in mind,
\[
\hat{u}_i(\theta) = \round{\angle{y_i} - \theta }_{m}
\]
which is equivalent to the definition from~\eqref{eq:hatdfinxtheta} modulo $2\pi$.

We only require to consider $\theta$ in the interval $[0, 2\pi)$.  Consider how $\hat{d}_i(\theta)$ changes as $\theta$ varies from $0$ to $2\pi$.  Let $b_i = \hat{d}_i(0)$ and let 
\[
z_i = \angle{y_i} - \hat{u}_i(0) = \angle{y_i} - \round{\angle{y_i}}_{m}.
\]
Then,
\begin{equation}\label{eq:uicombos}
\hat{d}_i(\theta) = 
\begin{cases}
b_i, &  0 \leq \theta < z_i + \frac{\pi}{m} \\
b_i e^{-j2\pi/m}, & z_i + \frac{\pi}{m} \leq \theta < z_i + \frac{3\pi}{m} \\ 
\vdots & \\
b_i e^{-j2\pi k /m}, & z_i + \frac{\pi(2k - 1)}{m} \leq \theta < z_i + \frac{\pi(2k + 1)}{m}  \\ 
\vdots & \\
b_i e^{-j2\pi}, &  z_i + \frac{\pi(2m - 1)}{m} \leq \theta < 2\pi. \\
\end{cases}
\end{equation}

Let 
\[
f(\theta) = \{ \hat{d}_i(\theta), i \in D \}
\]
be a function mapping the interval $[0, 2\pi)$ to a sequence of phase shift keyed symbols indexed by the elements of $D$.  Observe that $f(\theta)$ is piecewise continuous.  The subintervals of $[0, 2\pi)$ over which $f(\theta)$ remains contant are determined by the values of $\{z_i, i \in D\}$.  Let
\[
S = \{ f(\theta) \mid \theta \in [0, 2 \pi) \}
\]
be the set of all sequences $f(\theta)$ as $\theta$ varies from $0$ to $2\pi$.  If $\hat{\theta}$ is the least squares estimator the phase then $S$ contains the sequence $\{ \hat{d}_i(\hat{\theta}), i \in D \}$ corresponding to the least squares estimator of the data symbols, i.e., $S$ contains the minimiser of~\eqref{eq:SSdatasymbols}.  Observe from~\eqref{eq:uicombos} that there are atmost $\sum_{m\in G}(m+1)\abs{D_m} = O(L)$ sequences in $S$, because there are $m+1$ sequences for each $i \in D_m$.

The sequences in $S$ can be enumerated as follows.  Let 
\[
T = \bigcup_{m\in G} \bigcup_{ i \in D_m }\bigcup_{k=1}^{m}\left\{ (z_i + \tfrac{\pi(2k - 1)}{m}, i, m) \right\}
\]
be a set of triples with first element a real number, second element from $D$ and third element from $G$.  Let 
\[
H = \abs{T} = \sum_{m\in G}(m+1)\abs{D_m} = O(L)
\]  
denote the number of elements in $T$.  Let $t_1,\dots,t_H$ be an enumeration of the triples from $T$ sorted in accending order of the first element in the triple, that is, if $t_{i} = (t_{i1},t_{i2},t_{i3})$ then 
\[
t_{i1} \leq t_{ki}
\]
whenever $i < k$.  Put $\sigma(i) = t_{i2}$ and $m(i) = t_{i3}$ so that $\sigma(1), \dots, \sigma(H)$ and $m(1),\dots,m(H)$ correspond respectively with the second and third elements from the triples $t_1,\dots,t_H$ 
%So $a_1,\dots,a_H$ is a sequence of real numbers, $b_1,\dots,b_H$ is a sequence of integers from $D$ and $c_1,\dots,c_H$ is a sequence

The first sequence in $S$ is 
\[
f_1 = f(0) = \{ \hat{d}_i(0), i \in D \} = \{ b_i, i \in D \}.
\]  
The next sequence $f_1$ is given by replacing the element $b_{\sigma(1)}$ in $f_0$ with $b_{\sigma(1)}e^{-j2\pi/m(1)}$.  Given a sequence $x$ we use $x e_{k}$ to denote $x$ with the $\sigma(k)$th element replaced by $x_{\sigma(k)} e^{-j2\pi/m(k)}$.  Using this notation,  
\[
f_2 = f_1 e_1.
\] 
The next sequence in $S$ is correspondingly 
\[
f_3 = f_1 e_{1} e_{2} = f_2 e_{2},
\]
and the $k$th sequence is
\begin{equation}\label{eq:fkrec}
f_{k+1} = f_{k} e_{k}.
\end{equation}
In this way, all $(M+1)\abs{D}$ sequences in $S$ can be recursively enumerated.

We want to find the $f_k \in S$ corresponding to the minimiser of~(\ref{eq:SSdatasymbols}).  A na\"{\i}ve approach would be to compute $SS(f_k)$ for each $k \in \{1,\dots,H\}$.  Computing $SS(f_k)$ for any particular $k$ requires $O(L)$ arithmetic operations.  So, this na\"{\i}ve approach would require $O(LH) = O(L^2)$ operations in total.  Following Mackenthun~\cite{Mackenthun1994}, we show how $SS(f_k)$ can be computed recursively.

Let,
\begin{equation}\label{eq:SSfk}
SS(f_k) = A - \frac{1}{L}\abs{Y_k}^2,
\end{equation}
where, 
\begin{align*}
Y_k = Y( f_k ) &= \sum_{i \in P} y_i p_i^*  + \sum_{i \in D} y_i f_{ki}^* \\
&= B + \sum_{i \in D}g_{ki},
\end{align*}
where $B = \sum_{i \in P} y_i p_i^*$ is a constant, independent of the data symbols, and $f_{ki}$ denotes the $i$th symbol in $f_k$, and for convenience, we put $g_{ki}  = y_i f_{ki}^*$.  Letting $g_{k}$ be the sequence $\{g_{ik}, i \in D\}$ we have, from~\eqref{eq:fkrec}, that $g_k$ satisfies the recursive equation
\[
g_{k+1} = g_{k} e_{k}^*,
\]
where $g_{k} e_{k}^*$ indicates the sequence $g_k$ with the $\sigma(k)$th element replaced by $g_{k \sigma(k)}e^{j2\pi/m(k)}$.  Now,
\[
Y_1 = B + \sum_{i \in D} g_{1i}
\] 
can be computed in $O(L)$ operations, and
\begin{align*}
Y_2 &= B + \sum_{i \in D} g_{1i} \\
&= B +  (e^{j2\pi/m(1)} - 1)g_{1\sigma(1)} + \sum_{i \in D} g_{1i} \\
&= Y_0 + (e^{j2\pi/m(1)} - 1)g_{1\sigma(1)},
\end{align*}
In general,
\[
Y_{k+1} = Y_k + (e^{j2\pi/m(k)} - 1) g_{k\sigma(k)}.
\]
So, each $Y_k$ can be computed from it predecessor $Y_{k-1}$ in a constant number of operations.  Given $Y_k$, the value of $SS(f_k)$ can be computed in a constant number of operations using~\eqref{eq:SSfk}.  Let $\hat{k} = \arg\min SS(f_k)$.  The least squares estimator of the complex amplitude is then computed according to~\eqref{eq:hata},
\begin{equation}\label{eq:ahatYhat}
\hat{a} = \frac{1}{L} Y_{\hat{k}}.
\end{equation}
Psuedocode is given in Algorithm~\ref{alg:loglinear}.  BLERG

%Line~\ref{alg_sortindices} contains the function $\operatorname{sortindicies}$ that, given $z = \{z_i, i \in D\}$, returns the permutation $\sigma$ as described in~\eqref{eq:sigmasortind}.  The $\operatorname{sortindicies}$ function requires sorting $\abs{D}$ elements.  This requires $O(L \log L)$ operations.  The $\operatorname{sortindicies}$ function is the primary bottleneck in this algorithm when $L$ is large.  The loops on lines~\ref{alg_loop_setup} and~\ref{alg_loop_search} and the operations on lines~\ref{alg_Y} to lines~\ref{alg_Q} all require $O(L)$ or less operations.  %In the next sections we will show how to the sortinces function can be avoided.  This leads to an algorithm that requires only $O(L)$ operations.

\begin{algorithm}[t] \label{alg:loglinear}
\SetAlCapFnt{\small}
\SetAlTitleFnt{}
\caption{Mackenthun's algorithm with pilot symbols}
\DontPrintSemicolon
\KwIn{$\{y_i, i \in P \cup D \}$}
\For{$m \in G$ \nllabel{alg_loop_setup}}{
\For{$i \in D_m$ }{
$\phi = \angle{y_i}$ \;
$u = \round{\phi}_m $ \;
$g_i = y_i e^{-j u}$ \;
\For{$k = 1, \dots, m$ }{ $t_{mik} =  (z_i + \tfrac{\pi(2k + 1)}{m}, i, m)$} 
}
}
$\operatorname{sort}(t_1,\dots,t_H)$ \;
$Y = \sum_{i \in P} y_i p_i^* + \sum_{i \in D} g_i $ \nllabel{alg_Y}\;
$\hat{a} = \frac{1}{L} Y$ \;
$\hat{Q} = \frac{1}{L}\abs{Y}^2$ \nllabel{alg_Q} \;
\For{$k = 1,\dots,H$ \nllabel{alg_loop_search}}{
$Y = Y + (e^{j2\pi/m(k)} - 1) g_{\sigma(k)}$ \;
$g_{\sigma(k)} = e^{j2\pi/m(k)} g_{\sigma(k)} $\;
$Q = \frac{1}{L}\abs{Y}^2$\;
\If{$Q > \hat{Q}$}{
 	$\hat{Q} = Q$ \;
 	$\hat{a} =  \frac{1}{L} Y$ \;
 }
}
\Return{$\hat{a}$ \nllabel{alg_return}}
\end{algorithm}


\section{Simulations}\label{sec:simulations}

We present the results of Monte-Carlo simulations with the least squares estimator.  In all simulations the noise $w_1,\dots,w_L$ is independent and identically distributed circularly symmetric and Gaussian with real and imaginary parts having variance $\sigma^2$.  Under these conditions the least squares estimator is also the maximum likelihood estimator.  Simulations are run with $M=2,4,8$ (BPSK and QPSK and $8$-PSK) and with signal to noise ratio $\text{SNR} = \tfrac{\rho_0^2}{2\sigma^2}$ between \unit[-20]{dB} and \unit[20]{dB}.  The amplitude $\rho_0=1$ and $\theta_0$ is uniformly distributed on $[-\pi, \pi)$.  For each value of SNR, $T = 5000$ replications are performed to obtain $T$ estimates $\hat{\rho}_1, \dots, \hat{\rho}_T$ and $\hat{\theta}_1, \dots, \hat{\theta}_T$.  

Figures~\ref{fig:plotphaseBPSK},~\ref{fig:plotphaseQPSK}~and~\ref{fig:plotphase8PSK} show the mean square error (MSE) of the phase estimator when $M=2,4,8$ with $L=4096$ and for varying proportions of pilots symbols $\abs{P} = 0, \tfrac{L}{2}, \tfrac{L}{8}, \tfrac{L}{32}, L$.  When $\abs{P} \neq 0$ (i.e. coherent detection) the mean square error is computed as $\tfrac{1}{T}\sum_{i=1}^T\sfracpart{\hat{\theta}_i - \theta_0}_\pi^2$.  Otherwise, when $\abs{P}=0$ the mean square error is computed as $\tfrac{1}{T}\sum_{i=1}^T\sfracpart{\hat{\theta}_i - \theta_0}^2$ as in~\cite{McKilliam_leastsqPSKnoncoICASSP_2012}.  The dots, squares, circles and crosses are the results of Monte-Carlo simulations with the least square estimator.  The solid lines are the MSEs predicted a theoretical result not described here.  %The theory accurately predicts the behaviour of the phase estimator when $L$ is sufficiently large and when the SNR is not too small.  As the SNR decreases the variance of the phase estimator approaches that of the uniform distribution on $[-\pi, \pi)$ when $\abs{P} \neq 0$ and the uniform distribution on $[-\tfrac{\pi}{M}, \tfrac{\pi}{M})$ when $\abs{P}=0$~\cite{McKilliam_leastsqPSKnoncoICASSP_2012}. 

% Figures~\ref{fig:plotampBPSK},~\ref{fig:plotampQPSK}~and~\ref{fig:plotamp8PSK} show the variance of the amplitude estimator when $M=2,4,8$ and with $L=32, 256, 2048$ and when the number of pilots symbols is $\abs{P} = 0, \tfrac{L}{2}, L$.  The solid lines are the variance predicted by Theorem~\ref{thm:normality}.  The dots and crosses show the results of Monte-Carlo simulations.  Each point is computed as $\tfrac{1}{T}\sum_{i=1}^T\big(\hat{\rho}_i - \rho_0G(0)\big)^2$.  This requires $G(0)$ to be known.  In practice $G(0)$ may not be known at the receiver, so Figures~\ref{fig:plotampBPSK},~\ref{fig:plotampQPSK}~and~\ref{fig:plotamp8PSK} serve to validate the correctness of our asymptotic theory, rather than to suggest the practical performance of the amplitude estimator.  When SNR is large $G(0)$ is close to $1$ and the bias of the amplitude estimator is small.  However, $G(0)$ grows without bound as the variance of the noise increases, so the bias is significant when SNR is small.

Figure~\ref{fig:plotphaseQPSKmultL} shows the MSE of the phase estimator when $M=4$ and $L=32,256, 2048$ and the number of pilots is $\abs{P}=\tfrac{L}{8},L$.  The figure depicts an interesting phenomenon.  When $L=2048$ and $\abs{P} = \tfrac{L}{8} = 256$ the number of pilots symbols is the same as when $L=\abs{P} = 256$.  When the SNR is small (approximately less than \unit[0]{dB}) the least squares estimator using the $256$ pilots symbols and also the $2048-256=1792$ data symbols performs \emph{worse} than the estimator that uses only the $256$ pilots symbols.  A similar phenomenon occurs when $L=256$ and $\abs{P} = \tfrac{L}{8} = 32$.  %When the SNR is small it is better to ignore the data symbols and use only the pilot symbols.  
This behaviour suggests modifying the objective function to give the pilots symbols more importance when the SNR is low.  For example, rather the minimise~\eqref{eq:SSdefn} we could instead minimise a weighted version of it,
\[
SS_{\beta}(a, \{d_i, i \in D\}) = \sum_{i \in P} \abs{ y_i - a s_i }^2 + \beta \sum_{i \in D} \abs{ y_i - a d_i }^2,
\]
where the weight $\beta$ would be small when SNR is small and near $1$ when SNR is large.  %For fixed $\beta$ the asymptotic properties of this weighted estimator could be derived using the techniques we have developed in Sections~\ref{sec:stat-prop-least},~\ref{sec:proof-almost-sure} and~\ref{sec:proof-asympt-norm} and also in~\cite{McKilliam_leastsqPSKpilotsdata_2012appendix}.  This would enable a rigorous theory for selection of $\beta$ at the receiver.  One caveat is that the receiver would require knowledge about the noise distribution in order to advantageously choose $\beta$.  We do not investigate this further here.

% \begin{figure}[p]
% 	\centering
% 		\includegraphics[width=\linewidth]{code/data/plotM2-2.mps}
% 		\caption{Phase error versus SNR for BPSK with $L=4096$.}
% 		\label{fig:plotphaseBPSK}
% \end{figure}

% \begin{figure}[p]
% 	\centering
% 		\includegraphics[width=\linewidth]{code/data/plotM4-2.mps}
% 		\caption{Phase error versus SNR for QPSK with $L=4096$.}
% 		\label{fig:plotphaseQPSK}
% \end{figure}

% \begin{figure}[p]
% 	\centering
% 		\includegraphics[width=\linewidth]{code/data/plotM8-2.mps}
% 		\caption{Phase error versus SNR for $8$-PSK with $L=4096$.}
% 		\label{fig:plotphase8PSK}
% \end{figure}



% \begin{figure}[p]
% 	\centering
% 		\includegraphics[width=\linewidth]{code/data/plotM2-1.mps}
% 		\caption{Unbiased amplitude error versus SNR for BPSK.}
% 		\label{fig:plotampBPSK}
% \end{figure}

% \begin{figure}[p]
% 	\centering
% 		\includegraphics[width=\linewidth]{code/data/plotM4-1.mps}
% 		\caption{Unbiased amplitude error versus SNR for QPSK.}
% 		\label{fig:plotampQPSK}
% \end{figure}

% \begin{figure}[p]
% 	\centering
% 		\includegraphics[width=\linewidth]{code/data/plotM8-1.mps}
% 		\caption{Unbiased amplitude error versus SNR for $8$-PSK.}
% 		\label{fig:plotamp8PSK}
% \end{figure}

%\begin{figure}[tp]
%	\centering
%		\includegraphics[width=\linewidth]{code/data/plotM2-3.mps}
%		\caption{Phase error versus SNR for BPSK.}
%		\label{fig:plotphase}
%\end{figure}

% \begin{figure}[tp]
% 	\centering
% 		\includegraphics[width=\linewidth]{code/data/plotM4-3.mps}
% 		\caption{Phase error versus SNR for QPSK.}
% 		\label{fig:plotphaseQPSKmultL}
% \end{figure}


\section{Conclusion}

We considered least squares estimators of carrier phase and amplitude from noisy communications signals that contain both pilot signals, known to the receiver, and data signals, unknown to the receiver.  We focused on $M$-ary phase shift keying constellations.  The least squares estimator can be computed in $O(L\log L)$ operations using a modification of an algorithm due to Mackenthun~\cite{Mackenthun1994}, and is the maximum likelihood estimator in the case that the noise is additive white and Gaussian.  

%We showed, under some conditions on the distribution of the noise, that the phase estimator $\hat{\theta}$ is strongly consistent and asymptotically normally distributed.  However, the amplitude estimator $\hat{\rho}_0$ is biased, and converges to $G(0)\rho_0$.  This bias is large when the signal to noise ratio is small.  It would be interesting to investigate methods for correcting this bias.  A method for estimating $G(0)$ at the receiver appears to be required.

Monte Carlo simulations were used to assess the performance of the least squares estimator.  Interestingly, when the SNR is small, it is counter productive to use the data symbols to estimate the phase (Figure~\ref{fig:plotphaseQPSKmultL}).  This suggests the use of a weighed objective function, which would be an interesting topic for future research.



\small
\bibliography{bib}



 
\end{document}
